\embedWidthFigure{1.0}{../images/model-cartoons/model-choice-island-cartoon-mixed-compressed.pdf}{
    An illustration of the general phylogeographic model using three
    insular lizard taxa.
    The top two comparisons are pairs of populations for which we are
    interested in comparing their time of divergence.
    The bottom comparison is a single population for which we are interested in
    comparing the time of population-size change.
    In this scenario, three species of lizards co-occured on an island that was
    fragmented by a rise in sea levels at time \etime[1].
    Due to the fragmentation, the second lineage (from the top) diverged into
    two descendant populations while the population size of the third lineage
    was reduced.
    The first lineage diverged later at \etime[2] due to another mechanism,
    such as over-water dispersal.
    The five possible event models are shown to the right, with the
    correct model indicated.
    The event time parameters (\etime[1] and \etime[2])
    and the pair-specific event times (\comparisonetime[1],
    \comparisonetime[2], and \comparisonetime[3])
    are shown.
    The notation used in the text for the biallelic character data and
    effective sizes of the ancestral and descendant populations is shown for
    the third comparison.
    The lizard silhouette for the middle pair is from pixabay.com, and the
    other two are from phylopic.org; all were licensed under the Creative
    Commons (CC0) 1.0 Universal Public Domain Dedication.
}{fig:modelCartoon}
