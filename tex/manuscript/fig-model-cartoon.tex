\embedWidthFigure{1.0}{../images/model-cartoons/mixed-model-description.pdf}{
    An illustration of the general comparative model implemented in
    \ecoevolity.
    The top two comparisons are pairs of populations for which we are
    interested in comparing their time of divergence (``divergence
    comparisons'').
    The bottom comparison is a single population for which we are interested in
    comparing the time of population-size change (``demographic comparison'').
    % In this scenario, three species of lizards co-occured on an island that was
    % fragmented by a rise in sea levels at time \etime[1].
    % Due to the fragmentation, the second lineage (from the top) diverged into
    % two descendant populations while the population size of the third lineage
    % was reduced.
    % The first lineage diverged later at \etime[2] due to another mechanism,
    % such as over-water dispersal.
    With three comparisons, there are five possible event models
    \citep[i.e., five ways to assign the comparisons to anywhere from one to three event times;][]{Bell1934},
    which are shown to the right with the
    example model indicated.
    The event time
    (\etime[1] and \etime[2])
    and effective population size
    (\epopsize[\rootpopindex],
    \epopsize[\descendantpopindex{}])
    parameters are shown.
    Event times can be shared among comparisons, but each ancestral and
    descendant population has a unique effective population size.
    % and the pair-specific event times (\comparisonetime[1],
    % \comparisonetime[2], and \comparisonetime[3])
    % are shown.
    % The notation used in the text for the biallelic character data and
    % effective sizes of the ancestral and descendant populations is shown for
    % the third comparison.
    % The lizard silhouette for the middle pair is from pixabay.com, and the
    % other two are from phylopic.org; all were licensed under the Creative
    % Commons (CC0) 1.0 Universal Public Domain Dedication.
    % Modified from \citet{Oaks2018ecoevolity}.
}{fig:modelCartoon}
