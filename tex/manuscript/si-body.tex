\section{Methods}

\subsection{Initial simulation conditions}

We initially simulated data under distributions we hoped comprised a mix of
conditions that were favorable and challenging for estimating the timing and
sharing of demographic changes.
For these initial conditions, we simulated \datasets with three populations
that underwent a demographic change, under five different distributions on the
relative effective size of the ancestral population
(\rootrelativepopsize;
see Table~S\ref{table:prelimsimconditions} and left column of
\figs S\ref{fig:valsimsetimesinitial} and S\ref{fig:valsimsmodelinitial}),
which ranged from having a mean 4-fold population-size increase (\psimfourinc)
to a 2-fold decrease (\psimtwodec) and a ``worst-case'' scenario
where there was essentially no population-size change in the
history of the populations (\psimnochange).

\begin{table}[htbp]
    \sffamily
    \footnotesize
    % \rowcolors{2}{}{mygray}
    % \addtolength{\tabcolsep}{-0.1cm}
\captionsetup{name=Table S, labelformat=noSpace}
\caption{
    Simulation and analysis conditions for preliminary validation analyses.
    The distributions from which parameter values where drawn for simulating
    data with \simcoevolity are given for event times (\etime), the relative
    effective size of the root (ancestral) population (\rootrelativepopsize),
    and the effective size of the descendant population
    ($\epopsize[\descendantpopindex{}]\murate$),
    along with the prior distributions used for these parameters when the
    simulated data sets were analyzed with \ecoevolity.
    When the latter is represented by a dash, this means the prior distribution
    matched the distribution under which the data were simulated.
    $\textrm{G}(\cdots)$
    and
    $\textrm{E}(\cdots)$
    represent gamma and exponential distributions, respectively, and the first
    number provided for the gamma distributions is the shape parameter.
}
    \centering
    \begin{tabular}{ @{}llllllll@{} }
        \toprule
        & \multicolumn{3}{c}{\textbf{Simulated distribution}}
        &
        & \multicolumn{3}{c}{\textbf{Prior distribution}} \tn
        \cline{2-4}\cline{6-8} \tn[-1.2ex]
        Label
        & \multicolumn{1}{c}{\etime}
        & \multicolumn{1}{c}{\rootrelativepopsize}
        & \multicolumn{1}{c}{$\epopsize[\descendantpopindex{}]\murate$}
        &
        & \multicolumn{1}{c}{\etime}
        & \multicolumn{1}{c}{\rootrelativepopsize}
        & \multicolumn{1}{c}{$\epopsize[\descendantpopindex{}]\murate$}
        \tn
        \midrule
        \multicolumn{8}{c}{\textbf{\textit{Preliminary validation simulation conditions}}} \tn
        % \midrule
        \psimfourinc
        & \sdexponential{0.01}
        & \sdgamma{10}{0.25}
        & \sdgamma{5}{0.002}
        &
        & -
        & -
        & -
        \tn
        \psimtwoinc
        & \sdexponential{0.01}
        & \sdgamma{10}{0.5}
        & \sdgamma{5}{0.002}
        &
        & -
        & -
        & -
        \tn
        \psimtwodec
        & \sdexponential{0.01}
        & \sdgamma{10}{2}
        & \sdgamma{5}{0.002}
        &
        & -
        & -
        & -
        \tn
        \psimcentered
        & \sdexponential{0.01}
        & \sdgamma{10}{1}
        & \sdgamma{5}{0.002}
        &
        & -
        & -
        & -
        \tn
        \psimnochange
        & \sdexponential{0.01}
        & \sdgamma{100}{1}
        & \sdgamma{5}{0.002}
        &
        & -
        & -
        & -
        \tn
        \bottomrule
    \end{tabular}
    \label{table:prelimsimconditions}
\end{table}

% \begin{enumerate}[label=A.\arabic*]
%     \item \dgamma{10}{0.25} (4-fold population increase) \label{sims:initialFourFoldIncrease}
%     \item \dgamma{10}{0.5} (2-fold population increase)  \label{sims:initialTwoFoldIncrease}
%     \item \dgamma{10}{2} (2-fold population decrease)    \label{sims:initialTwoFoldDecrease}
%     \item \dgamma{10}{1} (no change on average, but a fair amount of variance) \label{sims:initialCenter}
%     \item \dgamma{100}{1} (no change on average, little variance) \label{sims:initialCenterNarrow}
% \end{enumerate}

For the mutation-scaled effective size of the descendant populations
($\epopsize[\descendantpopindex{}]\murate$; i.e., the population size after the
demographic change),
we used a gamma distribution with a shape of 5 and mean of 0.002
(Table~S\ref{table:prelimsimconditions}).
The timing of the demographic events was exponentially distributed with a mean
of 0.01 substitutions per site.
Taken together, the mean of the distribution on event times in units of $4N_e$
generations is approximately 1.56.
We chose this distribution in order to span times of demographic change from
very recent (i.e., most gene lineages coalesce before the change) to old (i.e.,
most gene lineages coalesce after the change),
which we assumed would include conditions under which the method performed both
well and poorly.
The assignment of the population-size change of the three simulated populations
to 1, 2, or 3 demographic events was controlled by a Dirichlet process with a
mean number of two events across the three populations.
We generated 500 \datasets under each of these five simulation conditions, all
of which were analyzed using the same simulated distributions as priors.

\section{Results}
Despite our attempt to capture a mix of favorable and challenging parameter
values,
estimates of the timing
(\fig{}~S\ref{fig:valsimsetimesinitial})
and sharing
(\fig{}~S\ref{fig:valsimsmodelinitial})
of demographic events were quite poor across all the simulation conditions
we initially explored.
Under the ``worst-case'' scenario of very little population-size change
(bottom row of \figs S\ref{fig:valsimsetimesinitial} and
S\ref{fig:valsimsmodelinitial}),
our method is unable to identify the timing or model of demographic change.
Under these conditions, our method returns the prior on the timing of events
(bottom row of \fig{}~S\ref{fig:valsimsetimesinitial})
and almost always prefers either a model with a single, shared demographic
event (model "000") or independent demographic changes (model "012";
bottom row of \fig{}~S\ref{fig:valsimsmodelinitial}).
This behavior is expected, because there is very little information in the
data about the timing of demographic changes, and a Dirichlet process with a
mean of 2.0 demographic events, puts approximately 0.24 of the prior
probability on the models with one and three events, and 0.515 prior
probability on the three models with two events (approximately 0.17 each).
As a result, with little information, the method samples from the prior
distribution on the timing of events, and prefers one of the two models with
the largest (and equal) prior probability.

Under considerable changes in population size, the method only fared
moderately better at estimating the timing of demographic events
(top three rows of \fig{}~S\ref{fig:valsimsetimesinitial}).
The ability to identify the model improved under these
conditions, but the frequency at which the correct model
was preferred only exceeded 50\% for the large population
expansions
(top three rows of \fig{}~S\ref{fig:valsimsmodelinitial}).
The median posterior support for the correct model was very small (less than
0.58) under all conditions.
Under all simulation conditions, estimates of the timing and sharing of
demographic events are better when using all characters, rather than only
variable characters
(second versus third column of \figs
S\ref{fig:valsimsetimesinitial}
and
S\ref{fig:valsimsmodelinitial}).
Likewise, we see better estimates of effective population sizes when using the
invariant characters
(\figs
S\ref{fig:valsimsasizesinitial}
and
S\ref{fig:valsimsdsizesinitial}).


We observed numerical problems when the time of the demographic change was
either very recent or old relative to the effective size of the population
following the change
(\epopsize[\descendantpopindex{}]; the descendant population).
In such cases, either very few or almost all of the sampled gene copies
coalesce after the demographic change,
providing almost no information about the magnitude or
timing of the population-size change.
In these cases, the data are well-explained by a constant population size,
which can be achieved by the model in three ways:
(1) an expansion time of zero and an ancestral population
size that matched the true population size,
(2) an old expansion and a descendant population size that matched the true
population size,
or (3) an intermediate expansion time and both the ancestral and descendant
sizes matched the true size.
The true population size being matched in these modelling conditions is that of
the descendant or ancestral population if the expansion was old or recent,
respectively.
This caused MCMC chains to converge to different regions of parameter
space
(highlighted in orange in \fig{}~S\ref{fig:valsimsetimesinitial}).
