\subsection{Initial simulation conditions}

We initially simulated data under distributions we hoped comprised a mix of
conditions that were favorable and challenging for estimating the timing and
sharing of demographic changes.
For these initial conditions, we simulated \datasets with three populations
that underwent a demographic change, under five different distributions on the
relative effective size of the ancestral population
(\rootrelativepopsize; see left column of
\figs
\labelcref{fig:valsimsmodelinitial,fig:valsimsetimesinitial}):
\begin{enumerate}[label=A.\arabic*]
    \item \dgamma{10}{0.25} (4-fold population increase) \label{sims:initialFourFoldIncrease}
    \item \dgamma{10}{0.5} (2-fold population increase)  \label{sims:initialTwoFoldIncrease}
    \item \dgamma{10}{2} (2-fold population decrease)    \label{sims:initialTwoFoldDecrease}
    \item \dgamma{10}{1} (no change on average, but a fair amount of variance) \label{sims:initialCenter}
    \item \dgamma{100}{1} (no change on average, little variance) \label{sims:initialCenterNarrow}
\end{enumerate}
The last distribution was chosen to represent a ``worst-case'' scenario where
there was almost no demographic change in the history of the populations.
For the mutation-scaled effective size of the descendant populations
($\epopsize[\descendantpopindex{}]\murate$; i.e., the population size after the
demographic change),
we used a gamma distribution with a shape of 5 and mean of 0.002.
The timing of the demographic events was exponentially distributed with a mean
of 0.01 expected substitutions per site;
given the mean of our distribution on the effective size of the descendant
populations, this puts the expectation of demographic-change times in units of
$4N_e$ generations at 1.25.
This distribution on times was chosen to span times of demographic change from
very recent (i.e., most gene lineages coalesce before the change) to old (i.e.,
most gene lineages coalesce after the change).
The assignment of the three simulated populations to 1, 2, or 3 demographic
events was controlled by a Dirichlet process with a mean number of 2.0
demographic events across the three populations.
We generated 500 \datasets under each of these five simulation conditions, all
of which were analyzed using the same simulated distributions as priors.
