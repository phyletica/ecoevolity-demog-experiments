Many biotic and abiotic factors that influence the distribution, abundance, and
diversification of species can simultaneously affect multiple species within or
across communities.
These include environmental changes and inter-specific ecological interactions
that cause the ranges of multiple, co-distributed species to contract, expand,
or fragment.
Such processes predict temporally clustered patterns of evolutionary events
across species, such as synchronous population divergences and/or changes in
population size.
This has generated interest in developing statistical methods that infer such
patterns from genetic data.
We introduce a general, full-likelihood Bayesian method that can estimate
temporal clustering of an arbitrary mix of population divergences and
population size changes.
We use this method to assess how well we can infer temporal patterns of shared
changes in effective population size when using all the information in genomic
data.
We find that estimating the timing and sharing of demographic changes is much
more challenging than divergence times.
Even under favorable simulation conditions, the ability to infer shared
demographic events is quite limited.
Our results also suggest that previous estimates of co-expansion among
stickleback populations are likely caused by lack of information in
single-nucleotide polymorphism (SNP) data that ignores invariant sites.
