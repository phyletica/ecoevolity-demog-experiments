Factors that influence the distribution, abundance, and diversification of
species can simultaneously affect multiple evolutionary lineages within or
across communities.
These include changes to the environment or inter-specific ecological
interactions that cause ranges of multiple species to contract, expand, or
fragment.
Such processes predict temporally clustered evolutionary events across species,
such as synchronous population divergences and/or changes in population size.
There have been a number of methods developed to infer shared divergences or
changes in population size, but not both, and the latter has been limited to
approximate methods.
We introduce a full-likelihood Bayesian method that uses genomic data to
estimate temporal clustering of an arbitrary mix of population divergences and
population-size changes across taxa.
Using simulated data, we find that estimating the timing and sharing of
demographic changes tends to be inaccurate and sensitive to prior assumptions,
which is in contrast to accurate, precise, and robust estimates of shared
divergence times.
We also show previous estimates of co-expansion among five Alaskan populations
of threespine sticklebacks (\spp{Gasterosteus aculeatus}) were likely driven by
prior assumptions and ignoring invariant characters. 
We conclude by discussing potential avenues to improve the estimation of
synchronous demographic changes across populations.
