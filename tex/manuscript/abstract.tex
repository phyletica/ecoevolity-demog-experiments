% Many
% biotic and abiotic
Factors that influence the distribution, abundance, and
diversification of species can simultaneously affect multiple evolutionary
lineages within or across communities.
These include environmental changes and inter-specific ecological interactions
that cause ranges of multiple, co-distributed species to contract,
expand, or become fragmented.
Such processes predict temporally clustered patterns of evolutionary events
across species, such as synchronous population divergences and/or changes in
population size.
% This has generated interest in developing statistical methods that infer such
% patterns from genetic data.
There have been a number of methods developed to infer shared divergences or
changes in effective population size, but not both, and the latter has been
limited to approximate Bayesian computation (ABC).
We introduce a general, full-likelihood Bayesian method that can use genomic
data to estimate temporal clustering of an arbitrary mix of population
divergences and population-size changes across taxa.
% We use this method to assess how well we can infer temporal patterns of shared
% population-size changes compared to divergences when using all the information
% in genomic data.
Applying this method to simulated data,
we find that estimating the timing and sharing of demographic changes is much
more challenging than divergences.
Even under favorable simulation conditions, the ability to infer shared
demographic events is quite limited and very sensitive to prior assumptions,
which is in sharp contrast to accurate, precise, and robust estimates of shared
divergence times.
Our results also suggest that previous estimates of co-expansion among five
Alaskan populations of threespine sticklebacks (\spp{Gasterosteus aculeatus})
were likely spurious, and driven by a combination of misspecified prior
assumptions and the lack of information about the timing of demographic changes
when invariant characters are ignored.
We conclude by discussing potential avenues to improve the estimation of
synchronous demographic changes across populations.
