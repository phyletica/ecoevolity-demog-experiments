\siFigure{0.8}{../../results/cropped-grid-ancestor-size-t-1_0-0_01-0_0.pdf}{
    \footnotesize
    The accuracy and precision of estimates of the effective size (scaled by
    the mutation rate) of the population
    before
    a demographic change
    (``ancestral''
    population)
    when data are simulated and analyzed under the same model (i.e., no model
    misspecification), and event times are exponentially distributed with a
    mean of 0.01
    (1.25 units of $4N_e$ generations).
    \validationsimsthreecolumndescription
    Each plotted circle and associated error bars represent the posterior mean
    and 95\% credible interval.
    Estimates for which the potential-scale reduction factor was greater than
    1.2 \citep{Brooks1998} are highlighted in orange.
    Each plot consists of 1500 estimates---500 simulated \datasets, each with
    three taxa.
    \accuracyscatterplotannotations{\epopsize\murate}
    \weusedmatplotlib
}{fig:valsimsasizesinitial}

\siFigure{0.8}{../../results/cropped-grid-descendant-size-t-1_0-0_01-0_0.pdf}{
    \footnotesize
    The accuracy and precision of estimates of the effective size (scaled by
    the mutation rate) of the population
    after
    a demographic change
    (``descendant''
    population)
    when data are simulated and analyzed under the same model (i.e., no model
    misspecification), and event times are exponentially distributed with a
    mean of 0.01
    (1.25 units of $4N_e$ generations).
    \validationsimsthreecolumndescription
    Each plotted circle and associated error bars represent the posterior mean
    and 95\% credible interval.
    Estimates for which the potential-scale reduction factor was greater than
    1.2 \citep{Brooks1998} are highlighted in orange.
    Each plot consists of 1500 estimates---500 simulated \datasets, each with
    three taxa.
    \accuracyscatterplotannotations{\epopsize\murate}
    \weusedmatplotlib
}{fig:valsimsdsizesinitial}


\siFigure{0.8}{../../results/cropped-grid-ancestor-size-t-4_0-0_000475-0_0001.pdf}{
    \footnotesize
    The accuracy and precision of estimates of the effective size (scaled by
    the mutation rate) of the population
    before
    a demographic change
    (``ancestral''
    population)
    when data are simulated and analyzed under the same model (i.e., no model
    misspecification),
    and event times are gamma-distributed with a shape of 4, offset of 0.0001,
    and mean of 0.002 (0.25 units of $4N_e$ generations).
    \validationsimsthreecolumndescription
    Each plotted circle and associated error bars represent the posterior mean
    and 95\% credible interval.
    Estimates for which the potential-scale reduction factor was greater than
    1.2 \citep{Brooks1998} are highlighted in orange.
    Each plot consists of 1500 estimates---500 simulated \datasets, each with
    three taxa.
    \accuracyscatterplotannotations{\epopsize\murate}
    \weusedmatplotlib
}{fig:valsimsasizesopt}

\siFigure{0.8}{../../results/cropped-grid-descendant-size-t-4_0-0_000475-0_0001.pdf}{
    \footnotesize
    The accuracy and precision of estimates of the effective size (scaled by
    the mutation rate) of the population
    after
    a demographic change
    (``descendant''
    population)
    when data are simulated and analyzed under the same model (i.e., no model
    misspecification),
    and event times are gamma-distributed with a shape of 4, offset of 0.0001,
    and mean of 0.002 (0.25 units of $4N_e$ generations).
    \validationsimsthreecolumndescription
    Each plotted circle and associated error bars represent the posterior mean
    and 95\% credible interval.
    Estimates for which the potential-scale reduction factor was greater than
    1.2 \citep{Brooks1998} are highlighted in orange.
    Each plot consists of 1500 estimates---500 simulated \datasets, each with
    three taxa.
    \accuracyscatterplotannotations{\epopsize\murate}
    \weusedmatplotlib
}{fig:valsimsdsizesopt}


\siFigure{1.0}{../../results/cropped-grid-ancestor-size-diffuse-prior.pdf}{
    The accuracy and precision of estimates of the
    effective size (scaled by the mutation rate)
    of the
    ancestral
    population
    of demographic comparisons
    (top two rows)
    versus divergence comparisons
    (bottom two rows)
    when the priors are correct
    (first and third rows)
    versus when the priors are diffuse
    (second and fourth rows).
    \diffusesimsfourcolumndescription
    Each plotted circle and associated error bars represent the posterior mean
    and 95\% credible interval.
    Estimates for which the potential-scale reduction factor was greater than
    1.2 \citep{Brooks1998} are highlighted in orange.
    Each plot comprises 500 simulated \datasets, each with three taxa.
    \accuracyscatterplotannotations{\epopsize\murate}
    The first row of plots are repeated from
    \fig{}~S\ref{fig:valsimsasizesopt}
    for comparison.
    \weusedmatplotlib
}{fig:valsimsasizesdiffuse}

\siFigure{1.0}{../../results/cropped-grid-descendant-size-diffuse-prior.pdf}{
    The accuracy and precision of estimates of the
    effective size (scaled by the mutation rate)
    of the
    descendant
    population(s)
    of demographic comparisons
    (top two rows)
    versus divergence comparisons
    (bottom two rows)
    when the priors are correct
    (first and third rows)
    versus when the priors are diffuse
    (second and fourth rows).
    \diffusesimsfourcolumndescription
    Each plotted circle and associated error bars represent the posterior mean
    and 95\% credible interval.
    Estimates for which the potential-scale reduction factor was greater than
    1.2 \citep{Brooks1998} are highlighted in orange.
    Each plot comprises 500 simulated \datasets, each with three taxa.
    \accuracyscatterplotannotations{\epopsize\murate}
    The first row of plots are repeated from
    \fig{}~S\ref{fig:valsimsdsizesopt}
    for comparison.
    \weusedmatplotlib
}{fig:valsimsdsizesdiffuse}


\siFigure{1.0}{../../results/cropped-grid-mixed-comparisons.pdf}{
    Analyses of six taxa comprising a mix of three populations that
    experienced a demogrpahic change and three pairs of populations that
    diverged.
    The performance of estimating the
    timing of events (Row 1),
    sharing of events (Rows 2--3),
    ancestral population size (Row 4),
    and descendant population size (Row 5)
    are shown separately for the three populations that experienced a
    demographic change (Columns 1 and 2) and the three pairs of populations
    that diverged (Columns 3 and 4).
    The plots of the demographic comparisons (Columns 1 and 2) are comparable
    to the second column of \figs
    \ref{fig:valsimsetimesopt},
    \ref{fig:valsimsmodelopt},
    S\ref{fig:valsimsdsizesopt},
    and
    S\ref{fig:valsimsasizesopt};
    the same priors on event times and ancestral population size were used.
    Estimates for which the potential-scale reduction factor was greater than
    1.2 \citep{Brooks1998} are highlighted in orange.
    Each plot shows the results from 500 simulated \datasets, each with
    six taxa.
    \weusedmatplotlib
}{fig:mixsimsfull}


\siFigure{0.9}{../../results/cropped-grid-loci.pdf}{
    \footnotesize
    Results of analyses of simulated \datasets
    with 5000 100-base-pair loci when using all characters (left column) or
    only unlinked variable characters (right column).
    The plots are comparable to the first row of \figs
    \ref{fig:valsimsetimesopt},
    \ref{fig:valsimsmodelopt},
    S\ref{fig:valsimsdsizesopt},
    and
    S\ref{fig:valsimsasizesopt};
    the models were identical,
    the only difference is the linkage of characters into loci.
    Estimates for which the potential-scale reduction factor was greater than
    1.2 \citep{Brooks1998} are highlighted in orange.
    Each plot shows the results from 500 simulated \datasets, each with
    six taxa.
    \weusedmatplotlib
}{fig:locisimsfull}


\siFigure{1.0}{../../results/stickleback-plots/cropped-grid-by-dpp-stickleback-sumevents.pdf}{
    The prior (light bars) and posterior (dark bars) probabilities of the number of
    demographic events across five stickleback populations
    when all of the sites (left column) or only variable sites (right column)
    of the RADseq alignments are analyzed.
    Each row shows results under a different prior on the
    concentration parameter of the dirichlet process.
    \weusedggplot
}{fig:sticklebydppevents}

\siFigure{1.0}{../../results/stickleback-plots/cropped-grid-by-dpp-stickleback-sumtimes.pdf}{
    Estimates of the time of a change in population size across five stickleback populations
    when all of the sites (left column) or only variable sites (right column)
    of the RADseq alignments are analyzed.
    Each row shows results under a different prior on the
    concentration parameter of the dirichlet process.
    \weusedmatplotlib
}{fig:sticklebydpptimes}

\siFigure{1.0}{../../results/stickleback-plots/cropped-grid-by-dpp-stickleback-sumsizes.pdf}{
    Estimates of the effective population size before (``ancestor'') and after
    a demographic change across five stickleback populations
    when all of the sites (left column) or only variable sites (right column)
    of the RADseq alignments are analyzed.
    Each row shows results under a different prior on the
    concentration parameter of the dirichlet process.
    \weusedmatplotlib
}{fig:sticklebydppsizes}



\siFigure{1.0}{../../results/stickleback-plots/cropped-grid-by-time-stickleback-sumevents.pdf}{
    The prior (light bars) and posterior (dark bars) probabilities of the number of
    demographic events across five stickleback populations
    when all of the sites (left column) or only variable sites (right column)
    of the RADseq alignments are analyzed.
    Each row shows results under a different prior on the
    timing of the change in population size.
    \weusedggplot
}{fig:sticklebytimeevents}

\siFigure{1.0}{../../results/stickleback-plots/cropped-grid-by-time-stickleback-sumtimes.pdf}{
    Estimates of the time of a change in population size across five stickleback populations
    when all of the sites (left column) or only variable sites (right column)
    of the RADseq alignments are analyzed.
    Each row shows results under a different prior on the
    timing of the change in population size.
    \weusedmatplotlib
}{fig:sticklebytimetimes}

\siFigure{1.0}{../../results/stickleback-plots/cropped-grid-by-time-stickleback-sumsizes.pdf}{
    Estimates of the effective population size before (``ancestor'') and after
    a demographic change across five stickleback populations
    when all of the sites (left column) or only variable sites (right column)
    of the RADseq alignments are analyzed.
    Each row shows results under a different prior on the
    timing of the change in population size.
    \weusedmatplotlib
}{fig:sticklebytimesizes}



\siFigure{1.0}{../../results/stickleback-plots/cropped-grid-by-size-stickleback-sumevents.pdf}{
    The prior (light bars) and posterior (dark bars) probabilities of the number of
    demographic events across five stickleback populations
    when all of the sites (left column) or only variable sites (right column)
    of the RADseq alignments are analyzed.
    Each row shows results under a different prior on the
    relative effective size of the ancestral population.
    \weusedggplot
}{fig:sticklebysizeevents}

\siFigure{1.0}{../../results/stickleback-plots/cropped-grid-by-size-stickleback-sumtimes.pdf}{
    Estimates of the time of a change in population size across five stickleback populations
    when all of the sites (left column) or only variable sites (right column)
    of the RADseq alignments are analyzed.
    Each row shows results under a different prior on the
    relative effective size of the ancestral population.
    \weusedmatplotlib
}{fig:sticklebysizetimes}

\siFigure{1.0}{../../results/stickleback-plots/cropped-grid-by-size-stickleback-sumsizes.pdf}{
    Estimates of the effective population size before (``ancestor'') and after
    a demographic change across five stickleback populations
    when all of the sites (left column) or only variable sites (right column)
    of the RADseq alignments are analyzed.
    Each row shows results under a different prior on the
    relative effective size of the ancestral population.
    \weusedmatplotlib
}{fig:sticklebysizesizes}
