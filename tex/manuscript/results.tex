\subsection{Analyses of simulated data}

\subsubsection{Estimating timing and sharing of demographic changes}

We initially simulated data under distributions we hoped comprised a mix of
conditions that were favorable and challenging for estimating the timing and
sharing of demographic changes.
However, estimates of the timing
(\fig{}~\ref{fig:valsimsetimesinitial})
and sharing
(\fig{}~\ref{fig:valsimsmodelinitial})
of demographic events was quite poor across all the simulation conditions.
For these initial conditions, we simulated \datasets with three populations
that underwent a demographic change, under a variety of distributions on the
relative size of the ancestral population.
These included four gamma probability distributions with a shape of 10 and a
mean of
0.25 (4-fold population increase),
0.5 (2-fold population increase),
2.0 (2-fold population decrease), and
1.0 (no change on average, but a fair amount of variance).
We also used a gamma distribution with a shape of 100 and mean of 1.0
to mimic ``worst-case'' conditions when there was almost no demographic
change in the history of the populations.
The timing of the demographic events was exponentially distributed with a mean
of 0.01 expected substitutions per site;
the expected descendant population size
(scaled by the mutation rate; \epopsize\murate)
was 0.002,
which puts our expectation for the size-change times in units of $4N_e$
generations at 1.25.
This distribution was chosen to span times of demographic change from very
recent (i.e., most gene lineages coalesce before the change) to old (i.e., most
gene lineages coalesce after the change).
The assignment of three simulated populations to these events was controlled by
a Dirichlet process with a mean number of 2 demographic events across the 3
populations.
% Our motivation for this distribution of times was to span values
% that were ``easy'' and ``hard'' to estimate;
% i.e., a mix of size changes that occurred more recently
% than the gene copies coalesced and thus would be generate data with information
% about when the change happened,
% and those that occurred when few or no gene lineages were left to coalesce, and
% thus would generate data with limited information about the timing of the
% event.

\ifembed{
\siFigure{0.8}{../../results/cropped-grid-event-time-t-1_0-0_01-0_0.pdf}{
    \footnotesize
    The accuracy and precision of time estimates of demographic changes (in units of
    expected subsitutions per site) when data are simulated and analyzed under
    the same model (i.e., no model misspecification), and event times are
    exponentially distributed with a mean of 0.01
    (1.25 units of $4N_e$ generations).
    \validationsimsthreecolumndescription
    Each plotted circle and associated error bars represent the posterior mean
    and 95\% credible interval.
    Estimates for which the potential-scale reduction factor was greater than
    1.2 \citep{Brooks1998} are highlighted in orange.
    Each plot consists of 1500 estimates---500 simulated \datasets, each with
    three demographic comparisons.
    \accuracyscatterplotannotations{\comparisonetime{}}
    \weusedmatplotlib
}{fig:valsimsetimesinitial}

}{}

\ifembed{
\siFigure{0.8}{../../results/cropped-grid-model-t-1_0-0_01-0_0.pdf}{
    \scriptsize
    The performance of estimating the model of demographic changes when data
    are simulated and analyzed under the same model (i.e., no model
    misspecification), and event times are exponentially distributed with a
    mean of 0.01
    (1.56 units of $4N_e$ generations).
    \validationsimsthreecolumndescriptionmodels
    Each plot shows the results of the analyses of 500 simulated \datasets,
    each with three demographic comparisons;
    the number of \datasets that fall within each possible cell
    of true versus estimated model is shown, and cells with
    more \datasets are shaded darker.
    \modelplotannotations
    \weusedmatplotlib
}{fig:valsimsmodelinitial}

}{}

Under the ``worst-case'' scenario of essentially no population-size change
(bottom row of \figs \ref{fig:valsimsetimesinitial} and
\ref{fig:valsimsmodelinitial}),
our method is unable to identify the timing or model of demographic change.
As expected, our method returns the prior on the timing of events 
(bottom row of \fig{}~\ref{fig:valsimsetimesinitial})
and almost always prefers either a model with a single, shared demographic
event (model "000") or independent demographic changes (model "012").
(bottom row of \fig{}~\ref{fig:valsimsmodelinitial}).
This is expected behavior, because there is essentially no information in the
data about the timing of demographic changes, and a Dirichlet process with a
mean of 2 demographic events, puts approximately 0.24 of the prior probability
on the models with one and three events, and 0.515 prior probability on the
three models with two events (approximately 0.17 each).
Thus, with little information, the method samples from the prior distribution
on the timing of events, and prefers one of the two models with larger prior
probability.

Under considerable changes in population size, the method only faired
moderately better at estimating the timing of demographic events
(top three rows of \fig{}~\ref{fig:valsimsetimesinitial}).
The ability to identify the model improved under these
conditions, but the frequency at which the correct model
was preferred only exceeded 50\% for the large population
expansions
(top three rows of \fig{}~\ref{fig:valsimsmodelinitial}).
The median posterior support for the correct model was very small (less than
0.58) under all conditions
Under all simulation conditions, estimates of the timing and sharing of
demographic events are better when using all characters, rather than only
variable characters
(second versus third column of \figs
\ref{fig:valsimsetimesinitial}
and
\ref{fig:valsimsmodelinitial}).
Likewise, we see better estimates of effective population sizes when using the
invariant characters
(\figs
S\ref{fig:valsimsasizesinitial}
and
S\ref{fig:valsimsdsizesinitial}).


We observed numerical problems when the change in population size was
either very recent or old relative to the effective size of the
population after the demographic change;
we refer to the population before the change as ``ancestral'' and after as
``descendant''.
In such cases, either very few or almost all of the gene lineages coalesce
after the demographic change,
providing almost no information about the magnitude or
timing of the population-size change.
In these cases, the data were well-explained by no change in population size,
which could be achieved in three ways:
(1) an expansion time of zero and an ancestral population
size that matched the true population size,
(2) an old expansion and a descendant population size that matched the true
population size,
or (3) an intermediate expansion time and both the ancestral and descendant
sizes matched the true time.
The true population size being matched was that of the descendant or ancestral
population if the expansion was old or recent, respectively.
This caused MCMC chains to converge to different regions of parameter
space
(highlighted in orange in \fig{}~\ref{fig:valsimsetimesinitial}).

In an effort to find conditions under which the timing and sharing of
demographic changes could be better estimated, and avoid combinations
of parameter values that caused identifiability problems,
we next explored simulations under distributions on times and population sizes
offset from zero, but with much more recent demographic event times.
% Event time ~ gamma(shape=4.0, scale=0.000475, offset=0.0001); mean 0.002
% relative root size ~ gamma(shape=5.0, scale = 0.04, offset = 0.05); mean 0.25
% relative root size ~ gamma(shape=5.0, scale = 0.09, offset = 0.05); mean 0.5
% relative root size ~ gamma(shape=5.0, scale = 0.79, offset = 0.05); mean 4
% relative root size ~ gamma(shape=5.0, scale = 0.19, offset = 0.05); mean 1
% relative root size ~ gamma(shape=50.0, scale = 0.02, offset = 0.0); mean 1
For the distribution of event times, we used a gamma distribution
with a shape of 4, offset of 0.0001, and a mean of 0.002 (accounting
for the offset; 0.25 units of $4N_e$ generations, on average).
In combination with this distribution on event times,
we used five different gamma distributions on the relative
effective size of the ancestral population.
Four of these had a shape of 5, an offset of 0.05, and a mean
of
0.25 (4-fold increase),
0.5 (2-fold increase),
4 (4-fold decrease),
and
1 (no change on average).
The last gamma distribution on the relative ancestral size had a shape of 50
and a mean of 1.

Even when trying to select simulation conditions that are favorable for
identifying the event times, the ability to infer the correct timing and
sharing of demographic events remains quite poor
(\figs \ref{fig:valsimsetimesopt} and \ref{fig:valsimsmodelopt}).
Under the recent (but not too recent) 4-fold population-size increase (on
average) scenario, we do see better estimates of the times of the demographic
change
(top row of \fig{}~\ref{fig:valsimsetimesopt}),
but the ability to identify the correct number of events and the assignment of
the populations to those events remains quite poor;
\ecoevolity only prefers the correct model 57\% of the time, with
a median posterior probability of the correct model of 0.42
(top row of \fig{}~\ref{fig:valsimsmodelopt}).
Under the most extreme population retraction scenario (4-fold, on average),
\ecoevolity only estimates the correct model 40\% of the time, with a median
posterior probability of the correct model of only 0.26
(middle row of \fig{}~\ref{fig:valsimsmodelopt}).
Estimates are especially poor when using only variable characters,
so we focus on the results using all characters
(second versus third column of \figs
\ref{fig:valsimsetimesinitial}
and
\ref{fig:valsimsmodelinitial});
we also see worse estimates of population sizes when excluding invariant
characters
(\figs
S\ref{fig:valsimsasizesopt}
and 
S\ref{fig:valsimsdsizesopt}).

\ifembed{
\embedHeightFigure{0.8}{../../results/cropped-grid-event-time-t-4_0-0_000475-0_0001.pdf}{
    \footnotesize
    The accuracy and precision of time estimates of demographic changes (in
    units of expected subsitutions per site) when data were simulated and
    analyzed under the same distributions (Table~\ref{table:simconditions}).
    % , and event
    % times are gamma-distributed with a shape of 4, offset of 0.0001, and mean
    % of 0.002
    % (0.3 units of $4N_e$ generations).
    \validationsimsthreecolumndescription
    Each plotted circle and associated error bars represent the posterior mean
    and 95\% credible interval.
    Estimates for which the potential-scale reduction factor was greater than
    1.2 \citep{Brooks1998} are highlighted in orange.
    Each plot consists of 1500 estimates---500 simulated \datasets, each with
    three demographic comparisons.
    \accuracyscatterplotannotations{\comparisonetime{}}
    \weusedmatplotlib
}{fig:valsimsetimesopt}

}{}

\ifembed{
\embedHeightFigure{0.8}{../../results/cropped-grid-model-t-4_0-0_000475-0_0001.pdf}{
    \footnotesize
    The performance of estimating the model of demographic changes when data
    are simulated and analyzed under the same distributions
    (Table~\ref{table:simconditions}).
    % misspecification), and event times are
    % gamma-distributed with a shape of 4, offset of 0.0001, and mean of 0.002
    % (0.3 units of $4N_e$ generations).
    \validationsimsthreecolumndescriptionmodels
    Each plot shows the results of the analyses of 500 simulated \datasets,
    each with three demographic comparisons;
    the number of \datasets that fall within each possible cell
    of true versus estimated model is shown, and cells with
    more \datasets are shaded darker.
    \modelplotannotations
    \weusedmatplotlib
}{fig:valsimsmodelopt}

}{}


\subsubsection{Simulations to assess sensitivity to prior assumptions}

We observe the best estimates of the timing and sharing of demographic
events under narrow distributions on the relative effective size of the
ancestral population
(top row of \figs
\labelcref{fig:valsimsetimesinitial,fig:valsimsmodelinitial,fig:valsimsetimesopt,fig:valsimsmodelopt}),
which were used to both simulate the data and as the prior
when those data were analyzed.
These distributions are unrealistically informative priors for empirical
studies, for which there is usually little \emph{a priori} information about
past population sizes.
To determine whether the better performance under these distributions was
caused by more informative data or priors, we simulated \datasets under a
narrow distribution on the relative ancestral population size, and analyzed
these \datasets under more realistic, ``diffuse'' prior distributions on
populations sizes and event times.
This will also allow us to assess how sensitive estimates of the timing and
sharing of demographic events are to prior assumptions.
For comparison, we repeated the same simulations for pairs of populations for
which we estimated the timing and sharing of their divergences (the same total
number of gene copies were sampled from each pair).

% Sim disributions:
% Event time ~ gamma(shape=4.0, scale=0.000475, offset=0.0001); mean 0.002
% relative root size ~ gamma(shape=5.0, scale = 0.04, offset = 0.05); mean 0.25
% descendant size ~ gamma(4.0, scale=0.0005, offset = 0.0001); mean 0.0021
% Priors:
% Event time ~ exponential(mean = 0.005)
% relative root size ~ Exponential(mean = 2.0)
% descendant size ~ gamma(2.0, scale 0.001); mean 0.002

We find that the inference of shared demographic events is quite sensitive to
the priors, and we see a strong bias toward underestimating the number of
events under the more realistic diffuse priors
(top two rows of \fig{}~\ref{fig:valsimsmodeldiffuse}).
Under the diffuse priors, the probability of inferring the correct model of
demographic events decreases from 0.57 to 0.434 when all characters are used,
and from 0.36 to 0.284 when only variable characters are used.
The median posterior probability of the correct model also decreases from
0.422 to 0.292 when all characters are used,
and from 0.231 to 0.178 when only variable characters are used
(top two rows of \fig{}~\ref{fig:valsimsmodeldiffuse}).
In comparison, the inference of shared divergences is much more
accurate and precise than shared demographic changes, and
the former is much more robust to the diffuse priors
(bottom two rows of \fig{}~\ref{fig:valsimsmodeldiffuse}).
When all characters are used, under both the correct and diffuse
priors, the correct model is preferred over 91\% of the time,
and the median posterior probability of the correct model is over
0.93.

\ifembed{
\embedWidthFigure{1.0}{../../results/cropped-grid-model-diffuse-prior.pdf}{
    The performance of estimating the model of demographic changes
    (top two rows)
    versus model of divergences
    (bottom two rows)
    when the priors are correct
    (first and third rows)
    versus when the priors are diffuse
    (second and fourth rows).
    \diffusesimsfourcolumndescriptionmodels
    Each plot shows the results of the analyses of 500 simulated \datasets,
    each with three
    demographic comparisons (Rows 1--2) or
    divergence comparisons (Rows 3--4);
    the number of \datasets that fall within each possible cell
    of true versus estimated model is shown, and cells with
    more \datasets are shaded darker.
    \modelplotannotations
    The first row of plots are repeated from
    \fig{}~\ref{fig:valsimsmodelopt}
    for comparison.
    \weusedmatplotlib
}{fig:valsimsmodeldiffuse}

}{}

The results are similar when we look at the estimates of the timing of
demographic changes
(\fig{}~\ref{fig:valsimsetimesdiffuse}).
The precision of time estimates of demographic changes decreases
substantially under the diffuse priors
(top two rows of \fig{}~\ref{fig:valsimsetimesdiffuse}),
whereas the precision of the divergence-time estimates
is high and largely unchanged under the diffuse priors
(bottom two rows of \fig{}~\ref{fig:valsimsetimesdiffuse}).
We see the same patterns in the estimates of population sizes
(\figs
S\ref{fig:valsimsasizesdiffuse}
and
S\ref{fig:valsimsdsizesdiffuse})

\ifembed{
\embedWidthFigure{1.0}{../../results/cropped-grid-event-time-diffuse-prior.pdf}{
    The accuracy and precision of time estimates of demographic changes
    (top two rows)
    versus divergences
    (bottom two rows)
    when the priors are correct
    (first and third rows)
    versus when the priors are diffuse
    (second and fourth rows).
    Time is measured in units of expected subsitutions per site. 
    \diffusesimsfourcolumndescription
    Each plotted circle and associated error bars represent the posterior mean
    and 95\% credible interval.
    Estimates for which the potential-scale reduction factor was greater than
    1.2 \citep{Brooks1998} are highlighted in orange.
    Each plot consists of 1500 estimates---500 simulated \datasets, each with
    three
    demographic comparisons (Rows 1--2) or
    divergence comparisons (Rows 3--4).
    \accuracyscatterplotannotations{\comparisonetime{}}
    The first row of plots are repeated from
    \fig{}~\ref{fig:valsimsetimesopt}
    for comparison.
    \weusedmatplotlib
}{fig:valsimsetimesdiffuse}

}{}


\subsection{Inferring a mix of shared divergences and demographic changes}

When demographic and divergence comparisons are analyzed separately, the
performance of estimates of the timing and sharing of demographic changes and
divergences is dramatically different with the latter being much more accurate
and precise than the former
(e.g., see
\figs
\ref{fig:valsimsmodeldiffuse}
and
\ref{fig:valsimsetimesdiffuse}).
Perhaps, if we analyze a mix of demographic and divergence comparisons,
the informativeness of the divergence times can help ``anchor'' and
improve the estimates of shared demographic changes.
To explore this possibility, we simulated datasets with six comparisons,
comprising a mix of three populations that experienced a demographic change and
three pairs of populations that diverged.
For these mixed-comparison simulations, we used a gamma distribution on event
times with a shape of 4, offset of 0.0001, and a mean of 0.002 substitutions per
site (accounting for the offset; 0.25 units of $4N_e$ generations on average).
For the distribution on the relative size of the ancestral population,
we used a gamma distribution with a shape of 5, an offset of 0.05, and a mean
of 0.5; a 2-fold population size increase on average.
These are the same distributions used for the second row of
\figs
\ref{fig:valsimsetimesopt}
and
\ref{fig:valsimsmodelopt}.

% We summarized the timing and sharing of the demographic changes
% separately from the divergences so that we could determine whether the
% divergence-time estimates could help improve the estimates of the
% times of the demographic changes.
When analyzing a mix of demographic and divergence comparisons, the ability to
infer the timing and sharing of demographic changes remains poor, whereas
estimates of shared divergences remain accurate and precise
(\fig{}~\ref{fig:mixsims}).
The estimates of the timing and sharing of demographic events are very similar
to when we simulated and analyzed only three demographic comparisons under the
same distributions on event times and population sizes
(compare left column of \fig{}~\ref{fig:mixsims}
to the second row of \figs
\ref{fig:valsimsetimesopt}
and
\ref{fig:valsimsmodelopt}).
The same is true for the estimates of population sizes
(\fig{}~S\ref{fig:mixsimsfull}).
Thus, there is no mechanism by which the more informative divergence-time
estimates ``rescue'' the estimates of the timing and sharing of the demographic
changes.

\ifembed{
\embedWidthFigure{1.0}{../../results/cropped-grid-mixed-comparisons-allsites-short.pdf}{
    Analyses of six taxa comprising a mix of three populations that
    experienced a demogrpahic change and three pairs of populations that
    diverged.
    The performance of estimating the timing (top row) and sharing (bottom row)
    of events are shown separately for the three populations that experienced a
    demographic change (left column) and the three pairs of populations that
    diverged (right column).
    The plots of the demographic comparisons (left column) are comparable
    to the second column of \figs
    \ref{fig:valsimsetimesopt}
    and
    \ref{fig:valsimsmodelopt};
    the same priors on event times and ancestral population size were used.
    Time estimates for which the potential-scale reduction factor was greater than
    1.2 \citep{Brooks1998} are highlighted in orange.
    Each plot shows the results from 500 simulated \datasets, each with
    six taxa.
    % \accuracyscatterplotannotations{\comparisonetime{}}
    \weusedmatplotlib
}{fig:mixsims}

}{}


\subsection{Simulating linked sites}

Most reduced-representation genomic datasets are comprised of loci of
contiguous, linked nucleotides.
Thus, when using the method presented here that assumes each character is
effectively unlinked (i.e., evolved along a gene tree that is independent from
other characters, conditional on the population history), one either has to
violate this assumption, or discard all but (at most) one site per locus.
Given that all the results above indicate better estimates when all
characters are used (compared to using only variable characters), we
performed simulations to determine which strategy is better:
analyzing all linked sites and violating the assumption of unlinked characters,
or discarding all but (at most) one variable character per locus.

To do this, we repeated the most favorable simulation conditions (on average
4-fold population expansion; see first row of
\figs
\ref{fig:valsimsetimesopt}
and
\ref{fig:valsimsmodelopt}),
except that 100 characters were simulated along 5000
simulated gene trees.
In other words, the simulated \datasets comprised 5000
100-base-pair loci, rather than 500,000 unlinked sites.
The results are almost identical to when all the sites were unlinked
(compare first row of
\figs
\ref{fig:valsimsetimesopt}
and
\ref{fig:valsimsmodelopt}
to
\fig{}~\ref{fig:locisims},
and the first row of
\figs
S\ref{fig:valsimsasizesopt}
and
S\ref{fig:valsimsdsizesopt}
to the bottom two rows of
\fig{}~S\ref{fig:locisimsfull}).
Thus, violating the assumption of unlinked sites has little
affect on the estimation of the timing and sharing of
demographic changes;
this is also true for estimates of population sizes
(\fig{}~S\ref{fig:locisimsfull}).
This is consistent with the findings of
\citet{Oaks2018ecoevolity} and
\citet{Oaks2018paic}
that the violation of linked sites had little affect on the estimation of
shared divergence times.
These results suggest that analyzing all of the sites in loci assembled from
reduced-representation genomic libraries (e.g., sequence-capture or RADseq
loci) is a better strategy than excluding sites to avoid violating the
assumption of unlinked characters.

\ifembed{
\siFigure{1.0}{../../results/cropped-grid-loci-short.pdf}{
    Estimates of the timing (top row) and sharing (bottom row) of demographic
    changes when using all characters (left column) or only unlinked variable
    characters (right column) from \datasets simulated with 5000 loci of 100
    linked bases from three demographic comparisons.
    The plots are comparable to the first row of \figs
    \ref{fig:valsimsetimesopt}
    and
    \ref{fig:valsimsmodelopt};
    the only difference is the linkage of characters into loci.
    Time estimates for which the potential-scale reduction factor was greater than
    1.2 \citep{Brooks1998} are highlighted in orange.
    Each plot shows the results from 500 simulated \datasets, each with
    three demographic comparisons.
    \weusedmatplotlib
}{fig:locisims}

}{}


% \subsection{Data-acquisition bias?}


\subsection{Reassessing the co-expansion of stickleback populations}

Using an ABC analog to the model of shared demographic changes developed here,
\citet{Xue2015} found very strong support (0.99 posterior probability) that
five populations of threespine sticklebacks (\spp{Gasterosteus aculeatus})
from south central Alaska recently
co-expanded.
This inference was based on the publicly available RADseq data collected by
\citet{Hohenlohe2010}
(\url{https://trace.ncbi.nlm.nih.gov/Traces/sra/?study=SRP001747};
NCBI Short Read Archive accession numbers SRX015871--SRX015877).
We re-assembled and analyzed these data under our full-likelihood
Bayesian framework, both using all sites from assembled loci,
and only SNPs.

When analyzing all sites from the assembled stickleback
RADseq data, we find strong support for five independent
population expansions (no shared demographic events;
\fig{}~\ref{fig:sticklesummary}).
In sharp contrast, when analyzing only SNPs, we find
support for a single, shared, extremely recent population expansion
(\fig{}~\ref{fig:sticklesummary}).
The support for a single, shared event is consistent with the results from our
simulations using diffuse priors and only including SNPs, which showed
consistent, spurious support for a single event
(Row 2 of \fig{}~{fig:valsimsmodeldiffuse}).
These results are relatively robust to a broad range of prior
assumptions
(\figs
S\labelcref{%
fig:sticklebydppevents,fig:sticklebydpptimes,fig:sticklebydppsizes,%
fig:sticklebytimeevents,fig:sticklebytimetimes,fig:sticklebytimesizes,%
fig:sticklebysizeevents,fig:sticklebysizetimes,fig:sticklebysizesizes,%
}).

\ifembed{
\mFigure{1.0}{../../results/stickleback-plots/cropped-grid-stickleback-summary.pdf}{
    Estimates of the number (Row 1), timing (Row 2), and magnitude (Row 3)
    of demographic events across five stickleback populations, when using all
    sites (left column) or only variable sites (right column) and an exponential with mean of 0.001 as the prior on event times, an exponential with mean of 1 for the prior on the relative ancestral effective population size, and a gamma with shape of 2 and mean of 0.002.
    Analyses of six taxa comprising a mix of three populations that
    experienced a demogrpahic change and three pairs of populations that
    diverged.
    For the number of events (Row 1), the light and dark bars represent the
    prior and posterior probabilities, respectively.
    Time (Row 2) is in units of expected subsitutions per site.
    For the violin plots, each plotted circle and associated error bars
    represent the posterior mean and 95\% credible interval.
    Bar graphs were generated with ggplot2 Version 2.2.1 \citep{ggplot2};
    violin plots were generated with matplotlib Version 2.0.0
    \citep{matplotlib}.
}{fig:sticklesummary}

}{}

When using only SNPs, the estimates of the shared demographic event time are
essentially at the minimum of zero, suggesting that there is little information
about the timing of any demographic changes in the SNP data alone.
This is consistent with results of \citet{Xue2015} where the single, shared
event was also estimated to have occurred at the minimum (1000 generations) of
their uniform prior on the timing of demographic changes.
Their and our results based solely on SNPs seem to be an artifact of the lack
of information in the SNP-only data.
Based on our simulation results, our estimates using all of the sites
in the stickleback RADseq loci should be the most accurate.
However, the unifying theme of our simulation results is that all estimates of
shared demographic events tend to be poor and should be treated with
a lot of skepticism.
Given our results when using all the information in data simulated under
conditions favorable for estimating the timing of demographic changes, support
of 0.99 posterior probability for any particular scenario is almost certainly
spurious.
