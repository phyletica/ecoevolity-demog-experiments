\section{Introduction}

Many processes that shape the distribution of biodiversity across a landscape
are expected to affect multiple species.
Examples include changes to the environment itself that can cause
co-distributed species to contract or expand their ranges or become isolated.
Furthermore, ecological interactions can also cause demographic changes in
multiple species;
for example, the expansion of one species may predict the expansion of its
symbionts and the contraction of competitors.
Such processes predict that population divergences or demographic changes will
be temporally clustered across multiple species.
As a result, statistical methods that infer such patterns from genetic data
allow phylogeographers to test hypotheses about such processes.
There has been a lot of work on methods that test for shared divergence times
among pairs of populations
\citep{Hickerson2006,Hickerson2007,Huang2011,Oaks2014dpp,Oaks2018ecoevolity}.
Recently, there has also been a lot of work on methods that infer temporally
clustered changes in population sizes among species
\citep{Chan2014,Xue2015,Xue2017,Gehara2017}.

Uses of ABC estimation of shared size changes \citep{Burbrink2016}

To date, no method has allowed the co-estimation of both shared divergences
and population-size changes.
Given the overlap in the processes that can potentially cause divergence and
demographic changes across multiple species, such a method would be useful for
testing hypotheses about community-scale processes that shape biodiversity
across landscapes.
Here, we introduce a general, full-likelihood Bayesian method that can estimate
shared times among an arbitrary mix of population divergences and population
size changes (\fig{}~\ref{fig:modelCartoon}).

Furthermore, there has been a lot of work on how well the divergence-time
methods work
\citep{Oaks2012,Hickerson2013,Oaks2014reply,Oaks2014dpp,Overcast2017,Oaks2018ecoevolity}.
However, work on how well we can estimate shared population-size changes has
been much more limited.
There are theoretical reasons to suspect that estimating shared demographic
changes is more difficult than divergence times.
The parameter of interest can become unidentifiable in at least three ways.
First, as the magnitude of a change in population becomes smaller,
it becomes more difficult to estimate.
Second, as the age of the size change gets older, we run out of
genetic coalescent events that allow us to learn when it happened.
Third, as the age of the size change approaches zero, we once again will fail
to sample the coalescent events needed to inform us of the change.
Thus, our second goal is to leverage our new method to assess how well we can
infer shared population size changes among species.

\ifembed{
\embedWidthFigure{1.0}{../images/model-cartoons/model-choice-island-cartoon-mixed-simple-comp-labels-compressed.pdf}{
    An illustration of the general comparative model using three insular lizard
    taxa.
    The top two comparisons are pairs of populations for which we are
    interested in comparing their time of divergence (``divergence
    comparisons'').
    The bottom comparison is a single population for which we are interested in
    comparing the time of population-size change (``demographic comparison'').
    In this scenario, three species of lizards co-occured on an island that was
    fragmented by a rise in sea levels at time \etime[1].
    Due to the fragmentation, the second lineage (from the top) diverged into
    two descendant populations while the population size of the third lineage
    was reduced.
    The first lineage diverged later at \etime[2] due to another mechanism,
    such as over-water dispersal.
    The five possible event models are shown to the right, with the
    correct model indicated.
    % The event time parameters (\etime[1] and \etime[2])
    % and the pair-specific event times (\comparisonetime[1],
    % \comparisonetime[2], and \comparisonetime[3])
    % are shown.
    % The notation used in the text for the biallelic character data and
    % effective sizes of the ancestral and descendant populations is shown for
    % the third comparison.
    The lizard silhouette for the middle pair is from pixabay.com, and the
    other two are from phylopic.org; all were licensed under the Creative
    Commons (CC0) 1.0 Universal Public Domain Dedication.
    Modified from \citet{Oaks2018ecoevolity}.
}{fig:modelCartoon}

}{}


\section{Methods}

\subsection{The data}
As described by \citet{Oaks2018ecoevolity},
we assume we have orthologous, biallelic genetic characters collected from
taxa we wish to compare.
By biallelic, we mean that each character has at most two states,
which we refer to as ``red'' and ``green'' following \citet{Bryant2012}.
For each taxon, we either have these data from one or more individuals from a
single population, in which case we infer the timing and extent of a population
size change, or one or more individuals from two populations or species,
in which case we infer the time when they diverged (\fig{}~\ref{fig:modelCartoon}).

For each population and for each character we genotype \allelecount copies
of the locus, \redallelecount of which are copies of the red allele and the
remaining $\allelecount - \redallelecount$ are copies of the green allele.
Thus, for each population of a pair, and for each locus, we have a count of the
total sampled gene copies and how many of those are the red allele.

\ifembed{
\newcommand{\given}{\ensuremath{\,|\,}\xspace}
\newcommand{\pr}{\ensuremath{p}}

\newcommand{\data}{\ensuremath{D}\xspace}
\newcommand{\model}[1][]{\ensuremath{M_{#1}}\xspace}
\newcommand{\parameters}[1][]{\ensuremath{\Theta_{#1}}\xspace}
\newcommand{\parameter}[1][]{\ensuremath{\theta_{#1}}\xspace}
\newcommand{\diff}[1]{\ensuremath{\mathrm{d}#1}}

\newcommand{\distgamma}{\ensuremath{\textrm{Gamma}}\xspace}
\newcommand{\distexponential}{\ensuremath{\textrm{Exponential}}\xspace}
\newcommand{\dgamma}[2]{\ensuremath{\distgamma(\textrm{shape} = #1, \textrm{mean} = #2)}}
\newcommand{\dogamma}[3]{\ensuremath{\distgamma(\textrm{shape} = #1, \textrm{offset} = #3, \textrm{mean} = #2)}}
\newcommand{\dexponential}[1]{\ensuremath{\distexponential(\textrm{mean} = #1)}}

% \newcommand{\ncomparisons}{\ensuremath{N\xspace}}
\newcommand{\ncomparisons}{\ensuremath{\mathcal{N}\xspace}}
\newcommand{\nevents}[1][]{\ensuremath{k_{#1}\xspace}}
\newcommand{\nloci}[1][]{\ensuremath{m_{#1}\xspace}}

\newcommand{\observedallelecount}[1][]{\ensuremath{n_{#1}}\xspace}
\newcommand{\observedredallelecount}[1][]{\ensuremath{r_{#1}}\xspace}

\newcommand{\nodeallelecount}[2]{\ensuremath{n_{#1}^{#2}}}
\newcommand{\noderedallelecount}[2]{\ensuremath{r_{#1}^{#2}}}

\newcommand{\allelecount}[1][]{\ensuremath{\nodeallelecount{#1}{}}\xspace}
\newcommand{\redallelecount}[1][]{\ensuremath{\noderedallelecount{#1}{}}\xspace}

\newcommand{\leafallelecounts}[1][]{\ensuremath{\mathbf{n}_{#1}}\xspace}
\newcommand{\leafredallelecounts}[1][]{\ensuremath{\mathbf{r}_{#1}}\xspace}
\newcommand{\maxleafallelecounts}{\ensuremath{\textrm{max}(\mathbf{n})}\xspace}

\newcommand{\comparisondata}[1][]{\ensuremath{D_{#1}}\xspace}
\newcommand{\alldata}[1][]{\ensuremath{\mathbf{D}}\xspace}

\newcommand{\branchindex}{\ensuremath{x}\xspace}
\newcommand{\allelecountbottom}[1][\branchindex]{\nodeallelecount{#1}{B}}
\newcommand{\allelecounttop}[1][\branchindex]{\nodeallelecount{#1}{T}}
\newcommand{\redallelecountbottom}[1][\branchindex]{\noderedallelecount{#1}{B}}
\newcommand{\redallelecounttop}[1][\branchindex]{\noderedallelecount{#1}{T}}

\newcommand{\dppmsbayes}{\upshape\texttt{dpp-msbayes}\xspace}
\newcommand{\clang}{\upshape\texttt{C}\xspace}
\newcommand{\cpp}{\upshape\texttt{C++}\xspace}
\newcommand{\ecoevolity}{\upshape\texttt{ecoevolity}\xspace}
\newcommand{\timesizeratemixer}{\upshape\texttt{TimeSizeRateMixer}\xspace}
\newcommand{\timerootsizemixer}{\upshape\texttt{TimeRootSizeMixer}\xspace}
\newcommand{\simcoevolity}{\upshape\texttt{simcoevolity}\xspace}
\newcommand{\sumcoevolity}{\upshape\texttt{sumcoevolity}\xspace}
\newcommand{\pycoevolity}{\upshape\texttt{pycoevolity}\xspace}
\newcommand{\dpprobs}{\upshape\texttt{DPprobs}\xspace}

\newcommand{\rgmurate}{\ensuremath{u}\xspace}
\newcommand{\grmurate}{\ensuremath{v}\xspace}
\newcommand{\murate}[1][]{\ensuremath{\mu_{#1}}\xspace}
\newcommand{\murates}[1][]{\ensuremath{\boldsymbol{\mu}_{#1}}\xspace}
\newcommand{\gfreq}[1][]{\ensuremath{\pi_{#1}}\xspace}
\newcommand{\gfreqs}[1][]{\ensuremath{\boldsymbol{\pi}_{#1}}\xspace}

\newcommand{\comparisonetime}[1][]{\ensuremath{t_{#1}}\xspace}
\newcommand{\comparisonetimes}[1][]{\ensuremath{\mathbf{t}_{#1}}\xspace}
\newcommand{\etime}[1][]{\ensuremath{\tau_{#1}}\xspace}
\newcommand{\etimes}[1][]{\ensuremath{\boldsymbol{\tau}_{#1}}\xspace}
\newcommand{\etimemodel}[1][]{\ensuremath{T_{#1}}\xspace}
\newcommand{\etimesets}{\ensuremath{\mathcal{T}}\xspace}
\newcommand{\genetree}[1][]{\ensuremath{g_{#1}}\xspace}
\newcommand{\sptree}[1][]{\ensuremath{S_{#1}}\xspace}
\newcommand{\sptrees}[1][]{\ensuremath{\mathbf{S}_{#1}}\xspace}

\newcommand{\descendantpopindex}[1]{\ensuremath{D{#1}}}
\newcommand{\rootpopindex}[1][]{\ensuremath{R{#1}}\xspace}
\newcommand{\epopsize}[1][]{\ensuremath{N_{e}^{#1}}\xspace}
\newcommand{\sepopsize}[1][]{\ensuremath{N}\xspace}
\newcommand{\comparisonpopsizes}[1][]{\ensuremath{\mathbb{N}_{e}{#1}}\xspace}
\newcommand{\collectionpopsizes}[1][]{\ensuremath{\mathbf{N_{e}}_{#1}}\xspace}
\newcommand{\rootrelativepopsize}{\ensuremath{R_{\epopsize[\rootpopindex]}}\xspace}

\newcommand{\dirp}{\ensuremath{\textrm{DP}}\xspace}
\newcommand{\concentration}{\ensuremath{\alpha}\xspace}
\newcommand{\basedistribution}{\ensuremath{H}\xspace}
\newcommand{\gshape}{\ensuremath{k}\xspace}
\newcommand{\gscale}{\ensuremath{\theta}\xspace}

\newcommand{\multiplier}{\ensuremath{m}\xspace}
\newcommand{\proposed}{\ensuremath{^{\prime}}\xspace}
\newcommand{\tuningparameter}{\ensuremath{\lambda}\xspace}
\newcommand{\uniformdeviate}{\ensuremath{u}\xspace}
\newcommand{\sizechange}{\ensuremath{\delta}\xspace}

}{}

Following the notation of \citet{Oaks2018ecoevolity}
we will use \leafallelecounts and \leafredallelecounts to denote allele counts
for a locus from either one population if we are modeling population-size
change or both populations of a pair if we are modeling a divergence; i.e., 
$\leafallelecounts, \leafredallelecounts = (\allelecount, \redallelecount)$
or
$\leafallelecounts, \leafredallelecounts = (\allelecount[1],
\redallelecount[1]), (\allelecount[2], \redallelecount[2])$
(\fig{}~\ref{fig:modelCartoon}
and
Table~\ref{table:notation}).
For convenience will use \comparisondata[i] to denote these allele counts
across all the loci from taxon $i$, which can be a single population
or a pair of populations.
Finally, we use \alldata to represent the data across all the taxa for which we
wish to compare times of either divergence or population-size change.
Note, because the population of each compared taxon is modeled separately
(\fig{}~\labelcref{fig:modelCartoon,fig:dag}),
characters do not have to be orthologous across taxa, only within them.


\subsubsection{The evolution of markers}

We assume each character evolved along a gene tree (\genetree)
according to a finite-sites, continuous-time Markov chain (CTMC) model.
We assume the gene tree of each character is independent of the others,
conditional on the population history (i.e., the characters are effectively
unlinked).
As the marker evolves along the gene tree, forward in time, there is an
instantaneous relative rate \rgmurate of mutating from the red state to the
green state, and a corresponding relative rate \grmurate of mutation from
green to red \citep{Bryant2012,Oaks2018ecoevolity}.

As the marker evolves along the gene tree, forward in time, there is an
instantaneous relative rate \rgmurate of mutating from the red state to the
green state, and a corresponding relative rate \grmurate of mutation from green
to red.
The stationary frequency of the green state is then
$\gfreq = \rgmurate / (\rgmurate + \grmurate)$.
We will use \murate to denote the overall rate of mutation.
If $\murate = 1$, time is measured in
units of expected substitutions per site.
Alternatively, if a mutation rate per site per unit time is given, then time is
absolute.

\subsubsection{The evolution of gene trees}

We assume that the gene tree of each locus evolved within a simple
``species'' tree with one ancestral root population, which either
left one or two descendant branches with different effective population sizes
at time \comparisondivtime
(\fig{}~\ref{fig:modelCartoon}).
We will use
\comparisonpopsizes{}
to denote all the effective population sizes of a species tree;
\epopsize[\rootpopindex] and 
\epopsize[\descendantpopindex{1}] when modeling a population-size change, and
(\epopsize[\rootpopindex],
\epopsize[\descendantpopindex{1}],
and \epopsize[\descendantpopindex{2}] when modeling a divergence.
Following \citet{Oaks2018ecoevolity}, we use
\sptree{}
as shorthand for the species tree, which comprises the population sizes and
divergence time of a pair
(\comparisonpopsizes{} and \comparisondivtime{}).


\subsubsection{The likelihood}

As in \citet{Oaks2018ecoevolity},
we use the work of \citet{Bryant2012}
to analytically integrate over all possible gene trees and
character substitution histories to compute the likelihood
of the species tree directly from 
a biallelic character pattern under a coalescent model;
$\pr(\leafallelecounts, \leafredallelecounts \given \sptree, \murate, \gfreq)$.
The only difference that is necessary is for population-size-change models that
have a species tree with only one descendant population.
Equation 19 of \citet{Bryant2012} shows how to obtain the partial likelihoods
at the bottom of an ancestral branch from the partial likelihoods at the top of
its two descendant branches.
When there is only one descendant branch, this is simplified, and the partial
likelihoods at the bottom of the ancestral branch are equal to the partial
likelihoods at the top of its sole descendant branch.
Other than this small change, the probability of a biallelic character pattern
given the species tree, mutation rate, and equilibrium state frequencies
($\pr(\leafallelecounts, \leafredallelecounts \given \sptree, \murate, \gfreq)$)
is calculated the same as in \citet{Bryant2012} and \citet{Oaks2018ecoevolity}.


\begin{linenomath}
For a given taxon, we can calculate the probability of all \nloci{} loci
from which we have data given the species tree and other parameters by
assuming independence among loci (conditional on the species tree) and
taking the product over them,
\begin{equation}
    \pr(\comparisondata \given \sptree, \murate, \gfreq)
    =
    \prod_{i=1}^{\nloci}
    \pr(\leafallelecounts[i], \leafredallelecounts[i] \given \sptree, \murate, \gfreq).
    \label{eq:comparisonlikelihood}
\end{equation}
We assume we biallelic data from \ncomparisons{} taxa, which can be an
arbitrary mix of
(1) two populations or species for which \comparisondivtime represents
the time they diverged, or
(2) one population for which \comparisondivtime represents the time
of a change in population size.
The likelihood across all \ncomparisons{} taxa is simply the product of the
likelihood of each taxon,
\begin{equation}
    \pr(
    \alldata
    \given
    \sptrees,
    \murates,
    \gfreqs)
    =
    \prod_{i=1}^{\ncomparisons}
    \pr(\comparisondata[i] \given \sptree[i], \murate[i], \gfreq[i]),
    \label{eq:collectionlikelihood}
\end{equation}
where
$\alldata = \comparisondata[1], \comparisondata[2], \ldots, \comparisondata[\ncomparisons]$,
$\sptrees = \sptree[1], \sptree[2], \ldots, \sptree[\ncomparisons]$,
$\murates = \murate[1], \murate[2], \ldots, \murate[\ncomparisons]$,
and
$\gfreqs = \gfreq[1], \gfreq[2], \ldots, \gfreq[\ncomparisons]$.
\end{linenomath}
As described in \citet{Oaks2018ecoevolity},
if constant characters are not sampled for a taxon, we condition the likelihood
for that taxon on only having sampled variable characters.


\subsection{Bayesian inference}

\begin{linenomath}
As described by \citet{Oaks2018ecoevolity},
we treat the number of events (population-size changes and/or divergences)
and the assignment of taxa to tose events as
random variables under a Dirichlet process \citep{Ferguson1973,
    Antoniak1974}.
We use \etimesets to represent the partitioning of taxa to events,
which we will also refer to as the ``event model.''
The concentration parameter, \concentration, controls how clustered the
Dirichlet process is, and determines the probability of all possible \etimesets
(i.e., all possible set partitions of taxa to 1--\ncomparisons events).
We use \etimes to represent the times of the unique events in \etimesets.
Using this notation, the posterior distribution of our 
Dirichlet-process model is
\begin{equation}
    \pr(
    \concentration,
    \divtimes,
    \divtimesets,
    \collectionpopsizes,
    \murates,
    \gfreqs
    \given
    \alldata,
    \basedistribution
    )
    =
    \frac{
        \pr(
        \alldata
        \given
        \divtimes,
        \divtimesets,
        \collectionpopsizes,
        \murates,
        \gfreqs
        )
        \pr(\divtimesets \given \concentration)
        \pr(\divtimes \given \divtimesets, \basedistribution)
        \pr(\concentration)
        \pr(\collectionpopsizes)
        \pr(\murates)
        \pr(\gfreqs)
    }{
        \pr(
        \alldata,
        \basedistribution
        )
    }.
    \label{eq:bayesruleexpanded}
\end{equation}
where
\collectionpopsizes
is the collection of the effective population sizes (\comparisonpopsizes{})
across all of the pairs.

\subsubsection{Priors}

\paragraph{Prior on the concentration parameter}
Our implementation allows for a hierarchical approach to accommodeate
uncertainty in the concentration parameter of the Dirichlet process
by specifying a gamma distribution as a hyperprior on
\concentration \citep{Escobar1995,Heath2011}.
Alternatively, \concentration can also be fixed to a particular value,
which is likely sufficient when the number of pairs is small.

\paragraph{Prior on the divergence times}
Given the partitioning of taxa to events, we use a gamma
distribution for the prior on the time of each event,
$\divtime \given \divtimesets \sim \distgamma(\cdot, \cdot)$.

\paragraph{Prior on the effective population sizes}
We use a gamma distribution as the prior on the 
the effective size of each descendant population of each taxon.
Following \citet{Oaks2018ecoevolity},
we use a gamma distribution on the effective size of the ancestral population
\emph{relative} to the size of the descendant population(s), which we
denote as \rootrelativepopsize.
For a taxon with two descendant population (i.e., a divergence model), the
ancestral population size is relative to the mean of the descendant
populations.
For a taxon with only one descendant population (i.e., a size-change model),
the ancestral populations is relative to the size of that descendant.
% The goal of this approach is to allow more informative priors on the root
% population size; we often have stronger prior expectations for the relative
% size of the ancestral population than the absolute size.
% This is important, because the effective size of the ancestral population is a
% difficult nuisance parameter to estimate and can be strongly correlated with
% the divergence time.
% For example, if the divergence time is so old such that all the gene copies
% of a locus coalesce within the descendant populations, the locus
% provides very little information about the size of the ancestral
% population.
% As a result, a larger ancestral population and more recent divergence will have
% a very similar likelihood to a small ancestral population and an older
% divergence.
% Thus, placing more prior density on reasonable values of the ancestral
% population size can help improve the precision of divergence-time estimates.

\paragraph{Prior on mutation rates}
We follow the same approach explained by \citet{Oaks2018ecoevolity} to model
mutation rates across taxa.
The decision about how to model mutation rates is extremely important for any
comparative phylogeographic approach that models taxa as disconnected
species trees
\citep[\fig{}~\ref{fig:divCartoon}; e.g.,][]{Hickerson2006,Hickerson2007,Huang2011,Chan2014,Oaks2014dpp,Xue2015,Burbrink2016,Xue2017,Gehara2017,Oaks2018ecoevolity}.
Here, we quote \citet{Oaks2018ecoevolity} directly:
\begin{quotation}
Uses of ABC estimation of shared size changes \citep{Burbrink2016}
In the model presented above, for each population pair, the divergence time
(\divtime) and mutation rate (\murate) are inextricably linked.
For a single pair of populations, if little is known about the mutation rate,
this problem is easily solved by setting it to one ($\murate[1] = 1$) such
that time is in units of expected substitutions per site and the effective
population sizes are scaled by \murate.
However, what about the second pair of populations for which we wish to compare
the divergence time to the first?
Because the species trees in our model are disconnected
\ldots,
we cannot learn about
the relative rates of mutation across the population pairs from the data.
As a result, we need strong prior information about the relative rates of
mutation across population pairs for this model to work.

If the second pair of populations is closely related to the first, and shares a
similar life history, we could assume they share the same mutation rate and
set the mutation rate of the second pair to one as well ($\murate[1] = \murate[2] = 1$).
Alternatively, we could relax that assumption and put a prior on \murate[2].
However, this should be a strongly informative prior.
Placing a weakly informative prior on \murate[2] would mean that we can no
longer estimate its divergence time relative to the first pair,
which is our primary goal.
So, while it is possible to incorporate uncertainty in relative mutation rates,
it is important to keep in mind that the data cannot inform these parameters,
and thus the prior uncertainty in rates will be directly reflected in the
posterior of divergence times.
\end{quotation}

\paragraph{Prior on the equilibrium-state frequency}
Recoding four-state nucleotides to two states requires some arbitrary
decisions, and whenever $\gfreq \neq 0.5$, these decisions can affect
the likelihood of the model \citep{Oaks2018ecoevolity}.
Because DNA is the dominant character type for genomic data, we assume that
$\gfreq = 0.5$ in this paper.
The makes the CTMC model of character-state substitution a two-state analog of
the ``JC69'' model \citep{JC1969}.
However, if the genetic markers collected for one or more taxa are  naturally
biallelic, the frequencies of the two states can be meaningfully estimated, and
our implementation allows for a beta prior on \gfreq in such cases.
This makes the CTMC model of character-state substitution a two-state general
time-reversible model \citep{Tavare1986}.

\subsubsection{Approximating the posterior with MCMC}

We use Markov chain Monte Carlo (MCMC) algorithms to sample from the joint
posterior in Equation~\ref{eq:bayesruleexpanded}.
To sample across event models (\etimesets) during the MCMC chain, we use the
Gibbs sampling algorithm (Algorithm 8) of \citet{Neal2000}.
We also use univariate and multivariate Metropolis-Hastings algorithms
\citep{Metropolis1953,Hastings1970} to update the model,
the latter of which are detailed in \citet{Oaks2018ecoevolity}.

\subsection{Software implementation}
The model and Bayesian estimattion outlined above has been incorporated into
the open-source software package, \ecoevolity, written in the \cpp language.
The source code is freely available from
\url{https://github.com/phyletica/ecoevolity} and
includes an extensive test suite.
Documentation for how to install and use the software is available at
\url{http://phyletica.org/ecoevolity/}.
We have incorporated help in preprocessing data and postprocessing posterior
samples collectd by \ecoevolity into the Python package \pycoevolity, which is
available at
\url{https://github.com/phyletica/pycoevolity}.

We used Version 0.3.1 
(commit 9284417)
of the \ecoevolity software package for all of our analyses.
A detailed history of this project, including all of the data and scripts
needed to produce our results, is available at
\url{https://github.com/phyletica/ecoevolity-demog-experiments}.

\subsection{Analyses of simulated data}

\subsubsection{Assessing ability to estimate timing and sharing of demographic changes}

Settings used for all simulations include:
\begin{itemize}
    \item 3 pairs
    \item 500k characters
    \item 20 gene copies per population (10 diploid individuals)
    \item Concentration of DP 1.414216 (mean nevents of 2)
\end{itemize}

Settings used for all analyses include:
\begin{itemize}
    \item MCMC chain length of 75,000 sampled every 50 generations.
\end{itemize}

Initial pass. motivation: distribute event times to span values that are easy and hard.

Event times (in units of expected subsitutions per site) were exponentially
distributed with a mean of 0.01.
Considering the expected population size was 0.002, this puts our expectaction
for the event times in units of $4N_e$ generations at 1.25.
Thus, we expected to get a mix of demographic-change time that occurred more recently
than the gene trees coalesced and thus would be easy to estimate,
and those that occurred when few or no gene lineages were left to coalesce, and thus would
be difficult to estimate.

Results were quite poor across all conditions, but slightly better for the most dramatic pop expansions.
In an effort to find conditions where we could estimate demographic change times reasonble, we next simulated conditions where strong population expansions (relative root size gamma(10, mean = 0.25) and gamma(10, mean = 0.5)) occurred very recently (exponential with mean 0.001).

The improved performance was quite modest, and we ran into numerical issues when the
population expansion so recent that it was difficult to identify.
In these cases, the data were well explained by no expansion, which could be achieved in two ways: an expansion time of zero and an ancestral population size that matched the true ancestral population size, or an old expansion and a ldescendant population size that matched the true ancestral population size. The latter explained the data equally well when the divergence time was older than gene tree coalescences. This lead to MCMC chains converging to these different regions of parameter space.

This lead us to try tighter distribution on times with an offest to avoid
near-zero values, as well as offsest on the size of descendant populations and
ancestral relative sizes to avoid near-zero values (which cause rapid
coalescence, and thus no signal). The prior conditions we simulted under are:


\subsubsection{But what about under realistic prior information (diffuseprior)}

It was a lot of work to find conditions where we could reasonably estimate
the timing and number of demographic change events.
These condtions include quite informative distributions on parameters,
especially the relative size of the ancestral population.
We used the same distribution as the prior, and thus the informative priors
could be why the behavior is better (as opposed to the data containing a strong
signal of what happened).
To determine this, we simulated data under the best conditions,
but then analyzed these data under diffuse priors to see how
well we can expect to estimate.
This is important because in real world applications we usually know very
little about the timing and magnitude (and direction) of size changes.

Sim disributions:

Event time ~ gamma(shape=4.0, scale=0.000475, offset=0.0001); mean 0.002

relative root size ~ gamma(shape=5.0, scale = 0.04, offset = 0.05); mean 0.25

relative root size ~ gamma(shape=5.0, scale = 0.04, offset = 3.8); mean 4.0

descendant size ~ gamma(4.0, scale=0.0005, offset = 0.0001); mean 0.0021

Priors:

Event time ~ exponential(mean = 0.005)

relative root size ~ Exponential(mean = 2.0)

descendant size ~ gamma(2.0, scale 0.001); mean 0.002



\subsection{Simulating linked sites}
Should we analyze all the sites or just unlinked SNPs?


\subsection{Simulating a mix of divergence and pop-size-change comparisons}
What if we have a mix of pairs?

\subsection{Empirical application}
Stickleback data.


\section{Results}

\subsection{Analyses of simulated data}


\subsection{Empirical application}


\section{Discussion}
