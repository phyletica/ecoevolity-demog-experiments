\section{Introduction}

A primary goal of ecology and evolutionary biology is to understand the
processes influencing the distribution, abundance, and diversification of
species.
Many biotic and abiotic factors that shape the distribution of biodiversity
across a landscape are expected to affect multiple species.
Abiotic mechanisms include changes to the environment that can cause
co-distributed species to contract or expand their ranges and/or become
fragmented \citep{Wegener1966,Avise1987,Knowles2002}.
Biotic factors include inter-specific ecological interactions such as the range
expansion of a species causing the expansion of its symbionts and the range
contraction and/or fragmentation of its competitors
\citep{Lotka1920,Volterra1926}.
Such processes predict that evolutionary events, such as population divergences
or demographic changes, will be temporally clustered across multiple species.
As a result, statistical methods that infer such patterns from genetic data
allow ecologists and evolutionary biologists to test hypotheses about such
processes operating at or above the scale of communities of species.

Recent research has developed methods to infer
patterns of temporally clustered (or ``shared'') evolutionary events,
including shared divergence times among pairs of populations
\citep{Hickerson2006,Hickerson2007,Huang2011,Oaks2014dpp,Oaks2018ecoevolity}
and shared demographic changes in effective population size across populations
\citep{Chan2014,Xue2015,Burbrink2016,Prates2016,Xue2017,Gehara2017}
from comparative genetic data.
% We use ``demography'' to refer to the effective size of populations.
To date, no method has allowed the joint inference of both shared divergences
and population-size changes.
Given the overlap among processes that can potentially cause divergence and
demographic changes of populations across multiple species, such a method would
be useful for testing hypotheses about community-scale processes that shape
biodiversity across landscapes.
Here, we introduce a general, full-likelihood Bayesian method that can estimate
shared times among an arbitrary mix of population divergences and population
size changes (\fig{}~\ref{fig:modelCartoon}).
% We implemented this method in the software package \ecoevolity.

Whereas the theory and performance of methods that estimate shared divergence
times has been relatively well-investigated
\citep[e.g.,][]{Oaks2012,Hickerson2013,Oaks2014reply,Oaks2014dpp,Overcast2017,Oaks2018ecoevolity},
exploration into the estimation of shared changes in population size has been
much more limited.
There are theoretical reasons to suspect that estimating shared changes in
effective population size is more difficult than divergence times.
The parameter of interest (timing of a demographic change) is informed by
differing rates at which sampled copies of a gene ``find'' their common
ancestors (coalesce) going backward in time before and after the change in
population size, and this can become unidentifiable in three ways.
First, as the magnitude of a change in population becomes smaller,
it becomes more difficult to estimate.
Second, as the age of the demographic change gets older, fewer of the genetic
coalescent events occur prior to the change, resulting in less information
about the timing and magnitude of the change.
Third, information also decreases as the age of the demographic change
approaches zero, because fewer coalescent events occur after the change.
To explore these potential problems, we take advantage of our full-likelihood
method to assess how well we can infer shared demographic changes among
populations when using all the information in genomic data.
We apply our method to restriction-site-associated DNA sequence (RADseq) data
from five populations of three-spine stickleback
\citep[\spp{Gasterosteus aculeatus};][]{Hohenlohe2010}
that were previously estimated to have co-expanded with an approximate Bayesian
computation (ABC) approach \citep{Xue2015}.

\ifembed{
\embedWidthFigure{1.0}{../images/model-cartoons/model-choice-island-cartoon-mixed-simple-comp-labels-compressed.pdf}{
    An illustration of the general comparative model using three insular lizard
    taxa.
    The top two comparisons are pairs of populations for which we are
    interested in comparing their time of divergence (``divergence
    comparisons'').
    The bottom comparison is a single population for which we are interested in
    comparing the time of population-size change (``demographic comparison'').
    In this scenario, three species of lizards co-occured on an island that was
    fragmented by a rise in sea levels at time \etime[1].
    Due to the fragmentation, the second lineage (from the top) diverged into
    two descendant populations while the population size of the third lineage
    was reduced.
    The first lineage diverged later at \etime[2] due to another mechanism,
    such as over-water dispersal.
    The five possible event models are shown to the right, with the
    correct model indicated.
    % The event time parameters (\etime[1] and \etime[2])
    % and the pair-specific event times (\comparisonetime[1],
    % \comparisonetime[2], and \comparisonetime[3])
    % are shown.
    % The notation used in the text for the biallelic character data and
    % effective sizes of the ancestral and descendant populations is shown for
    % the third comparison.
    The lizard silhouette for the middle pair is from pixabay.com, and the
    other two are from phylopic.org; all were licensed under the Creative
    Commons (CC0) 1.0 Universal Public Domain Dedication.
    Modified from \citet{Oaks2018ecoevolity}.
}{fig:modelCartoon}

}{}

\section{The model}

We extended the model implemented in the software package \ecoevolity to
accommodate two types of temporal comparisons:
\begin{enumerate}
    \item A population that experienced a change from effective population size
        \epopsize[\rootpopindex]
        to effective size
        \epopsize[\descendantpopindex{}]
        at time \comparisonetime in the past.
        We will refer to this as a \emph{demographic comparison}
        (\fig{}~\ref{fig:modelCartoon}),
        and refer to the population before and after the change in population
        size as ``ancestral'' and ``descendant'', respectively.
    \item A population that diverged at time \comparisonetime in the past into
        two descendant populations, each with unique effective population
        sizes.
        We will refer to this as a \emph{divergence comparison}
        (\fig{}~\ref{fig:modelCartoon}).
\end{enumerate}
The type of each comparison is specified by the investigator; either the input
data specifies one (demographic comparison) or two (divergence comparison)
populations.
This allows inference of shared times of divergence and/or demographic change
across an arbitrary mix of demographic and divergence comparisons in a
full-likelihood, Bayesian framework.
Table~\ref{table:notation} provides a key to the notation we use throughout
this paper.

\ifembed{
\begin{table}[htbp]
    \sffamily
    \small
    \rowcolors{2}{}{mygray}
    \addtolength{\tabcolsep}{-0.1cm}
\caption{
    A key to some of the notation used in the text.
}
    \centering
    \begin{tabular}{ l >{\raggedright\hangindent=0.5cm}m{14cm} }
        \toprule
        \textbf{Symbol} & \textbf{Description} \tn
        \midrule
        \ncomparisons{} & The number of comparisons (or taxa); can be an
        arbitrary mix of populations (comparing timing of demographic change)
        and/or pairs of populations (comparing timing of divergence).
        \tn
        \nevents{} & The number of events (unique times) across the comparisons.
        \tn
        \comparisonetime[i] & The time in the past when comparison $i$ either
        diverged or experienced a change in effective population size.
        \tn
        \etime & An event time at which one or more comparisons
        experienced a divergence or change in effective population size.
        % \tn
        % \etimemodel & The event-time model, which comprises the event
        % times and the mapping of comparisons to those times.
        \tn
        \etimesets & The event-time model, which comprises the assignment of
        comparisons to events.
        \tn
        \etimes & All of the times of the events in the model
            ($\etimes = \etime[1], \ldots, \etime[\nevents]$).
        \tn
        % \basedistribution & The base distribution of the Dirichlet process.
        % \tn
        \concentration & The concentration parameter of the Dirichlet process.
        \tn
        \allelecount, \redallelecount & The number of copies of a locus sampled
            from a population, and the number of those copies that are the ``red''
            allele.
            \tn
        \leafallelecounts, \leafredallelecounts & The allele counts from 
        a comparison (one or two populations).
            \tn
        \comparisondata[i] & The allele counts across all characters from
            comparison $i$. I.e., all of the characters being analyzed for
            comparison $i$.
            \tn
        \nloci & The number of characters collected from a taxon (comparison).
        \tn
        \alldata & All of the data being analyzed, i.e., the
            character matrices from all comparisons.
        \tn
        \genetree & A gene tree with branch lengths.
        \tn
        \murate & The rate of mutation.
        \tn
        \rgmurate & Relative rate of mutating from the ``red'' to ``green'' state.
        \tn
        \grmurate & Relative rate of mutating from the ``green'' to ``red'' state.
        \tn
        \gfreq & The stationary frequency of the ``green'' state.
        \tn
        \epopsize[\descendantpopindex{}] &
            The effective size of a descendant population.
        \tn
        \epopsize[\rootpopindex] & The effective size of the root (ancestral) population.
        \tn
        \rootrelativepopsize & The relative effective population size of the root (ancestral)
        population; relative to the mean of the effective sizes of the descendant populations.
        \tn
        \comparisonpopsizes & Shorthand notation for all effective population
        sizes for a comparison (ancestral and one or two descendant
        populations).
        \tn
        \sptree & The species tree for a comparison. This comprises
        the effective population sizes and the time of demographic change or
        divergence.
            \tn
        \bottomrule
    \end{tabular}
    \label{table:notation}
\end{table}

}{}

\subsection{The data}
As described by \citet{Oaks2018ecoevolity},
we assume we have collected orthologous, biallelic genetic characters from taxa
we wish to compare.
By biallelic, we mean that each character has at most two states,
which we refer to as ``red'' and ``green'' following \citet{Bryant2012}.
For each taxon, we either have these data from one or more individuals from a
single population, in which case we infer the timing and extent of a population
size change, or one or more individuals from two populations or species,
in which case we infer the time when they diverged (\fig{}~\ref{fig:modelCartoon}).

For each population and for each character we genotype \allelecount copies
of the locus, \redallelecount of which are copies of the red allele and the
remaining $\allelecount - \redallelecount$ are copies of the green allele.
Thus, for each population, and for each locus, we have a count of the
total sampled gene copies and how many of those are the red allele.

Following the notation of \citet{Oaks2018ecoevolity}
we will use \leafallelecounts and \leafredallelecounts to denote allele counts
for a locus from either one population if we are modeling a population-size
change or both populations of a pair if we are modeling a divergence; i.e., 
$\leafallelecounts, \leafredallelecounts = (\allelecount, \redallelecount)$
or
$\leafallelecounts, \leafredallelecounts = (\allelecount[1],
\redallelecount[1]), (\allelecount[2], \redallelecount[2])$.
% (\fig{}~\ref{fig:modelCartoon}
% and
% Table~\ref{table:notation}).
For convenience will use \comparisondata[i] to denote these allele counts
across all the loci from taxon $i$, which can be a single population
or a pair of populations.
Finally, we use \alldata to represent the data across all the taxa for which we
wish to compare times of either divergence or population-size change.
Note, because the population history of each taxon is modeled separately
(\fig{}~\ref{fig:modelCartoon}),
characters do not have to be orthologous across taxa, only within them.


\subsection{The evolution of markers}

We assume each character evolved along a gene tree (\genetree)
according to a finite-sites, continuous-time Markov chain (CTMC) model.
We assume the gene tree of each character is independent of the others,
conditional on the population history (i.e., the characters are effectively
unlinked).
As the marker evolves along the gene tree, forward in time, there is a relative
rate \rgmurate of mutating from the red state to the green state,
and a corresponding relative rate \grmurate of mutation from green to red
\citep{Bryant2012,Oaks2018ecoevolity}.
The stationary frequency of the green state is then
$\gfreq = \rgmurate / (\rgmurate + \grmurate)$.
We will use \murate to denote the overall rate of mutation.
Evolutionary change is the product of \murate and time.
Thus, if $\murate = 1$, time is measured in
units of expected substitutions per site.
Alternatively, if a mutation rate per site per unit time is given, then time is
in those units (e.g., generations or years).

\subsection{The evolution of gene trees}

We assume that the gene tree of each locus evolved within a simple
``species'' tree with one ancestral root population, which either
left one or two descendant branches with different effective population sizes
at time \comparisonetime
(\fig{}~\ref{fig:modelCartoon}).
We will use
\comparisonpopsizes{}
to denote all the effective population sizes of a species tree;
\epopsize[\rootpopindex] and 
\epopsize[\descendantpopindex{}] when modeling a population-size change, and
\epopsize[\rootpopindex],
\epopsize[\descendantpopindex{1}],
and \epopsize[\descendantpopindex{2}] when modeling a divergence.
Following \citet{Oaks2018ecoevolity}, we use
\sptree{}
as shorthand for the species tree, which comprises the population sizes and
event time of a comparison
(\comparisonpopsizes{} and \comparisonetime{}).


\subsection{The likelihood}

As in \citet{Oaks2018ecoevolity},
we use the work of \citet{Bryant2012}
to analytically integrate over all possible gene trees and
character substitution histories to compute the likelihood
of the species tree directly from 
a biallelic character pattern under a coalescent model;
$\pr(\leafallelecounts, \leafredallelecounts \given \sptree, \murate, \gfreq)$.
We only need to make a small modification to accommodate population-size-change
models that have a species tree with only one descendant population.
Equation 19 of \citet{Bryant2012} shows how to obtain the partial likelihoods
at the bottom of an ancestral branch from the partial likelihoods at the top of
its two descendant branches.
When there is only one descendant branch, this is simplified, and the partial
likelihoods at the bottom of the ancestral branch are equal to the partial
likelihoods at the top of its sole descendant branch.
Other than this small change, the probability of a biallelic character pattern
given the species tree, mutation rate, and equilibrium state frequencies
($\pr(\leafallelecounts, \leafredallelecounts \given \sptree, \murate, \gfreq)$)
is calculated the same as in \citet{Bryant2012} and \citet{Oaks2018ecoevolity}.


\begin{linenomath}
For a given taxon, we can calculate the probability of all \nloci{} loci
from which we have data given the species tree and other parameters by
assuming independence among loci (conditional on the species tree) and
taking the product over them,
\begin{equation}
    \pr(\comparisondata \given \sptree, \murate, \gfreq)
    =
    \prod_{i=1}^{\nloci}
    \pr(\leafallelecounts[i], \leafredallelecounts[i] \given \sptree, \murate, \gfreq).
    \label{eq:comparisonlikelihood}
\end{equation}
We assume we have sampled biallelic data from \ncomparisons{} taxa, which can
be an arbitrary mix of
(1) two populations or species for which \comparisonetime represents
the time they diverged, or
(2) one population for which \comparisonetime represents the time
of a change in population size.
The likelihood across all \ncomparisons{} taxa is simply the product of the
likelihood of each taxon,
\begin{equation}
    \pr(
    \alldata
    \given
    \sptrees,
    \murates,
    \gfreqs)
    =
    \prod_{i=1}^{\ncomparisons}
    \pr(\comparisondata[i] \given \sptree[i], \murate[i], \gfreq[i]),
    \label{eq:collectionlikelihood}
\end{equation}
where
$\alldata = \comparisondata[1], \comparisondata[2], \ldots, \comparisondata[\ncomparisons]$,
$\sptrees = \sptree[1], \sptree[2], \ldots, \sptree[\ncomparisons]$,
$\murates = \murate[1], \murate[2], \ldots, \murate[\ncomparisons]$,
and
$\gfreqs = \gfreq[1], \gfreq[2], \ldots, \gfreq[\ncomparisons]$.
As described in \citet{Oaks2018ecoevolity},
if constant characters are not sampled for a taxon, we condition the likelihood
for that taxon on only having sampled variable characters.
\end{linenomath}


\subsection{Bayesian inference}

\begin{linenomath}
As described by \citet{Oaks2018ecoevolity},
we treat the number of events (population-size changes and/or divergences)
and the assignment of taxa to those events as
random variables under a Dirichlet process \citep{Ferguson1973,
    Antoniak1974}.
We use \etimesets to represent the partitioning of taxa to events,
which we will also refer to as the ``event model.''
The concentration parameter, \concentration, controls how clustered the
Dirichlet process is, and determines the probability of all possible \etimesets
(i.e., all possible set partitions of taxa to $1, 2, \ldots, \ncomparisons$ events).
We use \etimes to represent the unique times of events in \etimesets.
Using this notation, the posterior distribution of our 
Dirichlet-process model is
\begin{equation}
\begin{split}
    & \pr(
    \concentration,
    \etimes,
    \etimesets,
    \collectionpopsizes,
    \murates,
    \gfreqs
    \given
    \alldata
    % \basedistribution
    )
    = \\
    & \frac{
        \pr(
        \alldata
        \given
        \etimes,
        \etimesets,
        \collectionpopsizes,
        \murates,
        \gfreqs
        )
        \pr(\etimes \given \etimesets)%, \basedistribution)
        \pr(\etimesets \given \concentration)
        \pr(\concentration)
        \pr(\collectionpopsizes)
        \pr(\murates)
        \pr(\gfreqs)
    }{
        \pr(
        \alldata%,
        % \basedistribution
        )
    },
    \label{eq:bayesruleexpanded}
\end{split}
\end{equation}
where
\collectionpopsizes
is the collection of the effective population sizes (\comparisonpopsizes{})
across all of the pairs.
\end{linenomath}

\subsubsection{Priors}

\paragraph{Prior on the concentration parameter}
Our implementation allows for a hierarchical approach to accommodate
uncertainty in the concentration parameter of the Dirichlet process
by specifying a gamma distribution as a hyperprior on
\concentration \citep{Escobar1995,Heath2011}.
Alternatively, \concentration can also be fixed to a particular value,
which is likely sufficient when the number of pairs is small.

\paragraph{Prior on the divergence times}
Given the partitioning of taxa to events, we use a gamma
distribution for the prior on the time of each event,
$\etime \given \etimesets \sim \distgamma(\cdot, \cdot)$.

\paragraph{Prior on the effective population sizes}
We use a gamma distribution as the prior on
the effective size of each descendant population of each taxon.
Following \citet{Oaks2018ecoevolity},
we use a gamma distribution on the effective size of the ancestral population
\emph{relative} to the size of the descendant population(s), which we
denote as \rootrelativepopsize.
For a taxon with two descendant populations (i.e., a divergence comparison), the
ancestral population size is relative to the mean of the descendant
populations.
For a taxon with only one descendant population (i.e., a demographic
comparison), the ancestral populations is relative to the size of that
descendant.
% The goal of this approach is to allow more informative priors on the root
% population size; we often have stronger prior expectations for the relative
% size of the ancestral population than the absolute size.
% This is important, because the effective size of the ancestral population is a
% difficult nuisance parameter to estimate and can be strongly correlated with
% the divergence time.
% For example, if the divergence time is so old such that all the gene copies
% of a locus coalesce within the descendant populations, the locus
% provides very little information about the size of the ancestral
% population.
% As a result, a larger ancestral population and more recent divergence will have
% a very similar likelihood to a small ancestral population and an older
% divergence.
% Thus, placing more prior density on reasonable values of the ancestral
% population size can help improve the precision of divergence-time estimates.

\paragraph{Prior on mutation rates}
We follow the same approach explained by \citet{Oaks2018ecoevolity} to model
mutation rates across taxa.
The decision about how to model mutation rates is extremely important for any
comparative phylogeographic approach that models taxa as disconnected
species trees
\citep[\fig{}~\ref{fig:modelCartoon}; e.g.,][]{Hickerson2006,Hickerson2007,Huang2011,Chan2014,Oaks2014dpp,Xue2015,Burbrink2016,Xue2017,Gehara2017,Oaks2018ecoevolity}.
Time (\etime) and mutation rate (\murate) are inextricably linked, and because
the comparisons are modeled as separate species trees, the data cannot
inform the model about relative or absolute differences in \murate among the
comparisons.
We provide flexibility to the investigator to fix or place prior probability
distributions on the relative or absolute rate of mutation for each comparison.
However, if one chooses to accommodate uncertainty in the mutation rate of one
or more comparisons, the priors should be strongly informative.
Because of the inextricable link between rate and time,
placing a weakly informative prior on a comparison's mutation rate prevents
estimation of the time of its demographic change or divergence,
which is the primary goal.

\paragraph{Prior on the equilibrium state frequency}
Recoding four-state nucleotides to two states requires some arbitrary
decisions, and whenever $\gfreq \neq 0.5$, these decisions can affect
the likelihood of the model \citep{Oaks2018ecoevolity}.
Because DNA is the dominant character type for genomic data, we assume that
$\gfreq = 0.5$ in this paper.
This makes the CTMC model of character-state substitution a two-state analog of
the ``JC69'' model \citep{JC1969}.
However, if the genetic markers collected for one or more taxa are naturally
biallelic, the frequencies of the two states can be meaningfully estimated, and
our implementation allows for a beta prior on \gfreq in such cases.
This makes the CTMC model of character-state substitution a two-state general
time-reversible model \citep{Tavare1986}.

\subsubsection{Approximating the posterior with MCMC}

We use Markov chain Monte Carlo (MCMC) algorithms to sample from the joint
posterior in Equation~\ref{eq:bayesruleexpanded}.
To sample across event models (\etimesets) during the MCMC chain, we use the
Gibbs sampling algorithm (Algorithm 8) of \citet{Neal2000}.
We also use univariate and multivariate Metropolis-Hastings algorithms
\citep{Metropolis1953,Hastings1970} to update the model,
the latter of which are detailed in \citet{Oaks2018ecoevolity}.

\subsection{Software implementation}
The \cpp source code for \ecoevolity is freely available from
\url{https://github.com/phyletica/ecoevolity} and includes an extensive test
suite.
From the \cpp source code, two primary command-line tools are compiled:
(1) \ecoevolity, for performing Bayesian inference under the model described
above,
and
(2) \simcoevolity for simulating data under the model described above.
Documentation for how to install and use the software is available at
\url{http://phyletica.org/ecoevolity/}.
We have incorporated help in pre-processing data and post-processing posterior
samples collected by \ecoevolity in the Python package \pycoevolity, which is
available at
\url{https://github.com/phyletica/pycoevolity}.
We used Version 0.3.1
(Commit 9284417)
of the \ecoevolity software package for all of our analyses.
A detailed history of this project, including all of the data and scripts
needed to produce our results, is available at
\url{https://github.com/phyletica/ecoevolity-demog-experiments}
\citep{Oaks2019CodemogZenodo}.


\section{Materials \& Methods}

% \section{The model}

We extended the model implemented in the software package \ecoevolity to
accommodate two types of temporal comparisons:
\begin{enumerate}
    \item A population that experienced a change from effective population size
        \epopsize[\rootpopindex]
        to effective size
        \epopsize[\descendantpopindex{}]
        at time \comparisonetime in the past.
        We will refer to this as a \emph{demographic comparison}
        (\fig{}~\ref{fig:modelCartoon}),
        and refer to the population before and after the change in population
        size as ``ancestral'' and ``descendant'', respectively.
    \item A population that diverged at time \comparisonetime in the past into
        two descendant populations, each with unique effective population
        sizes.
        We will refer to this as a \emph{divergence comparison}
        (\fig{}~\ref{fig:modelCartoon}).
\end{enumerate}
The type of each comparison is specified by the investigator; either the input
data specifies one (demographic comparison) or two (divergence comparison)
populations.
This allows inference of shared times of divergence and/or demographic change
across an arbitrary mix of demographic and divergence comparisons in a
full-likelihood, Bayesian framework.
Table~\ref{table:notation} provides a key to the notation we use throughout
this paper.

\ifembed{
\begin{table}[htbp]
    \sffamily
    \small
    \rowcolors{2}{}{mygray}
    \addtolength{\tabcolsep}{-0.1cm}
\caption{
    A key to some of the notation used in the text.
}
    \centering
    \begin{tabular}{ l >{\raggedright\hangindent=0.5cm}m{14cm} }
        \toprule
        \textbf{Symbol} & \textbf{Description} \tn
        \midrule
        \ncomparisons{} & The number of comparisons (or taxa); can be an
        arbitrary mix of populations (comparing timing of demographic change)
        and/or pairs of populations (comparing timing of divergence).
        \tn
        \nevents{} & The number of events (unique times) across the comparisons.
        \tn
        \comparisonetime[i] & The time in the past when comparison $i$ either
        diverged or experienced a change in effective population size.
        \tn
        \etime & An event time at which one or more comparisons
        experienced a divergence or change in effective population size.
        % \tn
        % \etimemodel & The event-time model, which comprises the event
        % times and the mapping of comparisons to those times.
        \tn
        \etimesets & The event-time model, which comprises the assignment of
        comparisons to events.
        \tn
        \etimes & All of the times of the events in the model
            ($\etimes = \etime[1], \ldots, \etime[\nevents]$).
        \tn
        % \basedistribution & The base distribution of the Dirichlet process.
        % \tn
        \concentration & The concentration parameter of the Dirichlet process.
        \tn
        \allelecount, \redallelecount & The number of copies of a locus sampled
            from a population, and the number of those copies that are the ``red''
            allele.
            \tn
        \leafallelecounts, \leafredallelecounts & The allele counts from 
        a comparison (one or two populations).
            \tn
        \comparisondata[i] & The allele counts across all characters from
            comparison $i$. I.e., all of the characters being analyzed for
            comparison $i$.
            \tn
        \nloci & The number of characters collected from a taxon (comparison).
        \tn
        \alldata & All of the data being analyzed, i.e., the
            character matrices from all comparisons.
        \tn
        \genetree & A gene tree with branch lengths.
        \tn
        \murate & The rate of mutation.
        \tn
        \rgmurate & Relative rate of mutating from the ``red'' to ``green'' state.
        \tn
        \grmurate & Relative rate of mutating from the ``green'' to ``red'' state.
        \tn
        \gfreq & The stationary frequency of the ``green'' state.
        \tn
        \epopsize[\descendantpopindex{}] &
            The effective size of a descendant population.
        \tn
        \epopsize[\rootpopindex] & The effective size of the root (ancestral) population.
        \tn
        \rootrelativepopsize & The relative effective population size of the root (ancestral)
        population; relative to the mean of the effective sizes of the descendant populations.
        \tn
        \comparisonpopsizes & Shorthand notation for all effective population
        sizes for a comparison (ancestral and one or two descendant
        populations).
        \tn
        \sptree & The species tree for a comparison. This comprises
        the effective population sizes and the time of demographic change or
        divergence.
            \tn
        \bottomrule
    \end{tabular}
    \label{table:notation}
\end{table}

}{}

\subsection{The data}
As described by \citet{Oaks2018ecoevolity},
we assume we have collected orthologous, biallelic genetic characters from taxa
we wish to compare.
By biallelic, we mean that each character has at most two states,
which we refer to as ``red'' and ``green'' following \citet{Bryant2012}.
For each taxon, we either have these data from one or more individuals from a
single population, in which case we infer the timing and extent of a population
size change, or one or more individuals from two populations or species,
in which case we infer the time when they diverged (\fig{}~\ref{fig:modelCartoon}).

For each population and for each character we genotype \allelecount copies
of the locus, \redallelecount of which are copies of the red allele and the
remaining $\allelecount - \redallelecount$ are copies of the green allele.
Thus, for each population, and for each locus, we have a count of the
total sampled gene copies and how many of those are the red allele.

Following the notation of \citet{Oaks2018ecoevolity}
we will use \leafallelecounts and \leafredallelecounts to denote allele counts
for a locus from either one population if we are modeling a population-size
change or both populations of a pair if we are modeling a divergence; i.e., 
$\leafallelecounts, \leafredallelecounts = (\allelecount, \redallelecount)$
or
$\leafallelecounts, \leafredallelecounts = (\allelecount[1],
\redallelecount[1]), (\allelecount[2], \redallelecount[2])$.
% (\fig{}~\ref{fig:modelCartoon}
% and
% Table~\ref{table:notation}).
For convenience will use \comparisondata[i] to denote these allele counts
across all the loci from taxon $i$, which can be a single population
or a pair of populations.
Finally, we use \alldata to represent the data across all the taxa for which we
wish to compare times of either divergence or population-size change.
Note, because the population history of each taxon is modeled separately
(\fig{}~\ref{fig:modelCartoon}),
characters do not have to be orthologous across taxa, only within them.


\subsection{The evolution of markers}

We assume each character evolved along a gene tree (\genetree)
according to a finite-sites, continuous-time Markov chain (CTMC) model.
We assume the gene tree of each character is independent of the others,
conditional on the population history (i.e., the characters are effectively
unlinked).
As the marker evolves along the gene tree, forward in time, there is a relative
rate \rgmurate of mutating from the red state to the green state,
and a corresponding relative rate \grmurate of mutation from green to red
\citep{Bryant2012,Oaks2018ecoevolity}.
The stationary frequency of the green state is then
$\gfreq = \rgmurate / (\rgmurate + \grmurate)$.
We will use \murate to denote the overall rate of mutation.
Evolutionary change is the product of \murate and time.
Thus, if $\murate = 1$, time is measured in
units of expected substitutions per site.
Alternatively, if a mutation rate per site per unit time is given, then time is
in those units (e.g., generations or years).

\subsection{The evolution of gene trees}

We assume that the gene tree of each locus evolved within a simple
``species'' tree with one ancestral root population, which either
left one or two descendant branches with different effective population sizes
at time \comparisonetime
(\fig{}~\ref{fig:modelCartoon}).
We will use
\comparisonpopsizes{}
to denote all the effective population sizes of a species tree;
\epopsize[\rootpopindex] and 
\epopsize[\descendantpopindex{}] when modeling a population-size change, and
\epopsize[\rootpopindex],
\epopsize[\descendantpopindex{1}],
and \epopsize[\descendantpopindex{2}] when modeling a divergence.
Following \citet{Oaks2018ecoevolity}, we use
\sptree{}
as shorthand for the species tree, which comprises the population sizes and
event time of a comparison
(\comparisonpopsizes{} and \comparisonetime{}).


\subsection{The likelihood}

As in \citet{Oaks2018ecoevolity},
we use the work of \citet{Bryant2012}
to analytically integrate over all possible gene trees and
character substitution histories to compute the likelihood
of the species tree directly from 
a biallelic character pattern under a coalescent model;
$\pr(\leafallelecounts, \leafredallelecounts \given \sptree, \murate, \gfreq)$.
We only need to make a small modification to accommodate population-size-change
models that have a species tree with only one descendant population.
Equation 19 of \citet{Bryant2012} shows how to obtain the partial likelihoods
at the bottom of an ancestral branch from the partial likelihoods at the top of
its two descendant branches.
When there is only one descendant branch, this is simplified, and the partial
likelihoods at the bottom of the ancestral branch are equal to the partial
likelihoods at the top of its sole descendant branch.
Other than this small change, the probability of a biallelic character pattern
given the species tree, mutation rate, and equilibrium state frequencies
($\pr(\leafallelecounts, \leafredallelecounts \given \sptree, \murate, \gfreq)$)
is calculated the same as in \citet{Bryant2012} and \citet{Oaks2018ecoevolity}.


\begin{linenomath}
For a given taxon, we can calculate the probability of all \nloci{} loci
from which we have data given the species tree and other parameters by
assuming independence among loci (conditional on the species tree) and
taking the product over them,
\begin{equation}
    \pr(\comparisondata \given \sptree, \murate, \gfreq)
    =
    \prod_{i=1}^{\nloci}
    \pr(\leafallelecounts[i], \leafredallelecounts[i] \given \sptree, \murate, \gfreq).
    \label{eq:comparisonlikelihood}
\end{equation}
We assume we have sampled biallelic data from \ncomparisons{} taxa, which can
be an arbitrary mix of
(1) two populations or species for which \comparisonetime represents
the time they diverged, or
(2) one population for which \comparisonetime represents the time
of a change in population size.
The likelihood across all \ncomparisons{} taxa is simply the product of the
likelihood of each taxon,
\begin{equation}
    \pr(
    \alldata
    \given
    \sptrees,
    \murates,
    \gfreqs)
    =
    \prod_{i=1}^{\ncomparisons}
    \pr(\comparisondata[i] \given \sptree[i], \murate[i], \gfreq[i]),
    \label{eq:collectionlikelihood}
\end{equation}
where
$\alldata = \comparisondata[1], \comparisondata[2], \ldots, \comparisondata[\ncomparisons]$,
$\sptrees = \sptree[1], \sptree[2], \ldots, \sptree[\ncomparisons]$,
$\murates = \murate[1], \murate[2], \ldots, \murate[\ncomparisons]$,
and
$\gfreqs = \gfreq[1], \gfreq[2], \ldots, \gfreq[\ncomparisons]$.
As described in \citet{Oaks2018ecoevolity},
if constant characters are not sampled for a taxon, we condition the likelihood
for that taxon on only having sampled variable characters.
\end{linenomath}


\subsection{Bayesian inference}

\begin{linenomath}
As described by \citet{Oaks2018ecoevolity},
we treat the number of events (population-size changes and/or divergences)
and the assignment of taxa to those events as
random variables under a Dirichlet process \citep{Ferguson1973,
    Antoniak1974}.
We use \etimesets to represent the partitioning of taxa to events,
which we will also refer to as the ``event model.''
The concentration parameter, \concentration, controls how clustered the
Dirichlet process is, and determines the probability of all possible \etimesets
(i.e., all possible set partitions of taxa to $1, 2, \ldots, \ncomparisons$ events).
We use \etimes to represent the unique times of events in \etimesets.
Using this notation, the posterior distribution of our 
Dirichlet-process model is
\begin{equation}
\begin{split}
    & \pr(
    \concentration,
    \etimes,
    \etimesets,
    \collectionpopsizes,
    \murates,
    \gfreqs
    \given
    \alldata
    % \basedistribution
    )
    = \\
    & \frac{
        \pr(
        \alldata
        \given
        \etimes,
        \etimesets,
        \collectionpopsizes,
        \murates,
        \gfreqs
        )
        \pr(\etimes \given \etimesets)%, \basedistribution)
        \pr(\etimesets \given \concentration)
        \pr(\concentration)
        \pr(\collectionpopsizes)
        \pr(\murates)
        \pr(\gfreqs)
    }{
        \pr(
        \alldata%,
        % \basedistribution
        )
    },
    \label{eq:bayesruleexpanded}
\end{split}
\end{equation}
where
\collectionpopsizes
is the collection of the effective population sizes (\comparisonpopsizes{})
across all of the pairs.
\end{linenomath}

\subsubsection{Priors}

\paragraph{Prior on the concentration parameter}
Our implementation allows for a hierarchical approach to accommodate
uncertainty in the concentration parameter of the Dirichlet process
by specifying a gamma distribution as a hyperprior on
\concentration \citep{Escobar1995,Heath2011}.
Alternatively, \concentration can also be fixed to a particular value,
which is likely sufficient when the number of pairs is small.

\paragraph{Prior on the divergence times}
Given the partitioning of taxa to events, we use a gamma
distribution for the prior on the time of each event,
$\etime \given \etimesets \sim \distgamma(\cdot, \cdot)$.

\paragraph{Prior on the effective population sizes}
We use a gamma distribution as the prior on
the effective size of each descendant population of each taxon.
Following \citet{Oaks2018ecoevolity},
we use a gamma distribution on the effective size of the ancestral population
\emph{relative} to the size of the descendant population(s), which we
denote as \rootrelativepopsize.
For a taxon with two descendant populations (i.e., a divergence comparison), the
ancestral population size is relative to the mean of the descendant
populations.
For a taxon with only one descendant population (i.e., a demographic
comparison), the ancestral populations is relative to the size of that
descendant.
% The goal of this approach is to allow more informative priors on the root
% population size; we often have stronger prior expectations for the relative
% size of the ancestral population than the absolute size.
% This is important, because the effective size of the ancestral population is a
% difficult nuisance parameter to estimate and can be strongly correlated with
% the divergence time.
% For example, if the divergence time is so old such that all the gene copies
% of a locus coalesce within the descendant populations, the locus
% provides very little information about the size of the ancestral
% population.
% As a result, a larger ancestral population and more recent divergence will have
% a very similar likelihood to a small ancestral population and an older
% divergence.
% Thus, placing more prior density on reasonable values of the ancestral
% population size can help improve the precision of divergence-time estimates.

\paragraph{Prior on mutation rates}
We follow the same approach explained by \citet{Oaks2018ecoevolity} to model
mutation rates across taxa.
The decision about how to model mutation rates is extremely important for any
comparative phylogeographic approach that models taxa as disconnected
species trees
\citep[\fig{}~\ref{fig:modelCartoon}; e.g.,][]{Hickerson2006,Hickerson2007,Huang2011,Chan2014,Oaks2014dpp,Xue2015,Burbrink2016,Xue2017,Gehara2017,Oaks2018ecoevolity}.
Time (\etime) and mutation rate (\murate) are inextricably linked, and because
the comparisons are modeled as separate species trees, the data cannot
inform the model about relative or absolute differences in \murate among the
comparisons.
We provide flexibility to the investigator to fix or place prior probability
distributions on the relative or absolute rate of mutation for each comparison.
However, if one chooses to accommodate uncertainty in the mutation rate of one
or more comparisons, the priors should be strongly informative.
Because of the inextricable link between rate and time,
placing a weakly informative prior on a comparison's mutation rate prevents
estimation of the time of its demographic change or divergence,
which is the primary goal.

\paragraph{Prior on the equilibrium state frequency}
Recoding four-state nucleotides to two states requires some arbitrary
decisions, and whenever $\gfreq \neq 0.5$, these decisions can affect
the likelihood of the model \citep{Oaks2018ecoevolity}.
Because DNA is the dominant character type for genomic data, we assume that
$\gfreq = 0.5$ in this paper.
This makes the CTMC model of character-state substitution a two-state analog of
the ``JC69'' model \citep{JC1969}.
However, if the genetic markers collected for one or more taxa are naturally
biallelic, the frequencies of the two states can be meaningfully estimated, and
our implementation allows for a beta prior on \gfreq in such cases.
This makes the CTMC model of character-state substitution a two-state general
time-reversible model \citep{Tavare1986}.

\subsubsection{Approximating the posterior with MCMC}

We use Markov chain Monte Carlo (MCMC) algorithms to sample from the joint
posterior in Equation~\ref{eq:bayesruleexpanded}.
To sample across event models (\etimesets) during the MCMC chain, we use the
Gibbs sampling algorithm (Algorithm 8) of \citet{Neal2000}.
We also use univariate and multivariate Metropolis-Hastings algorithms
\citep{Metropolis1953,Hastings1970} to update the model,
the latter of which are detailed in \citet{Oaks2018ecoevolity}.

\subsection{Software implementation}
The \cpp source code for \ecoevolity is freely available from
\url{https://github.com/phyletica/ecoevolity} and includes an extensive test
suite.
From the \cpp source code, two primary command-line tools are compiled:
(1) \ecoevolity, for performing Bayesian inference under the model described
above,
and
(2) \simcoevolity for simulating data under the model described above.
Documentation for how to install and use the software is available at
\url{http://phyletica.org/ecoevolity/}.
We have incorporated help in pre-processing data and post-processing posterior
samples collected by \ecoevolity in the Python package \pycoevolity, which is
available at
\url{https://github.com/phyletica/pycoevolity}.
We used Version 0.3.1
(Commit 9284417)
of the \ecoevolity software package for all of our analyses.
A detailed history of this project, including all of the data and scripts
needed to produce our results, is available at
\url{https://github.com/phyletica/ecoevolity-demog-experiments}
\citep{Oaks2019CodemogZenodo}.

\subsection{The model}

We extended the software package, \ecoevolity, to accommodate two types of
temporal comparisons:
\begin{enumerate}
    \item A population that experienced a change from effective population size
        \epopsize[\rootpopindex]
        to effective size
        \epopsize[\descendantpopindex{}]
        at time \comparisonetime in the past.
        We will refer to this as a \emph{demographic comparison}
        (\fig{}~\ref{fig:modelCartoon}),
        and refer to the population before and after the change in population
        size as ``ancestral'' and ``descendant'', respectively.
    \item A population that diverged at time \comparisonetime in the past into
        two descendant populations, each with unique effective population
        sizes.
        We will refer to this as a \emph{divergence comparison}
        (\fig{}~\ref{fig:modelCartoon}).
\end{enumerate}
This allowed us to infer shared times of divergence and/or demographic change
across an arbitrary mix of demographic and divergence comparisons in a
full-likelihood, Bayesian framework.
During an ``event'' at time \etime, one or more demographic changes and/or
divergences can occur.
We estimate the number and timing of events and the assignment of comparisons
to those events under a Dirichlet-process \citep{Ferguson1973,Antoniak1974}
prior model.
The \emph{a priori} tendency for comparisons to share events is controlled by
the concentration parameter (\concentration) of the Dirichlet process.
We use Markov chain Monte Carlo (MCMC) algorithms
\citep{Metropolis1953,Hastings1970,Neal2000}
to sample from the joint posterior of the model.
See Appendix~\ref{appendix:model} for a full description of the model, and
Table~\ref{table:notation} for a key to the notation we use throughout this
paper.

\subsection{Software implementation}
The \cpp source code for \ecoevolity is freely available from
\url{https://github.com/phyletica/ecoevolity} and includes an extensive test
suite.
Documentation for how to install and use the software is available at
\url{http://phyletica.org/ecoevolity/}.
We have incorporated help in pre-processing data and post-processing posterior
samples collected by \ecoevolity into the Python package \pycoevolity, which is
available at
\url{https://github.com/phyletica/pycoevolity}.
We used Version 0.3.1
(Commit 9284417)
of the \ecoevolity software package for all of our analyses.
A detailed history of this project, including all of the data and scripts
needed to produce our results, is available at
\url{https://github.com/phyletica/ecoevolity-demog-experiments}.


\subsection{Analyses of simulated data}

\subsubsection{Assessing ability to estimate timing and sharing of demographic changes}

We used the \simcoevolity and \ecoevolity tools within the \ecoevolity software
package to simulate and analyze \datasets,
respectively, under a variety of conditions.
Each simulated \dataset comprised 500,000 characters collected from 10 diploid
individuals (20 genomes) sampled per population from each of three demographic
comparisons.
We specified the concentration parameter of the Dirichlet process so that
the mean number of events was 2 ($\concentration = 1.414216$).
We assumed the mutation rate of all three populations was equal and 1, such
that time and effective population sizes were scaled by the mutation rate.
When analyzing each simulated \dataset, we ran four MCMC chains run for 75,000
generations with a sample taken every 50 generations; we combined and
summarized the last 1000 samples of each of the four chains (the first 501
samples discarded from each chain).

\paragraph{Initial simulation conditions}

We initially simulated data under distributions we hoped comprised a mix of
conditions that were favorable and challenging for estimating the timing and
sharing of demographic changes.
For these initial conditions, we simulated \datasets with three populations
that underwent a demographic change, under five different distributions on the
relative effective size of the ancestral population
(\rootrelativepopsize; see left column of
\figs
\labelcref{fig:valsimsmodelinitial,fig:valsimsetimesinitial}):
\begin{enumerate}[label=A.\arabic*]
    \item \dgamma{10}{0.25} (4-fold population increase) \label{sims:initialFourFoldIncrease}
    \item \dgamma{10}{0.5} (2-fold population increase)  \label{sims:initialTwoFoldIncrease}
    \item \dgamma{10}{2} (2-fold population decrease)    \label{sims:initialTwoFoldDecrease}
    \item \dgamma{10}{1} (no change on average, but a fair amount of variance) \label{sims:initialCenter}
    \item \dgamma{100}{1} (no change on average, little variance) \label{sims:initialCenterNarrow}
\end{enumerate}
The last distribution was chosen to represent a ``worst-case'' scenario where
there was almost no demographic change in the history of the populations.
For the mutation-scaled effective size of the descendant populations
($\epopsize[\descendantpopindex{}]\murate$; i.e., the population size after the
demographic change),
we used a gamma distribution with a shape of 5 and mean of 0.002.
The timing of the demographic events was exponentially distributed with a mean
of 0.01 expected substitutions per site;
given the mean of our distribution on the effective size of the descendant
populations, this puts the expectation of demographic-change times in units of
$4N_e$ generations at 1.25.
This distribution on times was chosen to span times of demographic change from
very recent (i.e., most gene lineages coalesce before the change) to old (i.e.,
most gene lineages coalesce after the change).
The assignment of the three simulated populations to $1--3$ demographic events
was controlled by a Dirichlet process with a mean number of 2.0 demographic
events across the three populations.
We generated 500 \datasets under each of these five simulation conditions, all
of which were analyzed using the same simulated distributions as priors.

\paragraph{Simulation conditions chosen to improve performance}

Estimates of the timing and sharing of demographic events were quite poor
across all the initial simulation conditions (see results).
In an effort to find conditions under which the timing and sharing of
demographic changes could be better estimated, and avoid combinations
of parameter values that caused identifiability problems,
we next explored simulations under distributions on times and population sizes
offset from zero, but with much more recent demographic event times.
% Event time ~ gamma(shape=4.0, scale=0.000475, offset=0.0001); mean 0.002
% relative root size ~ gamma(shape=5.0, scale = 0.04, offset = 0.05); mean 0.25
% relative root size ~ gamma(shape=5.0, scale = 0.09, offset = 0.05); mean 0.5
% relative root size ~ gamma(shape=5.0, scale = 0.79, offset = 0.05); mean 4
% relative root size ~ gamma(shape=5.0, scale = 0.19, offset = 0.05); mean 1
% relative root size ~ gamma(shape=50.0, scale = 0.02, offset = 0.0); mean 1
For the mutation-scaled effective size of the descendant
population
($\epopsize[\descendantpopindex{}]\murate$),
we used an offset gamma distribution with a shape of 4, offset of 0.0001, and
mean of 0.0021 (accounting for the offset).
For the distribution of event times, we used a gamma distribution
with a shape of 4, offset of 0.0001, and a mean of 0.002 (accounting
for the offset; 0.25 units of $4N_e$ generations, on average).
Again, we used five different distributions on the relative effective size of
the ancestral population (see left column of
\figs
\labelcref{fig:valsimsmodelopt,fig:valsimsetimesopt}):
\begin{enumerate}[label=B.\arabic*]
    \item \dogamma{5}{0.25}{0.05} (4-fold population increase) \label{sims:optimalFourFoldIncrease}
    \item \dogamma{5}{0.5}{0.05} (2-fold population increase)  \label{sims:optimalTwoFoldIncrease}
    \item \dogamma{5}{4}{0.05} (4-fold population decrease)    \label{sims:optimalFourFoldDecrease}
    \item \dogamma{5}{1}{0.05} (no change on average, but a fair amount of variation) \label{sims:optimalCenter}
    \item \dogamma{50}{1}{0} (no change on average, little variance) \label{sims:optimalCenterNarrow}
\end{enumerate}
We generated 500 \datasets under each of these five distributions, all of which
were analyzed under priors that matched the generating distributions.

\subsubsection{Simulations to assess sensitivity to prior assumptions}

% Sim disributions:
% Event time ~ gamma(shape=4.0, scale=0.000475, offset=0.0001); mean 0.002
% relative root size ~ gamma(shape=5.0, scale = 0.04, offset = 0.05); mean 0.25
% descendant size ~ gamma(4.0, scale=0.0005, offset = 0.0001); mean 0.0021
% Priors:
% Event time ~ exponential(mean = 0.005)
% relative root size ~ Exponential(mean = 2.0)
% descendant size ~ gamma(2.0, scale 0.001); mean 0.002
Next, we simulated an additional 500 \datasets under
Condition~\ref{sims:optimalFourFoldIncrease} above.
We then analyzed each of these \datasets under ``diffuse'' prior
distributions:
\begin{itemize}
    \item $\etime \sim \dexponential{0.005}$
    \item $\rootrelativepopsize \sim \dexponential{2}$
    \item $\epopsize[\descendantpopindex{}] \sim \dgamma{2}{0.002}$
\end{itemize}
These distributions were chosen to reflect realistic amounts of prior
uncertainty about the timing of demographic changes and past and present
effective population sizes when analyzing empirical data.

For comparison, we repeated the same simulations and analyses for three pairs
of populations for which we estimated the timing and sharing of their
divergences
(i.e., three divergence comparisons rather than three demographic comparisons;
\fig{}~\ref{fig:modelCartoon}).
For these divergence comparisons, we simulated 10 sampled genomes per
population to match the total number sampled per comparison (20) as in the
demographic simulations.


\subsubsection{Simulating a mix of divergence and pop-size-change comparisons}

To explore how well our method can infer a mix of shared demographic changes
and divergence times, we simulated 500 \datasets comprised of 6 comparisons:
3 demographic comparisons and
3 divergence comparisons.
20 genomes (10 diploid individuals) were sampled from each comparison; for
divergence comparisons, 10 genomes were sampled from each of the two
populations.
We used the same simulation conditions described above for
\ref{sims:optimalTwoFoldIncrease}.
All simulated \datasets were analyzed under priors that matched
the generating distributions.


\subsubsection{Simulating linked sites}
To assess the effect of linked sites on the inference
of the timing and sharing of demographic changes,
we simulated \datasets comprising 5000 100-base-pair
loci (500,000 total characters).
The distributions on parameters were the same
as the conditions described for \ref{sims:optimalFourFoldIncrease} above.
These same distributions were used as priors when
analyzing the simulated \datasets.

% \subsection{Data-acquisition bias?}


\subsection{Empirical application to stickleback data}


\subsubsection{Assembly of loci}
We assembled the publicly available RADseq data collected by
\citet{Hohenlohe2010}
from five populations of threespine sticklebacks (\spp{Gasterosteus aculeatus})
from south-central Alaska.
After downloading the reads mapped to the stickleback genome by
\citet{Hohenlohe2010}
from Dryad
(doi:10.5061/dryad.b6vh6),
We assembled reference guided alignments of loci in Stacks v1.48
\citet{Catchen2013} with a minimum read depth of 3 identical reads per locus
within each individual and the bounded single-nucleotide polymorphism (SNP)
model with error bounds between
0.001 and 0.01.
To maximize the number of loci and minimize paralogy, we assembled each
population separately;
because \ecoevolity models each population separately
(\fig{}~\ref{fig:modelCartoon}),
the characters do not need to be orthologous across populations, only within
them.

\subsubsection{Inferring shared demographic changes with \ecoevolity}

We used a value for the concentration parameter of the Dirichlet process
that corresponds to a mean number of events of three
($\concentration = 2.22543$).
We used the following prior distributions on the timing of events and effective
sizes of populations:
\begin{itemize}
    \item $\etime \sim \dexponential{0.001}$
    \item $\rootrelativepopsize \sim \dexponential{1}$
    \item $\epopsize[\descendantpopindex{}] \sim \dgamma{2}{0.002}$
\end{itemize}
To assess the sensitivity of the results to these prior assumptions,
we also analyzed the data under two additional priors on
the concentration parameter, event times, and relative
effective population size of the ancestral population:
\begin{itemize}
    \item $\concentration = 13$ (half of prior probability on 5 events)
    \item $\concentration = 0.3725$ (half of prior probability on 1 event)
    \item $\etime \sim \dexponential{0.0005}$
    \item $\etime \sim \dexponential{0.01}$
    \item $\rootrelativepopsize \sim \dexponential{0.5}$
    \item $\rootrelativepopsize \sim \dexponential{0.1}$
\end{itemize}

For each prior setting, we ran 10 MCMC chains for 150,000 generations, sampling
every 100 generations; we did this using all the sites in the assembled
stickleback loci and only SNPs.
To assess convergence and mixing of the chains, we calculated the potential
scale reduction factor \citep[PSRF; the square root of Equation 1.1 in][]{Brooks1998}
and effective sample size \citep{Gong2014} of all continuous parameters and the
log likelihood using the \texttt{pyco-sumchains} tool of \pycoevolity
(Version 0.1.2 Commit 89d90a1).
We also visually inspected the sampled log likelihood and parameters values
over generations with the program Tracer Version 1.6 \citep{Tracer16}.
The MCMC chains for all analyses converged almost immediatley; we
conservatively removed the first 101 samples from each chain, resulting in
14,000 samples from the posterior (1400 samples from 10 chains) for each
analysis.



\section{Results \& Discussion}

\subsection{Analyses of simulated data}

\subsubsection{Estimating timing and sharing of demographic changes}

% Under our initial simulation conditions,
% that we
% selected to comprise a mix of conditions both favorable and challenging
% for estimating the timing and sharing of demographic changes
Despite our attempt to capture a mix of favorable and challenging parameter
values in our initial simulation condtions (Table~S\ref{table:prelimsimconditions}),
estimates of the timing and sharing of demographic events were quite poor
across all the simulation conditions we initially explored
(\figs{}~S\labelcref{fig:valsimsetimesinitial,fig:valsimsmodelinitial,fig:valsimsasizesinitial,fig:valsimsdsizesinitial}).
Even after we tried selecting simulation conditions that are more favorable for
identifying the event times, estimates of the timing and
sharing of demographic events remain quite poor
(\figs \ref{fig:valsimsetimesopt} and \ref{fig:valsimsmodelopt}).
Under the recent (but not too recent) 4-fold population-size increase (on
average) scenario, we do see better estimates of the times of the demographic
change
(\vsimfourinc; top row of \fig{}~\ref{fig:valsimsetimesopt}),
but the ability to identify the correct number of events and the assignment of
the populations to those events remains quite poor;
the correct model is preferred only 57\% of the time, and the
median posterior probability of the correct model is only 0.42
(top row of \fig{}~\ref{fig:valsimsmodelopt}).
Under the most extreme population retraction scenario
(\vsimfourdec; 4-fold, on average),
the correct model is preferred only 40\% of the time, and the median
posterior probability of the correct model is only 0.26
(middle row of \fig{}~\ref{fig:valsimsmodelopt}).
Estimates are especially poor when using only variable characters,
so we focus on the results using all characters
(second versus third column of \figs
\ref{fig:valsimsetimesopt}
and
\ref{fig:valsimsmodelopt});
we also see worse estimates of population sizes when excluding invariant
characters
(\figs
S\ref{fig:valsimsasizesopt}
and
S\ref{fig:valsimsdsizesopt}).

\ifembed{
\embedHeightFigure{0.8}{../../results/cropped-grid-event-time-t-4_0-0_000475-0_0001.pdf}{
    \footnotesize
    The accuracy and precision of time estimates of demographic changes (in
    units of expected subsitutions per site) when data were simulated and
    analyzed under the same distributions (Table~\ref{table:simconditions}).
    % , and event
    % times are gamma-distributed with a shape of 4, offset of 0.0001, and mean
    % of 0.002
    % (0.3 units of $4N_e$ generations).
    \validationsimsthreecolumndescription
    Each plotted circle and associated error bars represent the posterior mean
    and 95\% credible interval.
    Estimates for which the potential-scale reduction factor was greater than
    1.2 \citep{Brooks1998} are highlighted in orange.
    Each plot consists of 1500 estimates---500 simulated \datasets, each with
    three demographic comparisons.
    \accuracyscatterplotannotations{\comparisonetime{}}
    \weusedmatplotlib
}{fig:valsimsetimesopt}

}{}

\ifembed{
\embedHeightFigure{0.8}{../../results/cropped-grid-model-t-4_0-0_000475-0_0001.pdf}{
    \footnotesize
    The performance of estimating the model of demographic changes when data
    are simulated and analyzed under the same distributions
    (Table~\ref{table:simconditions}).
    % misspecification), and event times are
    % gamma-distributed with a shape of 4, offset of 0.0001, and mean of 0.002
    % (0.3 units of $4N_e$ generations).
    \validationsimsthreecolumndescriptionmodels
    Each plot shows the results of the analyses of 500 simulated \datasets,
    each with three demographic comparisons;
    the number of \datasets that fall within each possible cell
    of true versus estimated model is shown, and cells with
    more \datasets are shaded darker.
    \modelplotannotations
    \weusedmatplotlib
}{fig:valsimsmodelopt}

}{}

Under the ``worst-case'' scenario of little population-size change
(\vsimnochange; bottom row of \figs \ref{fig:valsimsetimesopt} and
\ref{fig:valsimsmodelopt}),
our method is unable to identify the timing or model of demographic change.
As expected, under these conditions our method returns the prior on the timing
of events (bottom row of \fig{}~\ref{fig:valsimsetimesopt})
and always prefers either a model with a single, shared demographic
event (model "000") or independent demographic changes (model "012";
bottom row of \fig{}~\ref{fig:valsimsmodelopt}).
This is expected behavior, because there is essentially no information in the
data about the timing of demographic changes, and a Dirichlet process with a
mean of 2.0 demographic events, puts approximately 0.24 of the prior
probability on the models with one and three events, and 0.515 prior
probability on the three models with two events (approximately 0.17 each).
Thus, with little information, the method samples from the prior distribution
on the timing of events, and prefers one of the two models with larger prior
probability.

Doubling the number of individuals sampled per population to 20 had very little
affect on the results
(\fig{}~S\ref{fig:valsimspopsamplesize}).
Likewise, doubling the number of demographic comparisons to six had no affect
on the accuracy or prescision of estimating the timing of demographic changes
or effective population sizes
(Rows 1, 3, and 4 of \fig{}~S\ref{fig:valsimscompsamplesize}).
The ability to infer the correct number of demographic events,
and assignment of population to the events (\etimesets),
is much worse when there are six comparisons, which is
not surprising given that the number of possible assignments
of populations to events is 203 for six comparisions, compared
to five for three comparisons \citep{Bell1934}.

The 95\% credible intervals of all the parameters
covers the true value approximately 95\% of the time
(\figs
\ref{fig:valsimsetimesopt},
S\ref{fig:valsimsasizesopt},
and
S\ref{fig:valsimsdsizesopt}).
Given that our priors match the underlying distributions that generated the
data, this coverage behavior is expected, and is an important validation
of our implemetation of the model and corresponding MCMC algorighms.


\subsubsection{Sensitivity to prior assumptions}

Above, we observe the best estimates of the timing and sharing of demographic
events under narrow distributions on the relative effective size of the
ancestral population
(\vsimfourinc; top row of \figs
\labelcref{fig:valsimsetimesopt,fig:valsimsmodelopt}),
which were used to both simulate the data and as the prior
when those data were analyzed.
Thus, the improved behavior could be due to these narrow prior distributions that
are unrealistically informative for empirical studies, for which there is
usually little \emph{a priori} information about past population sizes.
When we analyze data under more realistic, diffuse priors, estimates of the
timing and sharing of \emph{demographic} events deteriorate,
whereas estimates of the timing and sharing of \emph{divergence} events remain
robust
% Our results show that the better performance under these distributions was at
% least partially caused by greater prior information, and that
% inference of shared demographic events is much more sensitive to
% prior assumptions than shared divergences
(\figs \labelcref{fig:valsimsmodeldiffuse,fig:valsimsetimesdiffuse}).
% To determine whether the better performance under these distributions was
% caused by more informative data or priors, we simulated \datasets under a
% narrow distribution on the relative ancestral population size, and analyzed
% these \datasets under more realistic, ``diffuse'' prior distributions on
% populations sizes and event times.
% This will also allow us to assess how sensitive estimates of the timing and
% sharing of demographic events are to prior assumptions.
% For comparison, we repeated the same simulations for pairs of populations for
% which we estimated the timing and sharing of their divergences (the same total
% number of gene copies were sampled from each pair).
% Sim disributions:
% Event time ~ gamma(shape=4.0, scale=0.000475, offset=0.0001); mean 0.002
% relative root size ~ gamma(shape=5.0, scale = 0.04, offset = 0.05); mean 0.25
% descendant size ~ gamma(4.0, scale=0.0005, offset = 0.0001); mean 0.0021
% Priors:
% Event time ~ exponential(mean = 0.005)
% relative root size ~ Exponential(mean = 2.0)
% descendant size ~ gamma(2.0, scale 0.001); mean 0.002
Specifically, the precision of time estimates of demographic changes decreases
substantially under the diffuse priors
(top two rows of \fig{}~\ref{fig:valsimsetimesdiffuse}),
whereas the precision of the divergence-time estimates
is high and largely unchanged under the diffuse priors
(bottom two rows of \fig{}~\ref{fig:valsimsetimesdiffuse}).
We see the same patterns in the estimates of population sizes
(\figs
S\ref{fig:valsimsasizesdiffuse}
and
S\ref{fig:valsimsdsizesdiffuse}).

\ifembed{
\embedWidthFigure{1.0}{../../results/cropped-grid-event-time-diffuse-prior.pdf}{
    The accuracy and precision of time estimates of demographic changes
    (top two rows)
    versus divergences
    (bottom two rows)
    when the priors are correct
    (first and third rows)
    versus when the priors are diffuse
    (second and fourth rows).
    Time is measured in units of expected subsitutions per site. 
    \diffusesimsfourcolumndescription
    Each plotted circle and associated error bars represent the posterior mean
    and 95\% credible interval.
    Estimates for which the potential-scale reduction factor was greater than
    1.2 \citep{Brooks1998} are highlighted in orange.
    Each plot consists of 1500 estimates---500 simulated \datasets, each with
    three
    demographic comparisons (Rows 1--2) or
    divergence comparisons (Rows 3--4).
    \accuracyscatterplotannotations{\comparisonetime{}}
    The first row of plots are repeated from
    \fig{}~\ref{fig:valsimsetimesopt}
    for comparison.
    \weusedmatplotlib
}{fig:valsimsetimesdiffuse}

}{}

Furthermore, under the diffuse priors, the probability of inferring the correct
model of demographic events decreases from 0.57 to 0.434 when all characters
are used, and from 0.36 to 0.284 when only variable characters are used
(top two rows of \fig{}~\ref{fig:valsimsmodeldiffuse}).
The median posterior probability of the correct model also decreases from
0.422 to 0.292 when all characters are used,
and from 0.231 to 0.178 when only variable characters are used
(top two rows of \fig{}~\ref{fig:valsimsmodeldiffuse}).
Most importantly, we see a strong bias toward underestimating the number of
events under the more realistic diffuse priors
(top two rows of \fig{}~\ref{fig:valsimsmodeldiffuse}).
In comparison, the inference of shared divergence times is much more accurate,
precise, and robust to the diffuse priors
(bottom two rows of \fig{}~\ref{fig:valsimsmodeldiffuse}).
When all characters are used, under both the correct and diffuse
priors, the correct divergence model is preferred over 91\% of the time,
and the median posterior probability of the correct model is over
0.93.

\ifembed{
\embedWidthFigure{1.0}{../../results/cropped-grid-model-diffuse-prior.pdf}{
    The performance of estimating the model of demographic changes
    (top two rows)
    versus model of divergences
    (bottom two rows)
    when the priors are correct
    (first and third rows)
    versus when the priors are diffuse
    (second and fourth rows).
    \diffusesimsfourcolumndescriptionmodels
    Each plot shows the results of the analyses of 500 simulated \datasets,
    each with three
    demographic comparisons (Rows 1--2) or
    divergence comparisons (Rows 3--4);
    the number of \datasets that fall within each possible cell
    of true versus estimated model is shown, and cells with
    more \datasets are shaded darker.
    \modelplotannotations
    The first row of plots are repeated from
    \fig{}~\ref{fig:valsimsmodelopt}
    for comparison.
    \weusedmatplotlib
}{fig:valsimsmodeldiffuse}

}{}

Results are very similar whether the distribution on the
ancestral population size is peaked around a four-fold population
expansion or contraction
(Conditions \msimfourinc and \msimfourdec;
top two rows of \figs
\ref{fig:valsimsetimesdiffuseonly},
\ref{fig:valsimsmodeldiffuseonly},
S\ref{fig:valsimsasizesdiffuseonly},
and
S\ref{fig:valsimsdsizesdiffuseonly})
Likewise, even when population expansions
and contractions are 10-fold, the ability to infer
the timing and sharing of these events remains
poor
(Conditions \msimteninc and \msimtendec;
bottom two rows of figs
\ref{fig:valsimsetimesdiffuseonly} and 
\ref{fig:valsimsmodeldiffuseonly}).
This is not surprising when reflecting on the first principles of this
inference problem.
While it may seem intuitive that more dramatic changes in the rate
of coalescence should be easier to detect, such large changes
will cause fewer lineages to coalesce
after (in the case of a dramatic population expansion)
or
before (in the case of a dramatic population contraction)
the change in population size.
This reduces the information about the rate of coalescence on one side of the
demographic change and thus the magnitude and timing of the change in effective
population size.
Thus, the gain in information in the data is expected to plateau (and even
decrease as we see for Condition \msimtendec in
\fig{}~\ref{fig:valsimsmodeldiffuseonly})
as the magnitude of the change in effective population size increases.

\ifembed{
    \embedWidthFigure{1.0}{../../results/cropped-grid-model-diffuse-prior-expanded.pdf}{
    The performance of estimating the model of demographic changes when the
    prior distributions are diffuse
    (Conditions \missimcondition{1}--\missimcondition{4};
    Table~\ref{table:simconditions}).
    \diffusesimsonlyfourcolumndescriptionmodels
    Each plot consists of 1500 estimates---500 simulated \datasets, each with
    three demographic comparisons;
    the number of \datasets that fall within each possible cell
    of true versus estimated model is shown, and cells with
    more \datasets are shaded darker.
    \modelplotannotations
    The first row of plots are repeated from
    \fig{}~\ref{fig:valsimsetimesdiffuse}
    for comparison.
    \weusedmatplotlib
}{fig:valsimsmodeldiffuseonly}

}{}

\ifembed{
    \embedWidthFigure{1.0}{../../results/cropped-grid-event-time-diffuse-prior-expanded.pdf}{
    The accuracy and precision of time estimates of demographic changes when
    the prior distributions are diffuse
    (Conditions \missimcondition{1}--\missimcondition{4};
    Table~\ref{table:simconditions}).
    Time is measured in units of expected subsitutions per site. 
    \diffusesimsonlyfourcolumndescription
    Each plotted circle and associated error bars represent the posterior mean
    and 95\% credible interval.
    Estimates for which the potential-scale reduction factor was greater than
    1.2 \citep{Brooks1998} are highlighted in orange.
    Each plot consists of 1500 estimates---500 simulated \datasets, each with
    three
    demographic comparisons.
    \accuracyscatterplotannotations{\comparisonetime{}}
    The first row of plots are repeated from
    \fig{}~\ref{fig:valsimsetimesdiffuse}
    for comparison.
    \weusedmatplotlib
}{fig:valsimsetimesdiffuseonly}

}{}

\subsection{Inferring a mix of shared divergences and demographic changes}

When demographic and divergence comparisons are analyzed separately, the
performance of estimating the timing and sharing of demographic changes and
divergences is dramatically different, with the latter being much more accurate
and precise than the former
(e.g., see
\figs
\ref{fig:valsimsetimesdiffuse}
and
\ref{fig:valsimsmodeldiffuse}).
One might hope that
if we analyze a mix of demographic and divergence comparisons,
the informativeness of the divergence times can help ``anchor'' and
improve the estimates of shared demographic changes.
However, our results from simulating \datasets comprising a mix of three
demographic and three divergence comparisons rule out this possibility.
% To explore this possibility, we simulated datasets with six comparisons,
% comprising a mix of three populations that experienced a demographic change and
% three pairs of populations that diverged.
% For these mixed-comparison simulations, we used a gamma distribution on event
% times with a shape of 4, offset of 0.0001, and a mean of 0.002 substitutions per
% site (accounting for the offset; 0.25 units of $4N_e$ generations on average).
% For the distribution on the relative size of the ancestral population,
% we used a gamma distribution with a shape of 5, an offset of 0.05, and a mean
% of 0.5; a 2-fold population size increase on average.
% These are the same distributions used for the second row of
% \figs
% \ref{fig:valsimsetimesopt}
% and
% \ref{fig:valsimsmodelopt}.

% We summarized the timing and sharing of the demographic changes
% separately from the divergences so that we could determine whether the
% divergence-time estimates could help improve the estimates of the
% times of the demographic changes.
When analyzing a mix of demographic and divergence comparisons, the ability to
infer the timing and sharing of demographic changes remains poor, whereas
estimates of shared divergences remain accurate and precise
(\fig{}~\ref{fig:mixsims}).
The estimates of the timing and sharing of demographic events are nearly
identical to when we simulated and analyzed only three demographic comparisons
under the same distributions on event times and population sizes
(compare left column of \fig{}~\ref{fig:mixsims}
to the second row of \figs
\ref{fig:valsimsetimesopt}
and
\ref{fig:valsimsmodelopt}).
The same is true for the estimates of population sizes
(\fig{}~S\ref{fig:mixsimsfull}).
Thus, there does not appear to be any mechanism by which the more informative
divergence-time estimates ``rescue'' the estimates of the timing and sharing of
the demographic changes.

\ifembed{
\embedWidthFigure{1.0}{../../results/cropped-grid-mixed-comparisons-allsites-short.pdf}{
    Analyses of six taxa comprising a mix of three populations that
    experienced a demogrpahic change and three pairs of populations that
    diverged.
    The performance of estimating the timing (top row) and sharing (bottom row)
    of events are shown separately for the three populations that experienced a
    demographic change (left column) and the three pairs of populations that
    diverged (right column).
    The plots of the demographic comparisons (left column) are comparable
    to the second column of \figs
    \ref{fig:valsimsetimesopt}
    and
    \ref{fig:valsimsmodelopt};
    the same priors on event times and ancestral population size were used.
    Time estimates for which the potential-scale reduction factor was greater than
    1.2 \citep{Brooks1998} are highlighted in orange.
    Each plot shows the results from 500 simulated \datasets, each with
    six taxa.
    % \accuracyscatterplotannotations{\comparisonetime{}}
    \weusedmatplotlib
}{fig:mixsims}

}{}


\subsection{The effect of linked sites}

Most reduced-representation genomic datasets are comprised of loci of
contiguous, linked nucleotides.
Thus, when using the method presented here that assumes each character is
effectively unlinked,
% (i.e., evolved along a gene tree that is independent from other characters,
% conditional on the population history)
one either has to violate this assumption, or discard all but (at most) one
site per locus.
Given that all the results above indicate better estimates when all
characters are used (compared to using only variable characters), we
simulated linked sites to determine which strategy is better:
analyzing all linked sites and violating the assumption of unlinked characters,
or discarding all but (at most) one variable character per locus.

% To do this, we repeated the most favorable simulation conditions (on average
% 4-fold population expansion; see first row of
% \figs
% \ref{fig:valsimsetimesopt}
% and
% \ref{fig:valsimsmodelopt}),
% except that 100 characters were simulated along 5000
% simulated gene trees.
% In other words, the simulated \datasets comprised 5000
% 100-base-pair loci, rather than 500,000 unlinked sites.
The results are almost identical to when all the sites were unlinked
(compare first row of
\figs
\ref{fig:valsimsetimesopt}
and
\ref{fig:valsimsmodelopt}
to
\fig{}~\ref{fig:locisims},
and the first row of
\figs
S\ref{fig:valsimsasizesopt}
and
S\ref{fig:valsimsdsizesopt}
to the bottom two rows of
\fig{}~S\ref{fig:locisimsfull}).
Thus, violating the assumption of unlinked sites has little
affect on the estimation of the timing and sharing of
demographic changes;
this is also true for estimates of population sizes
(\fig{}~S\ref{fig:locisimsfull}).
This is consistent with the findings of
\citet{Oaks2018ecoevolity} and
\citet{Oaks2018paic}
that linked sites had little affect on the estimation of
shared divergence times.
These results suggest that analyzing all of the sites in loci assembled from
reduced-representation genomic libraries (e.g., sequence-capture or RADseq
loci) is a better strategy than excluding sites to avoid violating the
assumption of unlinked characters.

\ifembed{
\siFigure{1.0}{../../results/cropped-grid-loci-short.pdf}{
    Estimates of the timing (top row) and sharing (bottom row) of demographic
    changes when using all characters (left column) or only unlinked variable
    characters (right column) from \datasets simulated with 5000 loci of 100
    linked bases from three demographic comparisons.
    The plots are comparable to the first row of \figs
    \ref{fig:valsimsetimesopt}
    and
    \ref{fig:valsimsmodelopt};
    the only difference is the linkage of characters into loci.
    Time estimates for which the potential-scale reduction factor was greater than
    1.2 \citep{Brooks1998} are highlighted in orange.
    Each plot shows the results from 500 simulated \datasets, each with
    three demographic comparisons.
    \weusedmatplotlib
}{fig:locisims}

}{}


% \subsection{Data-acquisition bias?}


\subsection{Reassessing the co-expansion of stickleback populations}

Using an ABC analog to the model of shared demographic changes developed here,
\citet{Xue2015} found very strong support (0.99 posterior probability) that
five populations of threespine sticklebacks (\spp{Gasterosteus aculeatus})
from south-central Alaska recently
co-expanded.
This inference was based on the publicly available RADseq data collected by
\citet{Hohenlohe2010}.
% (\url{https://trace.ncbi.nlm.nih.gov/Traces/sra/?study=SRP001747};
% NCBI Short Read Archive accession numbers SRX015871--SRX015877).
We re-assembled and analyzed these data under our full-likelihood
Bayesian framework, both using all sites from assembled loci
and only SNPs.


Stacks produced a concatenated alignment with
2,115,588,
2,166,215,
2,081,863,
2,059,650, and
2,237,438
total sites, of which
118,462,
89,968,
97,557,
139,058, and
103,271
were variable for the Bear Paw Lake, Boot Lake, Mud Lake, Rabbit Slough, and
Resurrection Bay stickleback populations respectively.
When analyzing all sites from the assembled stickleback
RADseq data, we find strong support for five independent
population expansions (no shared demographic events;
\fig{}~\ref{fig:sticklesummary}).
In sharp contrast, when analyzing only SNPs, we find
support for a single, shared, extremely recent population expansion
(\fig{}~\ref{fig:sticklesummary}).
The support for a single, shared event is consistent with the results from our
simulations using diffuse priors and only including SNPs, which showed
consistent, spurious support for a single event
(Row 2 of \fig{}~\ref{fig:valsimsmodeldiffuse}).
These results are relatively robust to a broad range of prior
assumptions
(\figs
S\labelcref{%
fig:sticklebydppevents,fig:sticklebydpptimes,fig:sticklebydppsizes,%
fig:sticklebytimeevents,fig:sticklebytimetimes,fig:sticklebytimesizes,%
fig:sticklebysizeevents,fig:sticklebysizetimes,fig:sticklebysizesizes,%
}).

\ifembed{
\mFigure{1.0}{../../results/stickleback-plots/cropped-grid-stickleback-summary.pdf}{
    Estimates of the number (Row 1), timing (Row 2), and magnitude (Row 3)
    of demographic events across five stickleback populations, when using all
    sites (left column) or only variable sites (right column) and an exponential with mean of 0.001 as the prior on event times, an exponential with mean of 1 for the prior on the relative ancestral effective population size, and a gamma with shape of 2 and mean of 0.002.
    Analyses of six taxa comprising a mix of three populations that
    experienced a demogrpahic change and three pairs of populations that
    diverged.
    For the number of events (Row 1), the light and dark bars represent the
    prior and posterior probabilities, respectively.
    Time (Row 2) is in units of expected subsitutions per site.
    For the violin plots, each plotted circle and associated error bars
    represent the posterior mean and 95\% credible interval.
    Bar graphs were generated with ggplot2 Version 2.2.1 \citep{ggplot2};
    violin plots were generated with matplotlib Version 2.0.0
    \citep{matplotlib}.
}{fig:sticklesummary}

}{}

When using only SNPs, estimates of the timing of the single, shared demographic
event are essentially at the minimum of zero, suggesting that there is little
information about the timing of any demographic changes in the SNP data alone.
This is consistent with results of \citet{Xue2015} where the single, shared
event was also estimated to have occurred at the minimum (1000 generations) of
their uniform prior on the timing of demographic changes.
In light of our simulation results, the support for a single event based solely
on SNPs, seen here and in \citet{Xue2015}, is likely caused by a combination of
(1) misspecified priors, and
(2) the lack of information about demographic history when invariant characters
are discarded.
Our estimates using all of the sites in the stickleback RADseq loci should be
the most accurate, according to our results from simulated data.
However, the unifying theme from our simulations is that all estimates of
shared demographic events tend to be poor and should be treated with a lot of
skepticism.
% Given our results when using all the information from data simulated under
% conditions favorable for estimating the timing of demographic changes, strong
% posterior support for any particular scenario is almost certainly spurious.


\subsection{Comparison to previous models of shared demographic changes}
% TODO: compare and contrast ecoevolity method to Chan, Xue, and Burbrink
% methods.
\citet{Chan2014}:
\begin{itemize}
    \item Assumes only population expansions
    \item Assumes at most one simultaneous event (taxa are either part of the
        event or expand independently)
    \item Implemenation for single locus data
\end{itemize}

\citet{Gehara2017}:
Same as \citet{Chan2014} but play around with various ways of modelling
overdispersed timing of events.

\citet{Xue2015}:
\begin{itemize}
    \item Implementation for genomic data (uses SFS as summary statistic)
    \item Assumes only pop expansions
    \item Assumes at most one simult event (when applied to sticklebacks)
\end{itemize}

\citet{Xue2017}
\begin{itemize}
    \item Implementation for genomic data (uses SFS as summary statistic)
    \item Allows expansion and contraction
    \item General in regard to the number of shared events
    \item Time buffering
\end{itemize}

\citet{Prates2016}:
\begin{itemize}
    \item Populations that expand/contract are prespecified? No clear.
    \item Assumes at most one simultaneous event (taxa are either part of the
        event or expand independently)
\end{itemize}

\subsection{Biological realism of the model of shared demographic changes}
The model of shared population-size changes we present here, and used in
previous research \citep{Chan2014,Xue2015,Gehara2017} is quite unrealistic in
number of respects.
First, modeling the demographic history of a population with a single,
instantaneous change in population size does not reflect the continuous and
complex demographic changes most populations of organisms experience through
time.
However, this simple model is correct in our simulated data, and yet our method
struggles to accurately infer the timing and sharing of these single, dramatic,
instantaneous changes in effective population size.
Very unrealistic.
Incorporating more demographic realism into the model will introduce more
variation and thus make the inference problem even more difficult.
We expect more complicated demography.

Also, we expect that most processes that cause shared divergences and/or
demographic changes across species will produce some amount of temporal
variation in the effect among the species.
Thus, our model of simultaneous evolutionary events that affect multiple
species at the same instant is not biologically plausible.
If this lack of realism is problematic, it should cause the method to
overestimate the number events by identifying the temporal variation among
species affected by the same process as multiple events.
However, what we see here
(\fig{}~\ref{fig:valsimsmodeldiffuseonly}
and what has been shown previously
\citep{Oaks2012,Oaks2014reply,Oaks2014dpp,Oaks2018ecoevolity,Oaks2018paic}
is the opposite;
even when we model shared events as simultaneous, methods tend to underestimate
(and almost never overestimate) the number of events.
We do see overestimates when there is little information in the data
and the posterior largely reflects the prior 
(e.g., bottom two rows of \fig{}~\ref{fig:valsimsmodelopt}).
However, this is only true when the prior distributions match the true
underlying distributions that generated the data, and these overestimates would
be easy to identify in practice by testing for prior sensitivity and noticing
that the posterior probabilities for models were similar to the prior
probabilities (i.e., small Bayes factors).
Previous researchers
\citep{Overcast2017,Gehara2017,Xue2017}
have attempted to introduce realism into these comparative models by allowing
temporal variation among species affected by the same event by assuming that
processes of diversification and demographic change are temporally
overdispersed.
However, allowing temporal variation within events will only increase the
tendency of these methods to underestimate the number of events.
Also, assuming processes that cause shared evolutionary responses are somehow
conveniently staggered over time (overdispersed) seems biologically
implausible.

\section{Conclusions}

% We are threading a needle.
There is a narrow temporal window within which we can reasonably estimate the
time of a demographic change.
The width of this window is determined by how deep in the past the change
occurred relative to the effective size of the population (i.e., in coalescent
units).
If too old or recent, there are too few coalescence events before or after the
demographic change, respectively, to provide information about the effective
size of the population.
When we are careful to simulate data within this window, and the change in
population size is large enough, we can estimate the time of the demographic
changes reasonably well
(e.g., see the top row of \fig{}~\ref{fig:valsimsetimesopt}).
However, even under these condtions, the ability to correctly infer the number
of demographic events, and the assignment of populations to those events is
quite limited
(\fig{}~\ref{fig:valsimsmodelopt}).
When only variable characters are analyzed (i.e., SNPs), estimates of
the timing and sharing of demographic changes are consistently bad; we see this
across all the conditions we simulated.
Most alarmingly, when the priors are more diffuse than the distributions that
generated the data, as will be true in most empirical applications, there is a
strong bias toward estimating too few demographic events
(i.e., over-clustering comparisons to demographic events;
Row 2 of \fig{}~\ref{fig:valsimsmodeldiffuse}),
especially when only variable characters are analyzed.
These results help explain the stark contrast we see in our results from the
stickleback RADseq data when including versus excluding constant sites
(\fig{}~\ref{fig:sticklesummary}).
These findings are in sharp contrast to estimating shared \emph{divergence}
times, which is much more accurate, precise, and robust to prior assumptions
\citep[\figs \labelcref{fig:valsimsmodeldiffuse,fig:valsimsetimesdiffuse,fig:mixsims}][]{Oaks2018ecoevolity,Oaks2018paic}.

Given the poor estimates of co-demographic changes, even when all the
information in the data are leveraged by a full-likelihood method, any
inference of shared demographic changes should be treated with caution.
However, there are potential ways that estimates of shared demographic
events could be improved.
For example, modelling loci of contiguous, linked sites could help
extract more information about past demographic changes.
Longer loci can contain much more information about the lengths of branches in
the gene tree, which are critically informative about the size of the
population through time.
This is evidenced by the extensive literature on powerful 
``skyline plot'' and ``phylodynamic'' methods
\citep{Pybus2000,Strimmer2001,OpgenRhein2005,Drummond2005,Heled2008,Minin2008beast,Ho2011,Palacios2012,Palacios2012UAI,Stadler2013,Gill2013,Palacios2014,Lan2015,Karcher2016,Karcher2017,Faulkner2018,Karcher2019}.
With loci from across the genome, each with more information about the gene
tree they evolved along,
perhaps more information can be captured about temporally clustered changes in
the rate of coalescence across populations.
Another potential source of information could be captured by modelling
recombination along large regions of chromosomes.
By approximating the full coalescent process, many methods have been developed
to model recombination in a computationally feasible manner
\citep{McVean2005,Marjoram2006,Chen2009,Li2011,Sheehan2013,Schiffels2014,Rasmussen2014,Palacios2015}.
This could potentially leverage additional information from genomic data about the
linkage patterns among sites along chromosomes.
Both of the these avenues are worth pursuing given the myriad historical
processes that predict patterns of temporally clustered changes in population
sizes across populations.
