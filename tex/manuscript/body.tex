\section{Introduction}

Many processes that shape the distribution of biodiversity across a landscape
are expected to affect multiple species.
Examples include changes to the environment itself that can cause
co-distributed species to contract or expand their ranges or become isolated.
Furthermore, ecological interactions can also cause demographic changes in
multiple species;
for example, the expansion of one species may predict the expansion of its
symbionts and the contraction of competitors.
Such processes predict that population divergences or demographic changes will
be temporally clustered across multiple species.
As a result, statistical methods that infer such patterns from genetic data
allow phylogeographers to test hypotheses about such processes.
There has been a lot of work on methods that test for shared divergence times
among pairs of populations
\citep{Hickerson2006,Hickerson2007,Huang2011,Oaks2014dpp,Oaks2018ecoevolity}.
Recently, there has also been a lot of work on methods that infer temporally
clustered changes in population sizes among species
\citep{Chan2014,Xue2015,Xue2017,Gehara2017}.

Uses of ABC estimation of shared size changes \citep{Burbrink2016}

To date, no method has allowed the co-estimation of both shared divergences
and population-size changes.
Given the overlap in the processes that can potentially cause divergence and
demographic changes across multiple species, such a method would be useful for
testing hypotheses about community-scale processes that shape biodiversity
across landscapes.
Here, we introduce a general, full-likelihood Bayesian method that can estimate
shared times among an arbitrary mix of population divergences and population
size changes (\fig{}~\ref{fig:modelCartoon}).

Furthermore, there has been a lot of work on how well the divergence-time
methods work
\citep{Oaks2012,Hickerson2013,Oaks2014reply,Oaks2014dpp,Overcast2017,Oaks2018ecoevolity}.
However, work on how well we can estimate shared population-size changes has
been much more limited.
There are theoretical reasons to suspect that estimating shared demographic
changes is more difficult than divergence times.
The parameter of interest can become unidentifiable in at least three ways.
First, as the magnitude of a change in population becomes smaller,
it becomes more difficult to estimate.
Second, as the age of the size change gets older, we run out of
genetic coalescent events that allow us to learn when it happened.
Third, as the age of the size change approaches zero, we once again will fail
to sample the coalescent events needed to inform us of the change.
Thus, our second goal is to leverage our new method to assess how well we can
infer shared population size changes among species.

\ifembed{
\embedWidthFigure{1.0}{../images/model-cartoons/mixed-model-description.pdf}{
    An illustration of the general comparative model implemented in
    \ecoevolity.
    The top two comparisons are pairs of populations for which we are
    interested in comparing their time of divergence (``divergence
    comparisons'').
    The bottom comparison is a single population for which we are interested in
    comparing the time of population-size change (``demographic comparison'').
    % In this scenario, three species of lizards co-occured on an island that was
    % fragmented by a rise in sea levels at time \etime[1].
    % Due to the fragmentation, the second lineage (from the top) diverged into
    % two descendant populations while the population size of the third lineage
    % was reduced.
    % The first lineage diverged later at \etime[2] due to another mechanism,
    % such as over-water dispersal.
    With three comparisons, there are five possible event models
    \citep[i.e., five ways to assign the comparisons to anywhere from one to three event times;][]{Bell1934},
    which are shown to the right with the
    example model indicated.
    The event time
    (\etime[1] and \etime[2])
    and effective population size
    (\epopsize[\rootpopindex],
    \epopsize[\descendantpopindex{}])
    parameters are shown.
    Event times can be shared among comparisons, but each ancestral and
    descendant population has a unique effective population size.
    % and the pair-specific event times (\comparisonetime[1],
    % \comparisonetime[2], and \comparisonetime[3])
    % are shown.
    % The notation used in the text for the biallelic character data and
    % effective sizes of the ancestral and descendant populations is shown for
    % the third comparison.
    % The lizard silhouette for the middle pair is from pixabay.com, and the
    % other two are from phylopic.org; all were licensed under the Creative
    % Commons (CC0) 1.0 Universal Public Domain Dedication.
    % Modified from \citet{Oaks2018ecoevolity}.
}{fig:modelCartoon}

}{}


\section{Methods}

\subsection{The data}
As described by \citet{Oaks2018ecoevolity},
we assume we have orthologous, biallelic genetic characters collected from
species we wish to compare.
By biallelic we mean that each character has at most two states, so that
constant characters are allowed.
For each species, we either have these data from one or more individuals
from a single population, in which case we infer the timing and extent
of a population size change, or one or more individuals from two populations,
in which case we infer the time when they diverged (\fig{}~\ref{fig:modelCartoon}).
In either case, we assume each character evolved along a gene tree (\genetree)
according to a finite-sites, continuous-time Markov chain (CTMC) model.
We assume the gene tree of each character is independent of the others,
conditional on the population history (i.e., the characters are effectively
unlinked).
As the marker evolves along the gene tree, forward in time, there is an
instantaneous relative rate \rgmurate of mutating from the ``red'' state to the
``green'' state, and a corresponding relative rate \grmurate of mutation from
green to red \citep{Bryant2012,Oaks2018ecoevolity}.

For each population and for each character we genotype \allelecount copies
of the locus, \redallelecount of which are copies of the red allele and the
remaining $\allelecount - \redallelecount$ are copies of the green allele.
Thus, for each population of a pair, and for each locus, we have a count of the
total sampled gene copies and how many of those are the red allele.

\ifembed{
\input{../../tables/notation.tex}
}{}

Following the notation of \citet{Oaks2018ecoevolity}
we will use \leafallelecounts and \leafredallelecounts to denote allele counts
for a locus from either one population if we are modeling population-size
change or both populations of a pair if we are modeling a divergence; i.e., 
$\leafallelecounts, \leafredallelecounts = (\allelecount, \redallelecount)$
or
$\leafallelecounts, \leafredallelecounts = (\allelecount[1],
\redallelecount[1]), (\allelecount[2], \redallelecount[2])$
(Fig.~\ref{fig:modelCartoon}
and
Table~\ref{table:notation}).
For convenience will use \comparisondata[i] to denote these allele counts
across all the loci from taxon $i$, which can be a single population
or a pair of populations.
Finally, we use \alldata to represent the data across all the taxa for which we
wish to compare times of either divergence or population size change.
Note, because the population of each compared taxon is modeled separately
(Fig.~\labelcref{fig:modelCartoon,fig:dag}),
characters do not have to be orthologous across taxa, only within them.


% Below is from codiv ecoevolity paper


\subsection{The model}

\subsubsection{The evolution of markers}

% We begin unpacking our model by first focusing on a single pair of populations.
We assume a finite-sites, continuous-time Markov chain (CTMC) model for the
evolution of the biallelic characters along a gene tree with branch lengths,
\genetree.
As the marker evolves along the gene tree, forward in time, there is an
instantaneous relative rate \rgmurate of mutating from the red state to the
green state, and a corresponding relative rate \grmurate of mutation from green
to red.
The stationary frequency of the red and green state is then
$\grmurate / (\rgmurate + \grmurate)$
and
$\rgmurate / (\rgmurate + \grmurate)$, respectively.
Thus, if given the stationary frequency of the green allele, \gfreq, we can
obtain the relative rates of mutation between the two states.
We will denote the overall rate of mutation as \murate.
If a mutation rate per site per unit time is given, then branch lengths are in
absolute time.
Alternatively, if $\murate = 1$, the branch lengths of the gene tree are in
units of expected substitutions per site.
In such a case, for a given pair of populations, the \murate is 
redundant, because it can be incorporated into the branch lengths of the gene
tree.
However, we introduce the notation here, because it will be useful later when
we want to allow rate variation among pairs of populations.
% For now, we will assume $\murate = 1$ so that the gene tree branch lengths, and
% time in general, is in units of the expected substitutions per site.

\subsubsection{The evolution of gene trees}

We assume that each marker sampled from a pair of populations evolved within a
simple ``species'' tree with one ancestral root population that diverged into
two descendant (terminal) branches at time \comparisondivtime
(Fig.~\ref{fig:modelCartoon}).
Again, if the $\murate$ is given, \comparisondivtime is in units of absolute
time; however, if $\murate$ is set to one, time is in units of expected
substitutions per site.
We will use
\comparisonpopsizes{}
to denote all three
effective sizes of a population pair
(\epopsize[\rootpopindex],
\epopsize[\descendantpopindex{1}],
and \epopsize[\descendantpopindex{2}]).
We will also use
\sptree{}
as shorthand for the species tree, which comprises the population sizes and
divergence time of a pair
(\comparisonpopsizes{} and \comparisondivtime{}).

\subsubsection{The likelihood}

\begin{linenomath}
% Given \murate, \gfreq, \comparisondivtime and \comparisonpopsizes,
Given \murate, \gfreq, and \sptree{},
the probability of the observed data at a locus (\leafallelecounts and
\leafredallelecounts), is the probability of the character pattern given the
gene tree multiplied by the probability of the gene tree given the species
tree, summed over all possible gene tree topologies and integrated over all
possible gene tree branch lengths,
\begin{equation}
    \pr(\leafallelecounts, \leafredallelecounts \given \sptree, \murate, \gfreq)
    =
    \int_{\genetree}
    \pr(\leafallelecounts, \leafredallelecounts \given \genetree, \murate, \gfreq)
    \pr(\genetree, \murate, \gfreq \given \sptree)
    \diff{\genetree}
    \label{eq:markerlikelihood}
\end{equation}
\citep{Felsenstein1988,Nielsen2001,Rannala2003}.
% We take advantage of the mathematical work of \citep{Bryant2012} to
% analytically integrate over all possible gene trees and character
% mutational histories along those gene trees.
We take advantage of the mathematical work of \citep{Bryant2012} to
analytically integrate over all possible gene trees and all possible character
substitution histories along those gene trees.
This allows us to compute the likelihood of the species tree directly from a
biallelic character pattern under a coalescent model,
i.e.,
$\pr(\leafallelecounts, \leafredallelecounts \given \sptree, \murate, \gfreq)$.
We refer readers to
\citet{Bryant2012}
for the details of this likelihood and the algorithms to compute it.
% \thought{For completeness, should I lay out the math and algorithm for
%     calculating the likelihood in the supplemental materials?}
\end{linenomath}

\begin{linenomath}
Assuming independence among loci (conditional on the species tree), we can
calculate the probability of \nloci{} loci given the species tree by
simply taking the product over them,
\begin{equation}
    \pr(\comparisondata \given \sptree, \murate, \gfreq)
    =
    \prod_{i=1}^{\nloci}
    \pr(\leafallelecounts[i], \leafredallelecounts[i] \given \sptree, \murate, \gfreq).
    \label{eq:comparisonlikelihood}
\end{equation}
Finally, the likelihood across all of our \ncomparisons{} pairs is simply the
product of the likelihood of each pair,
\begin{equation}
    \pr(
    \alldata
    \given
    \sptrees,
    \murates,
    \gfreqs)
    =
    \prod_{i=1}^{\ncomparisons}
    \pr(\comparisondata[i] \given \sptree[i], \murate[i], \gfreq[i]),
    \label{eq:collectionlikelihood}
\end{equation}
where
$\alldata = \comparisondata[1], \comparisondata[2], \ldots, \comparisondata[\ncomparisons]$,
$\sptrees = \sptree[1], \sptree[2], \ldots, \sptree[\ncomparisons]$,
$\murates = \murate[1], \murate[2], \ldots, \murate[\ncomparisons]$,
and
$\gfreqs = \gfreq[1], \gfreq[2], \ldots, \gfreq[\ncomparisons]$.
\end{linenomath}

\subsubsection{Correcting for excluded constant characters}

\begin{linenomath}
If we exclude constant characters and only analyze variable characters, we need
to correct the sample space for the excluded constant characters.
We can correct the likelihood by simply dividing by the probability of a
variable character, which is equal to one minus the probability of a constant
character,
\begin{equation}
\begin{split}
    \pr(\leafallelecounts, \leafredallelecounts \given \sptree, \murate, \gfreq, \textrm{variable})
    & =
    \frac{
        \pr(\leafallelecounts, \leafredallelecounts \given \sptree, \murate, \gfreq)
    }{
        \pr(\textrm{variable} \given \sptree, \murate, \gfreq)
    } \\
    & =
    \frac{
        \pr(\leafallelecounts, \leafredallelecounts \given \sptree, \murate, \gfreq)
    }{
        1 - \pr(\textrm{constant} \given \sptree, \murate, \gfreq)
    } \\
    & =
    \frac{
        \pr(\leafallelecounts, \leafredallelecounts \given \sptree, \murate, \gfreq)
    }{
        1 - \pr(\leafallelecounts \textrm{ all red} \given \sptree, \murate, \gfreq)
        - \pr(\leafallelecounts \textrm{ all green} \given \sptree, \murate, \gfreq)
    }.
    \label{eq:variablemarkerlikelihood}
\end{split}
\end{equation}
When we take the product over loci to get the probability of all the variable
data collected from a pair of populations, we correct each character pattern to
allow for different numbers of sampled gene copies among loci,
\begin{equation}
    \pr(\comparisondata \given \sptree, \murate, \gfreq, \textrm{variable})
    =
    \prod_{i=1}^{\nloci}
    \frac{
        \pr(\leafallelecounts[i], \leafredallelecounts[i] \given \sptree, \murate, \gfreq)
    }{
        1 - \pr(\leafallelecounts[i] \textrm{ all red} \given \sptree, \murate, \gfreq)
        - \pr(\leafallelecounts[i] \textrm{ all green} \given \sptree, \murate, \gfreq)
    }.
    \label{eq:variablecomparisonlikelihood}
\end{equation}
This is a bit different than the correction done in the software SNAPP
\citep{Bryant2012}.
If we use \maxleafallelecounts to denote the maximum number of gene copies
sampled from each population, then the correction in SNAPP is
\begin{equation}
    \pr_{\tiny SNAPP}(\comparisondata \given \sptree, \murate, \gfreq, \textrm{variable})
    =
    \frac{
        \prod_{i=1}^{\nloci}
        \pr(\leafallelecounts[i], \leafredallelecounts[i] \given \sptree, \murate, \gfreq).
    }{
        (1 - \pr(\maxleafallelecounts \textrm{ all red} \given \sptree, \murate, \gfreq)
        - \pr(\maxleafallelecounts \textrm{ all green} \given \sptree, \murate, \gfreq))^{\nloci}
    }.
    \label{eq:snappvariablecomparisonlikelihood}
\end{equation}
These are equivalent if the same number of samples are collected across all
variable loci for each population (i.e., no missing gene copies), but will
deviate if fewer copies are sampled for at least one locus.
Thus, identical likelihoods between SNAPP and our method should not be expected
when analyzing variable-only data.
\end{linenomath}

\subsection{The data}

\subsection{The model}

\subsubsection{The evolution of markers}

\subsubsection{The evolution of gene trees}

\subsubsection{The likelihood}

\subsubsection{Correcting for excluded constant characters}

\subsection{Bayesian inference}

\subsubsection{Priors}

\paragraph{Prior on the concentration parameter}

\paragraph{Prior on the event times}

\paragraph{Prior on mutation rates}

\paragraph{Prior on the equilibrium-state frequency}

\subsection{Software implementation}

\subsection{Analyses of simulated data}

\subsubsection{Validation analyses}

\subsection{Empirical application}
Stickleback data?


\section{Results}

\subsection{Analyses of simulated data}

\subsubsection{Validation analyses}

\subsection{Empirical application}


\section{Discussion}
