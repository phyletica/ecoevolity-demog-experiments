% \section{The model}

We extended the model implemented in the software package \ecoevolity to
accommodate two types of temporal comparisons:
\begin{enumerate}
    \item A population that experienced a change from effective population size
        \epopsize[\rootpopindex]
        to effective size
        \epopsize[\descendantpopindex{}]
        at time \comparisonetime in the past.
        We will refer to this as a \emph{demographic comparison}
        (\fig{}~\ref{fig:modelCartoon}),
        and refer to the population before and after the change in population
        size as ``ancestral'' and ``descendant'', respectively.
    \item A population that diverged at time \comparisonetime in the past into
        two descendant populations, each with unique effective population
        sizes.
        We will refer to this as a \emph{divergence comparison}
        (\fig{}~\ref{fig:modelCartoon}).
\end{enumerate}
The type of each comparison is specified by the investigator; either the input
data specifies one (demographic comparison) or two (divergence comparison)
populations.
This allows inference of shared times of divergence and/or demographic change
across an arbitrary mix of demographic and divergence comparisons in a
full-likelihood, Bayesian framework.
Table~\ref{table:notation} provides a key to the notation we use throughout
this paper.

\ifembed{
\begin{table}[htbp]
    \sffamily
    \small
    \rowcolors{2}{}{mygray}
    \addtolength{\tabcolsep}{-0.1cm}
\caption{
    A key to some of the notation used in the text.
}
    \centering
    \begin{tabular}{ l >{\raggedright\hangindent=0.5cm}m{14cm} }
        \toprule
        \textbf{Symbol} & \textbf{Description} \tn
        \midrule
        \ncomparisons{} & The number of comparisons (or taxa); can be an
        arbitrary mix of populations (comparing timing of demographic change)
        and/or pairs of populations (comparing timing of divergence).
        \tn
        \nevents{} & The number of events (unique times) across the comparisons.
        \tn
        \comparisonetime[i] & The time in the past when comparison $i$ either
        diverged or experienced a change in effective population size.
        \tn
        \etime & An event time at which one or more comparisons
        experienced a divergence or change in effective population size.
        % \tn
        % \etimemodel & The event-time model, which comprises the event
        % times and the mapping of comparisons to those times.
        \tn
        \etimesets & The event-time model, which comprises the assignment of
        comparisons to events.
        \tn
        \etimes & All of the times of the events in the model
            ($\etimes = \etime[1], \ldots, \etime[\nevents]$).
        \tn
        % \basedistribution & The base distribution of the Dirichlet process.
        % \tn
        \concentration & The concentration parameter of the Dirichlet process.
        \tn
        \allelecount, \redallelecount & The number of copies of a locus sampled
            from a population, and the number of those copies that are the ``red''
            allele.
            \tn
        \leafallelecounts, \leafredallelecounts & The allele counts from 
        a comparison (one or two populations).
            \tn
        \comparisondata[i] & The allele counts across all characters from
            comparison $i$. I.e., all of the characters being analyzed for
            comparison $i$.
            \tn
        \nloci & The number of characters collected from a taxon (comparison).
        \tn
        \alldata & All of the data being analyzed, i.e., the
            character matrices from all comparisons.
        \tn
        \genetree & A gene tree with branch lengths.
        \tn
        \murate & The rate of mutation.
        \tn
        \rgmurate & Relative rate of mutating from the ``red'' to ``green'' state.
        \tn
        \grmurate & Relative rate of mutating from the ``green'' to ``red'' state.
        \tn
        \gfreq & The stationary frequency of the ``green'' state.
        \tn
        \epopsize[\descendantpopindex{}] &
            The effective size of a descendant population.
        \tn
        \epopsize[\rootpopindex] & The effective size of the root (ancestral) population.
        \tn
        \rootrelativepopsize & The relative effective population size of the root (ancestral)
        population; relative to the mean of the effective sizes of the descendant populations.
        \tn
        \comparisonpopsizes & Shorthand notation for all effective population
        sizes for a comparison (ancestral and one or two descendant
        populations).
        \tn
        \sptree & The species tree for a comparison. This comprises
        the effective population sizes and the time of demographic change or
        divergence.
            \tn
        \bottomrule
    \end{tabular}
    \label{table:notation}
\end{table}

}{}

\subsection{The data}
As described by \citet{Oaks2018ecoevolity},
we assume we have collected orthologous, biallelic genetic characters from taxa
we wish to compare.
By biallelic, we mean that each character has at most two states,
which we refer to as ``red'' and ``green'' following \citet{Bryant2012}.
For each taxon, we either have these data from one or more individuals from a
single population, in which case we infer the timing and extent of a population
size change, or one or more individuals from two populations or species,
in which case we infer the time when they diverged (\fig{}~\ref{fig:modelCartoon}).

For each population and for each character we genotype \allelecount copies
of the locus, \redallelecount of which are copies of the red allele and the
remaining $\allelecount - \redallelecount$ are copies of the green allele.
Thus, for each population, and for each locus, we have a count of the
total sampled gene copies and how many of those are the red allele.

Following the notation of \citet{Oaks2018ecoevolity}
we will use \leafallelecounts and \leafredallelecounts to denote allele counts
for a locus from either one population if we are modeling a population-size
change or both populations of a pair if we are modeling a divergence; i.e., 
$\leafallelecounts, \leafredallelecounts = (\allelecount, \redallelecount)$
or
$\leafallelecounts, \leafredallelecounts = (\allelecount[1],
\redallelecount[1]), (\allelecount[2], \redallelecount[2])$.
% (\fig{}~\ref{fig:modelCartoon}
% and
% Table~\ref{table:notation}).
For convenience will use \comparisondata[i] to denote these allele counts
across all the loci from taxon $i$, which can be a single population
or a pair of populations.
Finally, we use \alldata to represent the data across all the taxa for which we
wish to compare times of either divergence or population-size change.
Note, because the population history of each taxon is modeled separately
(\fig{}~\ref{fig:modelCartoon}),
characters do not have to be orthologous across taxa, only within them.


\subsection{The evolution of markers}

We assume each character evolved along a gene tree (\genetree)
according to a finite-sites, continuous-time Markov chain (CTMC) model.
We assume the gene tree of each character is independent of the others,
conditional on the population history (i.e., the characters are effectively
unlinked).
As the marker evolves along the gene tree, forward in time, there is a relative
rate \rgmurate of mutating from the red state to the green state,
and a corresponding relative rate \grmurate of mutation from green to red
\citep{Bryant2012,Oaks2018ecoevolity}.
The stationary frequency of the green state is then
$\gfreq = \rgmurate / (\rgmurate + \grmurate)$.
We will use \murate to denote the overall rate of mutation.
Evolutionary change is the product of \murate and time.
Thus, if $\murate = 1$, time is measured in
units of expected substitutions per site.
Alternatively, if a mutation rate per site per unit time is given, then time is
in those units (e.g., generations or years).

\subsection{The evolution of gene trees}

We assume that the gene tree of each locus evolved within a simple
``species'' tree with one ancestral root population, which either
left one or two descendant branches with different effective population sizes
at time \comparisonetime
(\fig{}~\ref{fig:modelCartoon}).
We will use
\comparisonpopsizes{}
to denote all the effective population sizes of a species tree;
\epopsize[\rootpopindex] and 
\epopsize[\descendantpopindex{}] when modeling a population-size change, and
\epopsize[\rootpopindex],
\epopsize[\descendantpopindex{1}],
and \epopsize[\descendantpopindex{2}] when modeling a divergence.
Following \citet{Oaks2018ecoevolity}, we use
\sptree{}
as shorthand for the species tree, which comprises the population sizes and
event time of a comparison
(\comparisonpopsizes{} and \comparisonetime{}).


\subsection{The likelihood}

As in \citet{Oaks2018ecoevolity},
we use the work of \citet{Bryant2012}
to analytically integrate over all possible gene trees and
character substitution histories to compute the likelihood
of the species tree directly from 
a biallelic character pattern under a coalescent model;
$\pr(\leafallelecounts, \leafredallelecounts \given \sptree, \murate, \gfreq)$.
We only need to make a small modification to accommodate population-size-change
models that have a species tree with only one descendant population.
Equation 19 of \citet{Bryant2012} shows how to obtain the partial likelihoods
at the bottom of an ancestral branch from the partial likelihoods at the top of
its two descendant branches.
When there is only one descendant branch, this is simplified, and the partial
likelihoods at the bottom of the ancestral branch are equal to the partial
likelihoods at the top of its sole descendant branch.
Other than this small change, the probability of a biallelic character pattern
given the species tree, mutation rate, and equilibrium state frequencies
($\pr(\leafallelecounts, \leafredallelecounts \given \sptree, \murate, \gfreq)$)
is calculated the same as in \citet{Bryant2012} and \citet{Oaks2018ecoevolity}.


\begin{linenomath}
For a given taxon, we can calculate the probability of all \nloci{} loci
from which we have data given the species tree and other parameters by
assuming independence among loci (conditional on the species tree) and
taking the product over them,
\begin{equation}
    \pr(\comparisondata \given \sptree, \murate, \gfreq)
    =
    \prod_{i=1}^{\nloci}
    \pr(\leafallelecounts[i], \leafredallelecounts[i] \given \sptree, \murate, \gfreq).
    \label{eq:comparisonlikelihood}
\end{equation}
We assume we have sampled biallelic data from \ncomparisons{} taxa, which can
be an arbitrary mix of
(1) two populations or species for which \comparisonetime represents
the time they diverged, or
(2) one population for which \comparisonetime represents the time
of a change in population size.
The likelihood across all \ncomparisons{} taxa is simply the product of the
likelihood of each taxon,
\begin{equation}
    \pr(
    \alldata
    \given
    \sptrees,
    \murates,
    \gfreqs)
    =
    \prod_{i=1}^{\ncomparisons}
    \pr(\comparisondata[i] \given \sptree[i], \murate[i], \gfreq[i]),
    \label{eq:collectionlikelihood}
\end{equation}
where
$\alldata = \comparisondata[1], \comparisondata[2], \ldots, \comparisondata[\ncomparisons]$,
$\sptrees = \sptree[1], \sptree[2], \ldots, \sptree[\ncomparisons]$,
$\murates = \murate[1], \murate[2], \ldots, \murate[\ncomparisons]$,
and
$\gfreqs = \gfreq[1], \gfreq[2], \ldots, \gfreq[\ncomparisons]$.
As described in \citet{Oaks2018ecoevolity},
if constant characters are not sampled for a taxon, we condition the likelihood
for that taxon on only having sampled variable characters.
\end{linenomath}


\subsection{Bayesian inference}

\begin{linenomath}
As described by \citet{Oaks2018ecoevolity},
we treat the number of events (population-size changes and/or divergences)
and the assignment of taxa to those events as
random variables under a Dirichlet process \citep{Ferguson1973,
    Antoniak1974}.
We use \etimesets to represent the partitioning of taxa to events,
which we will also refer to as the ``event model.''
The concentration parameter, \concentration, controls how clustered the
Dirichlet process is, and determines the probability of all possible \etimesets
(i.e., all possible set partitions of taxa to $1, 2, \ldots, \ncomparisons$ events).
We use \etimes to represent the unique times of events in \etimesets.
Using this notation, the posterior distribution of our 
Dirichlet-process model is
\begin{equation}
\begin{split}
    & \pr(
    \concentration,
    \etimes,
    \etimesets,
    \collectionpopsizes,
    \murates,
    \gfreqs
    \given
    \alldata
    % \basedistribution
    )
    = \\
    & \frac{
        \pr(
        \alldata
        \given
        \etimes,
        \etimesets,
        \collectionpopsizes,
        \murates,
        \gfreqs
        )
        \pr(\etimes \given \etimesets)%, \basedistribution)
        \pr(\etimesets \given \concentration)
        \pr(\concentration)
        \pr(\collectionpopsizes)
        \pr(\murates)
        \pr(\gfreqs)
    }{
        \pr(
        \alldata%,
        % \basedistribution
        )
    },
    \label{eq:bayesruleexpanded}
\end{split}
\end{equation}
where
\collectionpopsizes
is the collection of the effective population sizes (\comparisonpopsizes{})
across all of the pairs.
\end{linenomath}

\subsubsection{Priors}

\paragraph{Prior on the concentration parameter}
Our implementation allows for a hierarchical approach to accommodate
uncertainty in the concentration parameter of the Dirichlet process
by specifying a gamma distribution as a hyperprior on
\concentration \citep{Escobar1995,Heath2011}.
Alternatively, \concentration can also be fixed to a particular value,
which is likely sufficient when the number of pairs is small.

\paragraph{Prior on the divergence times}
Given the partitioning of taxa to events, we use a gamma
distribution for the prior on the time of each event,
$\etime \given \etimesets \sim \distgamma(\cdot, \cdot)$.

\paragraph{Prior on the effective population sizes}
We use a gamma distribution as the prior on
the effective size of each descendant population of each taxon.
Following \citet{Oaks2018ecoevolity},
we use a gamma distribution on the effective size of the ancestral population
\emph{relative} to the size of the descendant population(s), which we
denote as \rootrelativepopsize.
For a taxon with two descendant populations (i.e., a divergence comparison), the
ancestral population size is relative to the mean of the descendant
populations.
For a taxon with only one descendant population (i.e., a demographic
comparison), the ancestral populations is relative to the size of that
descendant.
% The goal of this approach is to allow more informative priors on the root
% population size; we often have stronger prior expectations for the relative
% size of the ancestral population than the absolute size.
% This is important, because the effective size of the ancestral population is a
% difficult nuisance parameter to estimate and can be strongly correlated with
% the divergence time.
% For example, if the divergence time is so old such that all the gene copies
% of a locus coalesce within the descendant populations, the locus
% provides very little information about the size of the ancestral
% population.
% As a result, a larger ancestral population and more recent divergence will have
% a very similar likelihood to a small ancestral population and an older
% divergence.
% Thus, placing more prior density on reasonable values of the ancestral
% population size can help improve the precision of divergence-time estimates.

\paragraph{Prior on mutation rates}
We follow the same approach explained by \citet{Oaks2018ecoevolity} to model
mutation rates across taxa.
The decision about how to model mutation rates is extremely important for any
comparative phylogeographic approach that models taxa as disconnected
species trees
\citep[\fig{}~\ref{fig:modelCartoon}; e.g.,][]{Hickerson2006,Hickerson2007,Huang2011,Chan2014,Oaks2014dpp,Xue2015,Burbrink2016,Xue2017,Gehara2017,Oaks2018ecoevolity}.
Time (\etime) and mutation rate (\murate) are inextricably linked, and because
the comparisons are modeled as separate species trees, the data cannot
inform the model about relative or absolute differences in \murate among the
comparisons.
We provide flexibility to the investigator to fix or place prior probability
distributions on the relative or absolute rate of mutation for each comparison.
However, if one chooses to accommodate uncertainty in the mutation rate of one
or more comparisons, the priors should be strongly informative.
Because of the inextricable link between rate and time,
placing a weakly informative prior on a comparison's mutation rate prevents
estimation of the time of its demographic change or divergence,
which is the primary goal.

\paragraph{Prior on the equilibrium state frequency}
Recoding four-state nucleotides to two states requires some arbitrary
decisions, and whenever $\gfreq \neq 0.5$, these decisions can affect
the likelihood of the model \citep{Oaks2018ecoevolity}.
Because DNA is the dominant character type for genomic data, we assume that
$\gfreq = 0.5$ in this paper.
This makes the CTMC model of character-state substitution a two-state analog of
the ``JC69'' model \citep{JC1969}.
However, if the genetic markers collected for one or more taxa are naturally
biallelic, the frequencies of the two states can be meaningfully estimated, and
our implementation allows for a beta prior on \gfreq in such cases.
This makes the CTMC model of character-state substitution a two-state general
time-reversible model \citep{Tavare1986}.

\subsubsection{Approximating the posterior with MCMC}

We use Markov chain Monte Carlo (MCMC) algorithms to sample from the joint
posterior in Equation~\ref{eq:bayesruleexpanded}.
To sample across event models (\etimesets) during the MCMC chain, we use the
Gibbs sampling algorithm (Algorithm 8) of \citet{Neal2000}.
We also use univariate and multivariate Metropolis-Hastings algorithms
\citep{Metropolis1953,Hastings1970} to update the model,
the latter of which are detailed in \citet{Oaks2018ecoevolity}.

\subsection{Software implementation}
The \cpp source code for \ecoevolity is freely available from
\url{https://github.com/phyletica/ecoevolity} and includes an extensive test
suite.
From the \cpp source code, two primary command-line tools are compiled:
(1) \ecoevolity, for performing Bayesian inference under the model described
above,
and
(2) \simcoevolity for simulating data under the model described above.
Documentation for how to install and use the software is available at
\url{http://phyletica.org/ecoevolity/}.
We have incorporated help in pre-processing data and post-processing posterior
samples collected by \ecoevolity in the Python package \pycoevolity, which is
available at
\url{https://github.com/phyletica/pycoevolity}.
We used Version 0.3.1
(Commit 9284417)
of the \ecoevolity software package for all of our analyses.
A detailed history of this project, including all of the data and scripts
needed to produce our results, is available at
\url{https://github.com/phyletica/ecoevolity-demog-experiments}
\citep{Oaks2019CodemogZenodo}.

\subsection{The model}

We extended the software package, \ecoevolity, to accommodate two types of
temporal comparisons:
\begin{enumerate}
    \item A population that experienced a change from effective population size
        \epopsize[\rootpopindex]
        to effective size
        \epopsize[\descendantpopindex{}]
        at time \comparisonetime in the past.
        We will refer to this as a \emph{demographic comparison}
        (\fig{}~\ref{fig:modelCartoon}),
        and refer to the population before and after the change in population
        size as ``ancestral'' and ``descendant'', respectively.
    \item A population that diverged at time \comparisonetime in the past into
        two descendant populations, each with unique effective population
        sizes.
        We will refer to this as a \emph{divergence comparison}
        (\fig{}~\ref{fig:modelCartoon}).
\end{enumerate}
This allowed us to infer shared times of divergence and/or demographic change
across an arbitrary mix of demographic and divergence comparisons in a
full-likelihood, Bayesian framework.
During an ``event'' at time \etime, one or more demographic changes and/or
divergences can occur.
We estimate the number and timing of events and the assignment of comparisons
to those events under a Dirichlet-process \citep{Ferguson1973,Antoniak1974}
prior model.
The \emph{a priori} tendency for comparisons to share events is controlled by
the concentration parameter (\concentration) of the Dirichlet process.
We use Markov chain Monte Carlo (MCMC) algorithms
\citep{Metropolis1953,Hastings1970,Neal2000}
to sample from the joint posterior of the model.
See Appendix~\ref{appendix:model} for a full description of the model, and
Table~\ref{table:notation} for a key to the notation we use throughout this
paper.

\subsection{Software implementation}
The \cpp source code for \ecoevolity is freely available from
\url{https://github.com/phyletica/ecoevolity} and includes an extensive test
suite.
Documentation for how to install and use the software is available at
\url{http://phyletica.org/ecoevolity/}.
We have incorporated help in pre-processing data and post-processing posterior
samples collected by \ecoevolity in the Python package \pycoevolity, which is
available at
\url{https://github.com/phyletica/pycoevolity}.
We used Version 0.3.1
(Commit 9284417)
of the \ecoevolity software package for all of our analyses.
A detailed history of this project, including all of the data and scripts
needed to produce our results, is available at
\url{https://github.com/phyletica/ecoevolity-demog-experiments}.


\subsection{Analyses of simulated data}

\subsubsection{Assessing ability to estimate timing and sharing of demographic changes}

We used the \simcoevolity and \ecoevolity tools within the \ecoevolity software
package to simulate and analyze \datasets,
respectively, under a variety of conditions.
Each simulated \dataset comprised 500,000 characters collected from 10 diploid
individuals (20 genomes) sampled per population from three demographic
comparisons.
We specified the concentration parameter of the Dirichlet process so that
the mean number of events was 2 ($\concentration = 1.414216$).
We assumed the mutation rates of all three populations were equal and 1, such
that time and effective population sizes were scaled by the mutation rate.
When analyzing each simulated \dataset, we ran four MCMC chains run for 75,000
generations with a sample taken every 50 generations; we combined and
summarized the last 1000 samples of each of the four chains (the first 501
samples discarded from each chain).

\paragraph{Initial simulation conditions}

We initially simulated data under distributions we hoped comprised a mix of
conditions that were favorable and challenging for estimating the timing and
sharing of demographic changes.
For these initial conditions, we simulated \datasets with three populations
that underwent a demographic change, under five different distributions on the
relative effective size of the ancestral population
(\rootrelativepopsize; see left column of
\figs
\labelcref{fig:valsimsmodelinitial,fig:valsimsetimesinitial}):
\begin{enumerate}[label=A.\arabic*]
    \item \dgamma{10}{0.25} (4-fold population increase) \label{sims:initialFourFoldIncrease}
    \item \dgamma{10}{0.5} (2-fold population increase)  \label{sims:initialTwoFoldIncrease}
    \item \dgamma{10}{2} (2-fold population decrease)    \label{sims:initialTwoFoldDecrease}
    \item \dgamma{10}{1} (no change on average, but a fair amount of variance) \label{sims:initialCenter}
    \item \dgamma{100}{1} (no change on average, little variance) \label{sims:initialCenterNarrow}
\end{enumerate}
The last distribution was chosen to represent a ``worst-case'' scenario where
there was almost no demographic change in the history of the populations.
For the mutation-scaled effective size of the descendant populations
($\epopsize[\descendantpopindex{}]\murate$; i.e., the population size after the
demographic change),
we used a gamma distribution with a shape of 5 and mean of 0.002.
The timing of the demographic events was exponentially distributed with a mean
of 0.01 expected substitutions per site;
given the mean of our distribution on the effective size of the descendant
populations, this puts the expectation of demographic-change times in units of
$4N_e$ generations at 1.25.
This distribution on times was chosen to span times of demographic change from
very recent (i.e., most gene lineages coalesce before the change) to old (i.e.,
most gene lineages coalesce after the change).
The assignment of the three simulated populations to 1, 2, or 3 demographic
events was controlled by a Dirichlet process with a mean number of 2.0
demographic events across the three populations.
We generated 500 \datasets under each of these five simulation conditions, all
of which were analyzed using the same simulated distributions as priors.

\paragraph{Simulation conditions chosen to improve performance}

Estimates of the timing and sharing of demographic events were quite poor
across all the initial simulation conditions (see results).
In an effort to find conditions under which the timing and sharing of
demographic changes could be better estimated, and avoid combinations
of parameter values that caused identifiability problems,
we next explored simulations under distributions on times and population sizes
offset from zero, and with much more recent demographic event times.
% Event time ~ gamma(shape=4.0, scale=0.000475, offset=0.0001); mean 0.002
% relative root size ~ gamma(shape=5.0, scale = 0.04, offset = 0.05); mean 0.25
% relative root size ~ gamma(shape=5.0, scale = 0.09, offset = 0.05); mean 0.5
% relative root size ~ gamma(shape=5.0, scale = 0.79, offset = 0.05); mean 4
% relative root size ~ gamma(shape=5.0, scale = 0.19, offset = 0.05); mean 1
% relative root size ~ gamma(shape=50.0, scale = 0.02, offset = 0.0); mean 1
For the mutation-scaled effective size of the descendant
population
($\epopsize[\descendantpopindex{}]\murate$),
we used an offset gamma distribution with a shape of 4, offset of 0.0001, and
mean of 0.0021 (accounting for the offset).
For the distribution of event times, we used a gamma distribution
with a shape of 4, offset of 0.0001, and a mean of 0.002 (accounting
for the offset; 0.25 units of $4N_e$ generations, on average).
Again, we used five different distributions on the relative effective size of
the ancestral population (see left column of
\figs
\labelcref{fig:valsimsmodelopt,fig:valsimsetimesopt}):
\begin{enumerate}[label=B.\arabic*]
    \item \dogamma{5}{0.25}{0.05} (4-fold population increase) \label{sims:optimalFourFoldIncrease}
    \item \dogamma{5}{0.5}{0.05} (2-fold population increase)  \label{sims:optimalTwoFoldIncrease}
    \item \dogamma{5}{4}{0.05} (4-fold population decrease)    \label{sims:optimalFourFoldDecrease}
    \item \dogamma{5}{1}{0.05} (no change on average, but a fair amount of variation) \label{sims:optimalCenter}
    \item \dogamma{50}{1}{0} (no change on average, little variance) \label{sims:optimalCenterNarrow}
\end{enumerate}
We generated 500 \datasets under each of these five distributions, and analyzed
all of them using priors that matched the generating distributions.

\subsubsection{Simulations to assess sensitivity to prior assumptions}

% Sim disributions:
% Event time ~ gamma(shape=4.0, scale=0.000475, offset=0.0001); mean 0.002
% relative root size ~ gamma(shape=5.0, scale = 0.04, offset = 0.05); mean 0.25
% descendant size ~ gamma(4.0, scale=0.0005, offset = 0.0001); mean 0.0021
% Priors:
% Event time ~ exponential(mean = 0.005)
% relative root size ~ Exponential(mean = 2.0)
% descendant size ~ gamma(2.0, scale 0.001); mean 0.002

Next, we simulated an additional 500 \datasets under
Condition~\ref{sims:optimalFourFoldIncrease} above.
We then analyzed each of these \datasets under ``diffuse'' prior
distributions:
\begin{itemize}
    \item $\etime \sim \dexponential{0.005}$
    \item $\rootrelativepopsize \sim \dexponential{2}$
    \item $\epopsize[\descendantpopindex{}] \sim \dgamma{2}{0.002}$
\end{itemize}
These distributions were chosen to reflect realistic amounts of prior
uncertainty about the timing of demographic changes and past and present
effective population sizes when analyzing empirical data.
For comparison, we also performed the same simulations and analyses under
diffuse priors for three divergence comparisons.
For these divergence comparisons, we simulated 10 sampled genomes per
population to match the same total number of samples per comparison (20) as the
demographic simulations.


\subsubsection{Simulating a mix of divergence and demographic comparisons}

To explore how well our method can infer a mix of shared demographic changes
and divergence times, we simulated 500 \datasets comprised of 6 comparisons:
3 demographic comparisons and
3 divergence comparisons.
To ensure the same amount of data across comparisons, we simulated
20 sampled genomes (10 diploid individuals) from each comparison
(i.e., 10 genomes from both populations of each divergence comparison).
We used the same simulation conditions described above for
\ref{sims:optimalTwoFoldIncrease},
and specified these same distributions as priors when analyzing all of the
simulated \datasets.


\subsubsection{Simulating linked sites}
To assess the effect of linked sites on the inference
of the timing and sharing of demographic changes,
we simulated \datasets comprising 5000 100-base-pair
loci (500,000 total characters).
The distributions on parameters were the same
as the conditions described for \ref{sims:optimalFourFoldIncrease} above.
These same distributions were used as priors when
analyzing the simulated \datasets.

% \subsection{Data-acquisition bias?}


\subsection{Empirical application to stickleback data}


\subsubsection{Assembly of loci}
We assembled the publicly available RADseq data collected by
\citet{Hohenlohe2010}
from five populations of threespine sticklebacks (\spp{Gasterosteus aculeatus})
from south-central Alaska.
After downloading the reads mapped to the stickleback genome by
\citet{Hohenlohe2010}
from Dryad
(doi:10.5061/dryad.b6vh6),
We assembled reference guided alignments of loci in Stacks v1.48
\citet{Catchen2013} with a minimum read depth of 3 identical reads per locus
within each individual and the bounded single-nucleotide polymorphism (SNP)
model with error bounds between
0.001 and 0.01.
To maximize the number of loci and minimize paralogy, we assembled each
population separately;
because \ecoevolity models each population separately
(\fig{}~\ref{fig:modelCartoon}),
the characters do not need to be orthologous across populations, only within
them.

\subsubsection{Inferring shared demographic changes with \ecoevolity}

We used a value for the concentration parameter of the Dirichlet process
that corresponds to a mean number of three events
($\concentration = 2.22543$).
We used the following prior distributions on the timing of events and effective
sizes of populations:
\begin{itemize}
    \item $\etime \sim \dexponential{0.001}$
    \item $\rootrelativepopsize \sim \dexponential{1}$
    \item $\epopsize[\descendantpopindex{}] \sim \dgamma{2}{0.002}$
\end{itemize}
To assess the sensitivity of the results to these prior assumptions,
we also analyzed the data under two additional priors on
the concentration parameter, event times, and relative
effective population size of the ancestral population:
\begin{itemize}
    \item $\concentration = 13$ (half of prior probability on 5 events)
    \item $\concentration = 0.3725$ (half of prior probability on 1 event)
    \item $\etime \sim \dexponential{0.0005}$
    \item $\etime \sim \dexponential{0.01}$
    \item $\rootrelativepopsize \sim \dexponential{0.5}$
    \item $\rootrelativepopsize \sim \dexponential{0.1}$
\end{itemize}

For each prior setting, we ran 10 MCMC chains for 150,000 generations, sampling
every 100 generations; we did this using all the sites in the assembled
stickleback loci and only SNPs.
To assess convergence and mixing of the chains, we calculated the potential
scale reduction factor \citep[PSRF; the square root of Equation 1.1 in][]{Brooks1998}
and effective sample size \citep{Gong2014} of all continuous parameters and the
log likelihood using the \texttt{pyco-sumchains} tool of \pycoevolity
(Version 0.1.2 Commit 89d90a1).
We also visually inspected the sampled log likelihood and parameters values
over generations with the program Tracer Version 1.6 \citep{Tracer16}.
The MCMC chains for all analyses converged almost immediatley; we
conservatively removed the first 101 samples from each chain, resulting in
14,000 samples from the posterior (1400 samples from 10 chains) for each
analysis.
