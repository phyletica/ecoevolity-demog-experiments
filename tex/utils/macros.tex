\newcommand{\jedit}[2]{\sout{#1}{\color{red}{#2}}}
% \newcommand{\jedit}[2]{#2}
\newcommand{\jcomment}[1]{({\color{pgreen}{JRO's comment:}} \textbf{\color{pgreen}{#1}})}

\newcommand{\citationNeeded}{\textcolor{magenta}{\textbf{[CITATION NEEDED]}}\xspace}
\newcommand{\tableNeeded}{\textcolor{magenta}{\textbf{[TABLE NEEDED]}}\xspace}
\newcommand{\figureNeeded}{\textcolor{magenta}{\textbf{[FIGURE NEEDED]}}\xspace}
\newcommand{\highLight}[1]{\textcolor{magenta}{\MakeUppercase{#1}}}

\newcommand{\fig}{Figure\xspace}
\newcommand{\figs}{Figures\xspace}
\newcommand{\tbl}{Table\xspace}
\newcommand{\tbls}{Tables\xspace}

\newcommand{\datasets}{data sets\xspace}
\newcommand{\dataset}{data set\xspace}

\newcommand{\editorialNote}[1]{\textcolor{red}{[\textit{#1}]}}
\newcommand{\ignore}[1]{}
\newcommand{\addTail}[1]{\textit{#1}.---}
\newcommand{\super}[1]{\ensuremath{^{\textrm{#1}}}}
\newcommand{\sub}[1]{\ensuremath{_{\textrm{#1}}}}
\newcommand{\dC}{\ensuremath{^\circ{\textrm{C}}}}
\newcommand{\tb}{\hspace{2em}}
\newcommand{\tn}{\tabularnewline}
\newcommand{\spp}[1]{\textit{#1}}

\providecommand{\e}[1]{\ensuremath{\times 10^{#1}}}

\newcommand{\change}[2]{{\color{red} #2}\xspace}
\newcommand{\thought}[1]{\textcolor{purple}{THOUGHT: #1}}

\newcommand{\widthFigure}[5]{\begin{figure}[htbp]
\begin{center}
    \includegraphics[width=#1\textwidth]{#2}
    \captionsetup{#3}
    \caption{#4}
    \label{#5}
    \end{center}
    \end{figure}}

\newcommand{\heightFigure}[5]{\begin{figure}[htbp]
\begin{center}
    \includegraphics[height=#1\textheight]{#2}
    \captionsetup{#3}
    \caption{#4}
    \label{#5}
    \end{center}
    \end{figure}}

\newcommand{\smartFigure}[5]{%
    \begin{figure}[htbp]
        \begin{center}
            \includegraphics[width=\textwidth,height=#1\textheight,keepaspectratio]{#2}
            \captionsetup{#3}
            \caption{#4}
            \label{#5}
        \end{center}
    \end{figure}
}

\newcommand{\mFigure}[4]{\smartFigure{#1}{#2}{listformat=figList}{#3}{#4}\clearpage}
\newcommand{\embedHeightFigure}[4]{\heightFigure{#1}{#2}{listformat=figList}{#3}{#4}}
\newcommand{\embedWidthFigure}[4]{\widthFigure{#1}{#2}{listformat=figList}{#3}{#4}}
\newcommand{\siFigure}[4]{\smartFigure{#1}{#2}{name=Figure S, labelformat=noSpace, listformat=sFigList}{#3}{#4}\clearpage}

\newcommand{\validationsimsthreecolumndescriptiontemplate}[1]{The left column
    of plots shows the gamma distribution from which the relative size of the
    ancestral population was drawn; this was also used as the prior when each
    simulated \dataset was analyzed.
    The center and right column of plots show true versus estimated #1 when
    using all characters (center) or only variable characters (right).
}
\newcommand{\validationsimsthreecolumndescription}{\validationsimsthreecolumndescriptiontemplate{values}}
\newcommand{\validationsimsthreecolumndescriptionmodels}{\validationsimsthreecolumndescriptiontemplate{models}}
\newcommand{\diffusesimsfourcolumndescriptiontemplate}[1]{The first and second columns of
    plots show the distribution on the relative effective size of the ancestral
    population for simulating the data (Column 1) and for the prior when
    analyzing the simulated data (Column 2).
    The third and fourth columns of plots show true versus estimated #1
    when using all characters (Column 3) or only variable characters (Column 4).
}
\newcommand{\diffusesimsfourcolumndescription}{\diffusesimsfourcolumndescriptiontemplate{values}}
\newcommand{\diffusesimsfourcolumndescriptionmodels}{\diffusesimsfourcolumndescriptiontemplate{models}}
\newcommand{\diffusesimsonlyfourcolumndescriptiontemplate}[1]{The first column of
    plots shows the distribution on the relative effective size of the ancestral
    population under which the data were simulated, and
    the second and third columns of plots show true versus estimated #1
    when using all characters (Column 2) or only variable characters (Column 3).
}
\newcommand{\diffusesimsonlyfourcolumndescription}{\diffusesimsonlyfourcolumndescriptiontemplate{values}}
\newcommand{\diffusesimsonlyfourcolumndescriptionmodels}{\diffusesimsonlyfourcolumndescriptiontemplate{models}}
\newcommand{\validationfiguregriddescription}{The first four columns show the
    results from different distributions on the relative effective size of the
    ancestral population, decreasing in variance from left to right.
    The fifth column shows results when the effective size
    ($\epopsize{}\murate{}$) of all populations is fixed to 0.002.
    For the first two and last two rows, the simulated character matrix for
    each population had 500,000 and 100,000 characters, respectively.
    The first and third rows show the results of analyses using all characters,
    whereas the second and fourth rows show the results when only variable
    characters are used.
}
\newcommand{\missingdatafigurecolumndescription}{The columns, from left to
    right, show the results when each simulated 500,000-character matrix has
    approximately 0\%, 10\%, 25\%, and 50\% missing cells.
}
\newcommand{\comparisoncolumndescription}[2]{
    For comparison, the first column shows the results of the 500 \datasets from
    Figure~#1\ref{#2}; the remaining columns show the results of
    100 \datasets.
}
\newcommand{\missingdatafigurerowdescription}{The rows show the results when
    (top) all sites and (bottom) only variable sites are analyzed.
}
\newcommand{\filtereddatafigurecolumndescription}{The columns, from left to
    right, show the results when each simulated
    500,000-character \dataset has a probability of 100\%, 80\%, 60\%, and 40\%
    of sampling each simulated singleton pattern.
    E.g., each character matrix analyzed in the far right column is missing
    approximately 60\% of characters where all but one gene copy has the same
    allele.
}
\newcommand{\filtereddatafigurerowdescription}{The rows show the results when
    (top) all sites and (bottom) only variable sites are analyzed.
}
\newcommand{\weusedmatplotlib}{We generated the plots using matplotlib Version
    2.0.0 \citep{matplotlib}.}
\newcommand{\weusedggplot}{We generated the plots with ggplot2 Version 2.2.1
    \citep{ggplot2}.}
\newcommand{\weusedggridges}{We generated the plots with ggridges Version 0.4.1
    \citep{ggridges041} and ggplot2 Version 2.2.1 \citep{ggplot2}.}
\newcommand{\neventplotannotations}{For each plot,
    the proportion of \datasets for which the number of events with the largest
    posterior probability matched the true number of events---$p(\hat{\nevents}
    = \nevents)$---is shown in the upper left corner,
    the median posterior probability of the correct number of events across all
    \datasets---$\widetilde{p(\nevents|\alldata)}$---is shown in the upper
    right corner, and
    the proportion of \datasets for which the true divergence model was
    included in the 95\% credible set---$p(\nevents \in
    \textrm{CS})$---is shown in the lower right.
}
\newcommand{\modelplotannotations}{Each model is represented along the plot
    axes by three integers that indicate the event category of each comparison
    (e.g., 011 represents the model in which the second and third comparison
    share the same event time that is distinct from the first).
    The estimates are based on the model with the maximum \textit{a posteriori}
    probability (MAP).
    For each plot,
    the proportion of \datasets for which the MAP model matched the true
    model---$p(\hat{\etimesets} = \etimesets)$---is shown in the upper left
    corner, and the median posterior probability of the correct model across
    all \datasets---$\widetilde{p(\etimesets|\alldata)}$---is shown in the
    upper right corner.
}
\newcommand{\accuracyscatterplotannotations}[1]{For each plot, the
    root-mean-square error (RMSE) and the proportion of estimates for which the
    95\% credible interval contained the true value---$p(#1 \in
    \textrm{CI})$---is given.
}
\newcommand{\neventsshadingdescription}{The number of \datasets that fall
    within each possible cell of true versus estimated numbers of events is
    shown, and cells with more \datasets are shaded darker.
}

%% macro to make long strings breakable over lines
\makeatletter
\def\breakable#1{\xHyphen@te#1$\unskip}
\def\xHyphen@te{\@ifnextchar${\@gobble}{\sw@p{\allowbreak{}\xHyphen@te}}}
% \def\xHyphen@te{\@ifnextchar${\@gobble}{\sw@p{\hskip 0pt plus 1pt\xHyphen@te}}}
\def\sw@p#1#2{#2#1}
\makeatother
