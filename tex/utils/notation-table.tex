\begin{table}[htbp]
    \sffamily
    \small
    \rowcolors{2}{}{mygray}
    \addtolength{\tabcolsep}{-0.1cm}
\caption{
    A key to some of the notation used in the text.
}
    \centering
    \begin{tabular}{ l >{\raggedright\hangindent=0.5cm}m{14cm} }
        \toprule
        \textbf{Symbol} & \textbf{Description} \tn
        \midrule
        \ncomparisons{} & The number of comparisons (or taxa); can be an
        arbitrary mix of populations (comparing timing of demographic change)
        and/or pairs of population (comparing timing of divergence).
        \tn
        \nevents{} & The number of events (unique times) across the population
        pairs being compared.
        \tn
        \comparisonetime[i] & The time in the past when either a population $i$
        had a change in effective size or population pair $i$ diverged.
        \tn
        \etime & A unique event time at which one or more comparisons
        experienced a demographic change or divergence.
        \tn
        \etimemodel & The event-time model, which comprises the event
        times and the mapping of comparisons to those times.
        \tn
        \etimes & All of the unique event times in the model
            ($\etimes = \etime[1], \ldots, \etime[\nevents]$).
        \tn
        \etimesets & The mapping of comparisons to events.
        \tn
        % \basedistribution & The base distribution of the Dirichlet process.
        % \tn
        \concentration & The concentration parameter of the Dirichlet process.
        \tn
        \allelecount, \redallelecount & The number of copies of a locus sampled
            from a population, and the number of those copies that are the ``red''
            allele.
            \tn
        \leafallelecounts, \leafredallelecounts & The allele counts from 
        a comparison (one or two populations).
            \tn
        \comparisondata[i] & The allele counts across all characters from
            comparison $i$. I.e., all of the characters being analyzed for
            comparison $i$.
            \tn
        \nloci & The number of characters collected from a taxon (comparison).
        \tn
        \alldata & All of the data being analyzed, i.e., the
            character matrices from all population pairs.
        \tn
        \genetree & A gene tree with branch lengths.
        \tn
        \murate & The rate of mutation.
        \tn
        \rgmurate & Relative rate of mutating from the ``red'' to ``green'' state.
        \tn
        \grmurate & Relative rate of mutating from the ``green'' to ``red'' state.
        \tn
        \gfreq & The stationary frequency of the ``green'' state.
        \tn
        \epopsize[\descendantpopindex{}] &
            The effective size of a descendant population.
        \tn
        \epopsize[\rootpopindex] & The effective size of the root (ancestral) population.
        \tn
        \rootrelativepopsize & The relative effective population size of the root (ancestral)
        population; relative to the mean of the effective sizes of the descendant populations.
        \tn
        \comparisonpopsizes & Shorthand notation for all effective population
        sizes for a comparison (ancestral and one or two descendant
        populations).
        \tn
        \sptree & The species tree for a pair of populations. This comprises
        the effective population sizes and the time of demographic change or
        divergence.
            \tn
        \bottomrule
    \end{tabular}
    \label{table:notation}
\end{table}
